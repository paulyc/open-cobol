@verbatim


# Value: any string
name: "GnuCOBOL"

# Value: enum
standard-define			0
#        CB_STD_OC = 0,
#        CB_STD_MF,
#        CB_STD_IBM,
#        CB_STD_MVS,
#        CB_STD_BS2000,
#        CB_STD_ACU,
#        CB_STD_85,
#        CB_STD_2002,
#        CB_STD_2014

# Value: int
tab-width:			8
text-column:			72
# Maximum word-length for COBOL words / Programmer defined words
# Be aware that GC checks the word length against COB_MAX_WORDLEN
# first (currently 61)
word-length:			31

# Maximum literal size in general
literal-length:			8191

# Maximum numeric literal size
numeric-literal-length:	61

# Value: 'mf', 'ibm'
#
assign-clause:			mf

# If yes, file names are resolved at run time using
# environment variables.
# For example, given ASSIGN TO "DATAFILE", the file name will be
#  1. the value of environment variable 'DD_DATAFILE' or
#  2. the value of environment variable 'dd_DATAFILE' or
#  3. the value of environment variable 'DATAFILE' or
#  4. the literal "DATAFILE"
# If no, the value of the assign clause is the file name.
#
filename-mapping:		yes

# Alternate formatting of numeric fields
pretty-display:			yes

# Allow complex OCCURS DEPENDING ON
complex-odo:			no

# Allow REDEFINES to other than last equal level number
indirect-redefines:		no

# Binary byte size - defines the allocated bytes according to PIC
# Value:         signed  unsigned  bytes
#                ------  --------  -----
# '2-4-8'        1 -  4    same        2
#                5 -  9    same        4
#               10 - 18    same        8
#
# '1-2-4-8'      1 -  2    same        1
#                3 -  4    same        2
#                5 -  9    same        4
#               10 - 18    same        8
#
# '1--8'         1 -  2    1 -  2      1
#                3 -  4    3 -  4      2
#                5 -  6    5 -  7      3
#                7 -  9    8 -  9      4
#               10 - 11   10 - 12      5
#               12 - 14   13 - 14      6
#               15 - 16   15 - 16      7
#               17 - 18   17 - 18      8
#
binary-size:			1-2-4-8

# Numeric truncation according to ANSI
binary-truncate:		yes

# Binary byte order
# Value: 'native', 'big-endian'
binary-byteorder:		big-endian

# Allow larger REDEFINES items
larger-redefines-ok:		no

# Allow certain syntax variations (eg. REDEFINES position)
relax-syntax-checks:		no

# Perform type OSVS - If yes, the exit point of any currently
# executing perform is recognized if reached.
perform-osvs:			no

# If yes, linkage-section items remain allocated
# between invocations.
sticky-linkage:			no

# If yes, allow non-matching level numbers
relax-level-hierarchy:		no

# Allow Hex 'F' for NUMERIC test of signed PACKED DECIMAL field
hostsign:			no

# If yes, set WITH UPDATE clause as default for ACCEPT dest-item,
# except if WITH NO UPDATE clause is used
accept-update:			no

# If yes, set WITH AUTO clause as default for ACCEPT dest-item,
# except if WITH TAB clause is used
accept-auto:			no

# If yes, DISPLAYs and ACCEPTs are, by default, done on the CRT (i.e., using
# curses).
console-is-crt:			no

# If yes, allow redefinition of the current program's name. This prevents its
# use in a prototype-format CALL/CANCEL statement.
program-name-redefinition:	yes

# Dialect features
# Value: 'ok', 'warning', 'archaic', 'obsolete', 'skip', 'ignore', 'error',
#        'unconformable'

alter-statement:			obsolete
comment-paragraphs:			obsolete
call-overflow:				archaic
data-records-clause:			obsolete
debugging-line:				ok
eject-statement:			skip
entry-statement:			obsolete
goto-statement-without-name:		obsolete
label-records-clause:			obsolete
memory-size-clause:			obsolete
move-noninteger-to-alphanumeric:	error
multiple-file-tape-clause:		obsolete
next-sentence-phrase:			archaic
odo-without-to:				warning
padding-character-clause:		obsolete
section-segments:			ignore
stop-literal-statement:			obsolete
synchronized-clause:			ok
top-level-occurs-clause:		ok
value-of-clause:			obsolete
numeric-boolean:			ok
hexadecimal-boolean:			ok
national-literals:			ok
hexadecimal-national-literals:			ok
acucobol-literals:			unconformable
word-continuation:			warning
not-exception-before-exception:		ok
accept-display-extensions:		ok
renames-uncommon-levels:		ok
program-prototypes:			ok

# If yes, all the reserved words must be specified in a list of reserved:
# entries; the default reserved word list will not be used.
specify-all-reserved: no

# not-reserved:
# Value: Word to be taken out of the reserved words list
# (case independent)
# Words that are in the (proposed) standard but may conflict

# reserved:
# Value: Word to make up reserved words list (case independent)
# All reserved entries listed will replace entire default reserved words list.
#   Words ending with * will be treated as context-sensitive words. This will be
# ignored if GnuCOBOL uses that word as a reserved word.
#   Entries of the form word-1=word-2 define word-1 as an alias for default
# reserved word word-2. No spaces are allowed around the equal sign.
reserved:	AUTO-SKIP=AUTO
reserved:	AUTOTERMINATE=AUTO
reserved:	BACKGROUND-COLOUR=BACKGROUND-COLOR
reserved:	BEEP=BELL
reserved:	BINARY-INT=BINARY-LONG
reserved:	BINARY-LONG-LONG=BINARY-DOUBLE
reserved:	EMPTY-CHECK=REQUIRED
reserved:	EQUALS=EQUAL
reserved:	FOREGROUND-COLOUR=FOREGROUND-COLOR
reserved:	INITIALISE=INITIALIZE
reserved:	INITIALISED=INITIALIZED
reserved:	LENGTH-CHECK=FULL
reserved:	ORGANISATION=ORGANIZATION
reserved:	SYNCHRONISED=SYNCHRONIZED
reserved:	TIMEOUT=TIME-OUT
@end verbatim

