@verbatim
GnuCOBOL compiler for most COBOL dialects with lots of extensions

Usage: cobc [options]... file...

Options:
  -h, -help             display this help and exit
  -V, -version          display compiler version and exit
  -i, -info             display compiler information (build/environment)
  -v, -verbose          display the commands invoked by the compiler
  -vv                   display compiler version and the commands
                        invoked by the compiler
  -q, -brief            reduced displays, commands invoked not shown
  -x                    build an executable program
  -m                    build a dynamically loadable module (default)
  -j [<args>], -job[=<args>] run program after build, passing <args>
  -std=<dialect>        warnings/features for a specific dialect
                        <dialect> can be one of:
                        cobol2014, cobol2002, cobol85, default,
                        ibm, mvs, bs2000, mf, acu;
                        see configuration files in directory config
  -F, -free             use free source format
  -fixed                use fixed source format (default)
  -O, -O2, -Os          enable optimization
  -g                    enable C compiler debug / stack check / trace
  -d, -debug            enable all run-time error checking
  -o <file>             place the output into <file>
  -b                    combine all input files into a single
                        dynamically loadable module
  -E                    preprocess only; do not compile or link
  -C                    translation only; convert COBOL to C
  -S                    compile only; output assembly file
  -c                    compile and assemble, but do not link
  -T <file>             generate and place a wide program listing into <file>
  -t <file>             generate and place a program listing into <file>
  --tlines=<lines>      specify lines per page in listing, default = 55
  -P[=<dir or file>]    generate preprocessed program listing (.lst)
  -Xref                 generate cross reference through 'cobxref'
                        (V. Coen's 'cobxref' must be in path)
  -I <directory>        add <directory> to copy/include search path
  -L <directory>        add <directory> to library search path
  -l <lib>              link the library <lib>
  -A <options>          add <options> to the C compile phase
  -Q <options>          add <options> to the C link phase
  -D <define>           define <define> for COBOL compilation
  -K <entry>            generate CALL to <entry> as static
  -conf=<file>          user-defined dialect configuration; see -std
  -list-reserved        display reserved words
  -list-intrinsics      display intrinsic functions
  -list-mnemonics       display mnemonic names
  -list-system          display system routines
  -save-temps[=<dir>]   save intermediate files
                        - default: current directory
  -ext <extension>      add file extension for resolving COPY

  -W                    enable all warnings
  -Wall                 enable most warnings (all except as noted below)
  -Wno-<warning>        disable warning enabled by -W or -Wall
  -Wobsolete            warn if obsolete features are used
  -Warchaic             warn if archaic features are used
  -Wredefinition        warn incompatible redefinition of data items
  -Wconstant            warn inconsistent constant
  -Woverlap             warn overlapping MOVE items
  -Wparentheses         warn lack of parentheses around AND within OR
  -Wstrict-typing       warn type mismatch strictly
  -Wimplicit-define     warn implicitly defined data items
  -Wcorresponding       warn CORRESPONDING with no matching items
  -Wexternal-value      warn EXTERNAL item with VALUE clause
  -Wprototypes          warn missing FUNCTION prototypes/definitions
  -Wcall-params         warn non 01/77 items for CALL params
			- NOT set with -Wall
  -Wcolumn-overflow     warn text after program-text area, FIXED format
			- NOT set with -Wall
  -Wterminator          warn lack of scope terminator END-XXX
			- NOT set with -Wall
  -Wtruncate            warn possible field truncation
			- NOT set with -Wall
  -Wlinkage             warn dangling LINKAGE items
			- NOT set with -Wall
  -Wunreachable         warn unreachable statements
			- NOT set with -Wall

  -fsign=[ASCII|EBCDIC] define display sign representation
			- default: machine native
  -ffold-copy=[UPPER|LOWER] fold COPY subject to value
			- default: no transformation
  -ffold-call=[UPPER|LOWER] fold PROGRAM-ID, CALL, CANCEL subject to value
			- default: no transformation
  -fdefaultbyte=0..255  initialize fields without VALUE to decimal value
			- default: initialize to picture
  -fintrinsics=[ALL|intrinsic function name(,name,...)] intrinsics to be used without FUNCTION keyword
  -ftrace               generate trace code
			- executed SECTION/PARAGRAPH
  -ftraceall            generate trace code
			- executed SECTION/PARAGRAPH/STATEMENTS
			- turned on by -debug
  -fsyntax-only         syntax error checking only; don't emit any output
  -fdebugging-line      enable debugging lines
			- 'D' in indicator column or floating >>D
  -fsource-location     generate source location code
			- turned on by -debug/-g/-ftraceall
  -fimplicit-init       automatic initialization of the COBOL runtime system
  -fstack-check         PERFORM stack checking
			- turned on by -debug or -g
  -fsyntax-extension    allow syntax extensions
			- eg. switch name SW1, etc.
  -fwrite-after         use AFTER 1 for WRITE of LINE SEQUENTIAL
			- default: BEFORE 1
  -fmfcomment           '*' or '/' in column 1 treated as comment
			- FIXED format only
  -facucomment          '$' in indicator area treated as '*',
			'|' treated as floating comment
  -fnotrunc             allow numeric field overflow
			- non-ANSI behaviour
  -fodoslide            adjust items following OCCURS DEPENDING
			- requires implicit/explicit relaxed syntax
  -fsingle-quote        use a single quote (apostrophe) for QUOTE
			- default: double quote
  -frecursive-check     check recursive program call
  -foptional-file       treat all files as OPTIONAL
			- unless NOT OPTIONAL specified

  -ftab-width=<number>  set number of spaces that are asumed for tabs
  -ftext-column=<number> set right margin for source (fixed format only)
  -fword-length=<number> maximum word-length for COBOL words / Programmer defined words
  -fliteral-length=<number> maximum literal size in general
  -fnumeric-literal-length=<number> maximum numeric literal size
  -fassign-clause=<value> set way of interpreting ASSIGN
  -fbinary-size=<value> binary byte size - defines the allocated bytes according to PIC
  -fbinary-byteorder=<value> binary byte order
  -fstandard-define=<value> 
  -ffilename-mapping    resolve file names at run time using environment variables.
  -fpretty-display      alternate formatting of numeric fields
  -fbinary-truncate     numeric truncation according to ANSI
  -fcomplex-odo         allow complex OCCURS DEPENDING ON
  -findirect-redefines  allow REDEFINES to other than last equal level number
  -flarger-redefines-ok allow larger REDEFINES items
  -frelax-syntax-checks allow certain syntax variations (eg. REDEFINES position)
  -fperform-osvs        exit point of any currently executing perform is recognized if reached
  -fsticky-linkage      linkage-section items remain allocated between invocations
  -frelax-level-hierarchy allow non-matching level numbers
  -fhostsign            allow hexadecimal value 'F' for NUMERIC test of signed PACKED DECIMAL field
  -faccept-update       set WITH UPDATE clause as default for ACCEPT dest-item, except if WITH NO UPDATE clause is used
  -faccept-auto         set WITH AUTO clause as default for ACCEPT dest-item, except if WITH TAB clause is used
  -fconsole-is-crt      assume CONSOLE IS CRT if not set otherwise
  -fprogram-name-redefinition program names don't lead to a reserved identifier
  -fspecify-all-reserved specify all reserved words
  -fcomment-paragraphs=<support> comment paragraphs in IDENTIFICATION DIVISION (AUTHOR, DATE-WRITTEN, ...)
  -fmemory-size-clause=<support> MEMORY-SIZE clause
  -fmultiple-file-tape-clause=<support> MULTIPLE-FILE-TAPE clause
  -flabel-records-clause=<support> LABEL-RECORDS clause
  -fvalue-of-clause=<support> VALUE-OF clause
  -fdata-records-clause=<support> DATA-RECORDS clause
  -ftop-level-occurs-clause=<support> OCCURS clause on top-level
  -fsynchronized-clause=<support> SYNCHRONIZED clause
  -fgoto-statement-without-name=<support> GOTO statement without name
  -fstop-literal-statement=<support> STOP-LITERAL statement
  -fdebugging-line=<support> DEBUGGING MODE and indicator 'D'
  -fpadding-character-clause=<support> PADDING CHARACTER clause
  -fnext-sentence-phrase=<support> NEXT SENTENCE phrase
  -feject-statement=<support> EJECT statement
  -fentry-statement=<support> ENTRY statement
  -fmove-noninteger-to-alphanumeric=<support> move noninteger to alphanumeric
  -fodo-without-to=<support> OCCURS DEPENDING ON without to
  -fsection-segments=<support> section segments
  -falter-statement=<support> ALTER statement
  -fcall-overflow=<support> OVERFLOW clause for CALL
  -fnumeric-boolean=<support> boolean literals (b'0001')
  -facucobol-literals=<support> ACUCOBOL-GT literals (#B #O #H #X)
  -fword-continuation=<support> continuation of COBOL words
  -fnot-exception-before-exception=<support> NOT ON EXCEPTION before ON EXCEPTION
  -faccept-display-extensions=<support> extensions to ACCEPT and DISPLAY
  -frenames-uncommon-levels=<support> RENAMES of 01-, 66- and 77-level items
  -fprogram-prototypes=<support> CALL/CANCEL with program-prototype-name
	where <support> is one of the following:
	'ok', 'warning', 'archaic', 'obsolete', 'skip', 'ignore', 'error', 'unconformable'
  -fnot-reserved=<word> word to be taken out of the reserved words list
  -freserved=<word>     word to be added to reserved words list
  -freserved=<word>:<alias> word to be added to reserved words list as alias


Report bugs to: bug-gnucobol@gnu.org
or (preferably) use the issue tracker via the home page.
GnuCOBOL home page: <http://www.gnu.org/software/gnucobol/>
General help using GNU software: <http://www.gnu.org/gethelp/>
@end verbatim

