@section Intrinsic REXX scripting

@*
REXX scripting is an optional build feature of GnuCOBOL.
@*
Requires Regina REXX and @code{./configure --with-rexx} during build.
@*
Adds new Intrinsics functions, @code{FUNCTION REXX(script [,arg,...])} and
@code{FUNCTION REXX-UNRESTRICTED(...)}.

These functions accept character data as the script, and unlimited arguments
(also character data), and returns an ALPHANUMERIC field from the REXX
interpreter.  Reference modification is allowed.

@* 
Includes a @code{rexxapi.cpy} copybook that defines error number constants and
an @code{EXTERNAL} variable, @code{SCRIPT-RETURN-CODE} that provides status
values along with a definition of the EXTERNAL status code identifier.
@code{rexxapi.cpy} will be installed to a system specific location that is part
of the default search path for the @code{COPY} directive.

@section REXX examples

Example:

@verbatim
    move rexx("variable = arg(1) * arg(2); return variable", 6, 7)
      to answer
@end verbatim

Parameter 1 is the script, all other arguments (unlimited) are passed to REXX
as ARG values. Return value is ALPHANUMERIC.

Reference modification is allowed.

    @code{display rexx("return xrange(0,a)")(1:10)}

As this is defined as an alphanumeric function, use @code{FUNCTION NUMVAL} to
use any REXX results in computations.

    @code{compute answer = numval(rexx("return 6 * arg(1)", 6)) + 6}

Regina REXX RESTRICTED mode can be disabled with @code{FUNCTION REXX-UNRESTRICTED}

    @code{display rexx-unrestricted("lineout('file', 'string')")}

The line above will allow a script to write to file. Normally IO will return an
error code from REXX as the @code{LINEOUT} feature is disabled in the default
RESTRICTED mode. 

@section Optional build feature

This is an optional build feature.

@code{./configure --with-rexx} to build a compiler with Regina REXX
scripting.

Requires Regina REXX with the @code{libregina} dynamic shared library in the
linker search path.
