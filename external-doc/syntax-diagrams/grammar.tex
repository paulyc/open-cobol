\documentclass[a4paper,oneside,svgnames]{scrbook}

\usepackage{microtype}
\usepackage[toc,page]{appendix}
\usepackage{booktabs}
\usepackage{textcomp}
\usepackage{roboto}
\usepackage[dvipsnames]{xcolor}
\usepackage{afterpage}
\usepackage{tikz}
\usepackage[printwatermark]{xwatermark}
\usepackage[tocindentauto]{tocstyle}
\usetocstyle{standard}
\usepackage[most]{tcolorbox}
\usepackage[pdfauthor={Edward Hart},
            pdftitle={The GnuCOBOL 2.0 Grammar},
            pdfkeywords={GnuCOBOL,COBOL,manual,grammar,guide},
            colorlinks]{hyperref}

% \addtokomafont{section}{\clearpage}
% \addtokomafont{subsection}{\clearpage}

\renewcommand{\familydefault}{\sfdefault}

\newcommand{\newlexicalelement}[1]{%
  \newcounter{#1}[subsection]%
  \expandafter\newcommand\csname #1\endcsname{%
    \stepcounter{#1}%
    #1-\arabic{#1}\ {}%
  }%
}

\newcommand{\newlexicalelementwithname}[2]{%
  \newcounter{#1}[subsection]%
  \expandafter\newcommand\csname #1\endcsname{%
    \stepcounter{#1}%
    #2-\arabic{#1}\ {}%
  }%
}

\newcounter{format}[subsection]
\newcommand{\format}[1]{%
  \stepcounter{format}%
  \paragraph{Format \arabic{format}\vspace{1em}\ {}(#1)}\ {}\newline%
}

\newcommand{\key}[1]{\underline{#1}}
\newcommand{\directiveindicator}[0]{$>\!>$}
\newcommand{\deleted}[1]{%
  \colorbox{red!75}{#1}}
\newcommand{\archaic}[1]{%
  \colorbox{pink}{#1}}
\newcommand{\obsolete}[1]{%
  \colorbox{red!60}{#1}}
\newcommand{\xopen}[1]{%
  \colorbox{green!75}{#1}}
\newcommand{\gnucobol}[1]{%
  \colorbox{orange!75}{#1}}
\newcommand{\miscext}[1]{%
  \colorbox{blue!50}{#1}}
\newcommand{\pending}[1]{%
  \textcolor{gray!75}{#1}}
\newcommand{\standard}[1]{%
  \colorbox{white}{#1}}

\newenvironment{0-1}{$\left[ \begin{tabular}{@{}l@{}}}{\end{tabular} \right]$}
\newenvironment{0+}{$\left[\left| \begin{tabular}{@{}l@{}}}{\end{tabular} \right|\right]$}
\newenvironment{1=}{$\left\{ \begin{tabular}{@{}l@{}}}{\end{tabular} \right\}$}
\newenvironment{1+}{$\left\{\left| \begin{tabular}{@{}l@{}}}{\end{tabular} \right|\right\}$}

\tcbset{
  frame code={}
  halign title=flush center,
  left=40mm,
  right=40mm,
  top=10pt,
  bottom=10pt,
  colback=Maroon,
  colupper=white,
  left skip=40mm,
  grow to left by=40mm,
  right skip=40mm,
  grow to right by=40mm,
  width=\paperwidth,
  text width=\textwidth,
  arc=0pt,outer arc=0pt,
}

\begin{document}

\newsavebox\mybox
\savebox\mybox{\tikz[color=gray,opacity=0.3]\node{DRAFT};}
% \newwatermark*[
%   allpages,
%   angle=45,
%   scale=12,
%   xpos=-42,
%   ypos=39
% ]{\usebox\mybox}

\newlexicalelementwithname{arithmeticexpression}{arithmetic-expression}
\newlexicalelementwithname{conditionname}{condition-name}
\newlexicalelementwithname{filename}{file-name}
\newlexicalelementwithname{functionname}{function-name}
\newlexicalelementwithname{imperativestatement}{imperative-statement}
\newlexicalelementwithname{cobolindexname}{index-name}
\newlexicalelementwithname{libraryname}{library-name}
\newlexicalelementwithname{mnemonicname}{mnemonic-name}
\newlexicalelementwithname{procedurename}{procedure-name}
\newlexicalelementwithname{pseudotext}{pseudo-text}
\newlexicalelementwithname{recordname}{record-name}
\newlexicalelementwithname{reportname}{report-name}
\newlexicalelementwithname{sourcetext}{source-text}
\newlexicalelementwithname{textname}{text-name}
\newlexicalelement{argument}
\newlexicalelement{condition}
\newlexicalelement{expression}
\newlexicalelement{identifier}
\newlexicalelement{integer}
\newlexicalelement{literal}
\newlexicalelementwithname{switchstatusname}{switch-status-name}
\newlexicalelementwithname{switchname}{switch-name}
\newlexicalelementwithname{systemname}{system-name}
\newlexicalelementwithname{computername}{computer-name}

\frontmatter

\begin{titlepage}
  \pagecolor{Maroon!75}\afterpage{\nopagecolor}

  \centering

  \vfill

  \begin{tcolorbox}[bottom=7.5pt]
    {\centering\fontsize{32pt}{0cm}\bfseries
      The GnuCOBOL 2.0 Grammar \par
    }
    \vspace{1em}
    {\centering\scshape\LARGE for r997 \par}
  \end{tcolorbox}

  \vspace{3cm}
  \begin{tcolorbox}
    {\LARGE\itshape Edward Hart\par}
    \vspace{5pt}
    {\large\href{mailto:edward.dan.hart@gmail.com}{\color{white}{edward.dan.hart@gmail.com}}}
  \end{tcolorbox}

  \vfill

  % Bottom of the page
  \begin{tcolorbox}[top=3pt,bottom=3pt]
    {\Large \today\par}
  \end{tcolorbox}
\end{titlepage}

\topskip0pt
\vspace*{\fill}

COBOL is an industry language and is not the property of any company or group of companies, or of any organisation or group of organisations.

No warranty, expressed or implied, is made by any contributor, or by the CODASYL COBOL Committee,\footnote{The CODASYL COBOL committee was dissolved in 1992. Its work was continued by ANSI X3J4 and then INCITS PL22.4, which was itself dissolved in 2015.} as to the accuracy and functioning of the programming system and language. Moreover, no responsibility is assumed by any contributor, or by the committee, in connection therewith.

The authors and copyright holders of the copyrighted materials used herein are:
\begin{itemize}
\item FLOW-MATIC (trademark of Sperry Rand Corporation), Programming for the UNIVAC \textregistered{} I and II, Data Automation Systems, copyrighted 1958, 1959 by Sperry Rand Corporation,\footnote{Sperry Rand's computer business is now part of Unisys.}
\item IBM Commercial Translator, Form No. F28-8013, copyrighted 1959 by IBM, and
\item FACT, DSI 27A5260-2760, copyrighted 1960 by Minneapolis-Honeywell.
\end{itemize}

They have specially authorised the use of this material, in whole or in part, in the COBOL specifications. Such authorisation extends to the reproduction and use of COBOL specifications in programming manuals or similar publications.

\vfill

\begin{center}
  This work is typeset in Roboto.
\end{center}


\vfill

\begin{center}
  Copyright \copyright{} \the\year{} Edward Hart

\vspace{5pt}

Permission is granted to copy, distribute and\slash{}or modify this document under the terms of the GNU Free Documentation License, Version 1.3 or any later version published by the Free Software Foundation; with no Invariant Sections, no Front-Cover Texts, and no Back-Cover Texts. Your attention is drawn to the copy of the license in Appendix \ref{label_fdl}.
\vspace{5pt}

  The moral rights of the author have been asserted.
\end{center}

\tableofcontents

\chapter{Foreword}

This document describes the syntax of COBOL as supported by GnuCOBOL. It is hoped it will complement Gary Cutler's 2013 \textit{GnuCOBOL Programmer's Guide} which does not document recent features added to GnuCOBOL. It is also formatted in \LaTeX, so that everything looks a bit prettier.

The syntax diagrams were transcribed from GnuCOBOL's parsers. It thus replicates some unusual syntax rules and misses some syntax rules implemented outside the parser. For example, the obsolete identification division comment paragraphs are allowed in any order and the syntax of \hyperref[file-control-entry]{a file-control entry} does not distinguish between for SEQUENTIAL, INDEXED and RELATIVE organisations.

This is a draft and so has many flaws. Designed to document features the \textit{Programmer's Guide} does not, it strangely lacks a list of these new features. Important syntax rules which cannot be contained in syntax diagrams are missing. There are no definitions of fundamental objects such as conditions and identifiers. If people find this document useful, I will try to fix these shortcomings.

\mainmatter

\renewcommand{\arraystretch}{1.2}
\setlength{\parskip}{\baselineskip}

\chapter{Key}

\begin{table}[!h]
  \centering
  \begin{tabular}[!h]{p{0.4\textwidth} p{0.5\textwidth}}
    \toprule
    Element & Notes \\ \midrule
    Braces, $\left\{\ {}\right\}$ & One element within the braces must be selected. \\
    Brackets, $\left[\ {}\right]$ & One or zero elements within the brackets must be selected. \\
    Vertical lines, $\left|\ {}\right|$ & Each element may be selected once and in any order; if within braces, at least one element must be selected. \\
    OPTIONAL-RESERVED-WORD & \\
    \key{MANDATORY-RESERVED-WORD} & Mandatory reserved words in brackets are often used instead of optional reserved words to indicate an optional feature. \\
    \deleted{Deleted element} & These elements were previously in the COBOL standard but have since been deleted. Their use is strongly discouraged. \\
    \archaic{Archaic element} & These elements remain in the standard, but their use is considered poor style and is strongly discouraged. \\
    \obsolete{Obsolete element} & These elements are slated to be deleted from the standard. Their use is strongly discouraged. \\
    \xopen{X\slash{}Open extension} & \\
    \gnucobol{GnuCOBOL-only extension} & \\
    \miscext{Miscellaneous extension} & An extension which may have come from COBOL dialects by Micro Focus, IBM, AcuCorp, Ryan-McFarland, Fujitsu or Microsoft. \\
    \pending{Unimplemented element} & These elements are recognised by GnuCOBOL, but result in errors. \\ \bottomrule
  \end{tabular}
\end{table}
\chapter{Compiler directives}

\section{D directive}

\gnucobol{\directiveindicator\key{D} \sourcetext}

\section{COPY statement}

\begin{1=}
  \key{COPY} \\
  \deleted{\key{INCLUDE}}
\end{1=}
\begin{1=}
  \literal \\
  \textname \\
\end{1=}
\begin{0-1}
  \begin{1=}
    \key{IN} \\
    \key{OF}
  \end{1=}
  \begin{1=}
    \literal \\
    \libraryname
  \end{1=}
\end{0-1}

\begin{0-1}
  \key{SUPPRESS} PRINTING
\end{0-1}

\begin{0-1}
  \key{REPLACING}
  \begin{1=}
    \begin{1=}
      == \pseudotext == \\
      \identifier \\
      \literal
    \end{1=}
    \key{BY}
    \begin{1=}
      == \pseudotext == \\
      \identifier \\
      \literal
    \end{1=} \\

    \begin{1=}
      \key{LEADING} \\
      \key{TRAILING}
    \end{1=}
    == partial-word-1 ==
    \key{BY}
    == partial-word-2 ==
  \end{1=}\ldots
\end{0-1}
.

\section{DEFINE directive}

\begin{1=}
  \directiveindicator \\
  \miscext{\textdollar}
\end{1=}
\key{DEFINE}
\gnucobol{
  \begin{0-1}
    \key{CONSTANT}
  \end{0-1}
}
compilation-variable-1 AS
\begin{1=}
  \begin{1=}
      \literal \\
      \key{PARAMETER}
  \end{1=}
  \begin{0-1}
    \key{OVERRIDE}
  \end{0-1} \\
  \key{OFF}
\end{1=}

\section{DISPLAY directive}

\miscext{
  \begin{1=}
    \directiveindicator \\
    \textdollar
  \end{1=}
  \key{DISPLAY} \sourcetext
}

\section{IF directive}

\begin{1=}
  \directiveindicator \\
  \miscext{\textdollar}
\end{1=}
\key{IF} compilation-variable-1 IS NOT
\begin{1=}
  \key{DEFINED} \\
  \key{SET} \\
  relation compilation-variable-2
\end{1=}

\sourcetext


\miscext{
  \begin{0-1}
    \begin{1=}
      \gnucobol{\directiveindicator} \\
      \textdollar
    \end{1=}
    \begin{1=}
      \key{ELIF} \\
      \key{ELSE-IF}
    \end{1=}
    \condition
    \sourcetext
  \end{0-1} \ldots
}

\begin{0-1}
  \begin{1=}
    \directiveindicator \\
    \miscext{\textdollar}
  \end{1=}
  \key{ELSE} \sourcetext
\end{0-1}

\begin{0-1}
  \directiveindicator\key{END-IF} \\
  \miscext{\textdollar\key{END}}
\end{0-1}


\section{LEAP-SECOND directive}

\pending{\directiveindicator\key{LEAP-SECOND}}

\section{LISTING directive}

\directiveindicator\key{LISTING}
\begin{1=}
  \key{ON} \\
  \key{OFF}
\end{1=}

\section{PAGE directive}

\directiveindicator\key{PAGE}
\begin{0-1}
  comment-text
\end{0-1}

\section{REPLACE statement}

\format{on}
\key{REPLACE}
\begin{0-1}
  \key{ALSO}
\end{0-1}
\begin{1=}
  \begin{1=}
    == \pseudotext == \\
    \identifier
  \end{1=}
  \key{BY}
  \begin{1=}
    == \pseudotext == \\
    \identifier
  \end{1=} \\

  \begin{1=}
    \key{LEADING} \\
    \key{TRAILING}
  \end{1=}
  == partial-word-1 ==
  \key{BY}
  == partial-word-2 ==
\end{1=}\ldots .

\format{off}
\key{REPLACE}
\begin{0-1}
  \key{LAST}
\end{0-1}
\key{OFF}.

\section{SET directive}

\miscext{
  \begin{1=}
    \gnucobol{\directiveindicator} \\
    \textdollar
  \end{1=}
  \key{SET}
  \begin{1=}
    \gnucobol{
      \key{CONSTANT} compilation-variable-1 AS \literal
    } \\

    \gnucobol{
      compilation-variable-2
      \begin{0-1}
        AS \literal
      \end{0-1}
    } \\

    \key{SOURCEFORMAT} AS \literal \\

    \begin{1=}
      \key{NO-FOLD-COPY-NAME} \\
      \key{NOFOLDCOPYNAME}
    \end{1=} \\

    \begin{1=}
    \key{FOLD-COPY-NAME} \\
    \key{FOLDCOPYNAME}
  \end{1=}
  AS \literal
  \end{1=} \ldots
}

\section{SOURCE directive}

\directiveindicator\key{SOURCE} FORMAT IS
\begin{1=}
  \key{FIXED} \\
  \key{FREE}
\end{1=}

\section{TURN directive}

\pending{
  \directiveindicator\key{TURN}
  \begin{1=}
    exception-name-1
  \end{1=} \ldots
  \begin{0-1}
    \key{ON} \\
    \key{OFF}
  \end{0-1}
  \begin{0-1}
    WITH \key{LOCATION}
  \end{0-1}
}

\section{Miscellaneous directives}

\begin{itemize}
\item \miscext{EJECT}
\item \miscext{\pending{PROCESS}}
\item \miscext{SKIP1}
\item \miscext{SKIP2}
\item \miscext{SKIP3}
\end{itemize}
\chapter{Identification division}

\begin{0-1}
  \begin{1=}
    \miscext{\key{ID}} \\
    \key{IDENTIFICATION}
  \end{1=}
  DIVISION.
\end{0-1}
\newline
\begin{1=}
  function-id-paragraph \\
  program-id-paragraph
\end{1=}
\newline
\deleted{
  \begin{0+}
    \key{AUTHOR}. comment-text. \\
    \key{DATE-WRITTEN}. comment-text. \\
    \key{DATE-MODIFIED}. comment-text. \\
    \key{DATE-COMPILED}. comment-text. \\
    \key{INSTALLATION}. comment-text. \\
    \key{REMARKS}. comment-text. \\
    \key{SECURITY}. comment-text. \\
  \end{0+} \gnucobol{\ldots}
}

\section{PROGRAM-ID paragraph}

\key{PROGRAM-ID}.
\begin{1=}
  program-name-1 \\
  \literal
\end{1=}
\begin{0-1} \key{AS} \literal \end{0-1}

\begin{0-1} IS
  \begin{1=}
    \miscext{\key{EXTERNAL}} \\

    \begin{1+}
      \key{COMMON} \\

      \begin{1=}
        \key{INITIAL} \\
        \key{RECURSIVE}
      \end{1=}
    \end{1+}
  \end{1=}
  PROGRAM
\end{0-1}.

\section{FUNCTION-ID paragraph}

\key{FUNCTION-ID}.
\begin{1=}
  \functionname \\
  \literal
\end{1=}
\begin{0-1} \key{AS} \literal \end{0-1}.

\chapter{Environment division}
\begin{0-1}
  \key{ENVIRONMENT} \key{DIVISION}.
\end{0-1}
\newline
\begin{0-1}
  configuration-section
\end{0-1}
\newline
\begin{0-1}
  input-output-section
\end{0-1}

\section{Configuration section}

\begin{0-1}
  \key{CONFIGURATION} \key{SECTION}.
\end{0-1}
\newline
\begin{0+}
  source-computer-paragraph \\
  object-computer-paragraph
\end{0+}
\newline
\begin{0-1}
  special-names-paragraph
\end{0-1}
\newline
\begin{0-1}
  special-names-entry
\end{0-1}
\newline
\begin{0-1}
  repository-paragraph
\end{0-1}

\subsection{SOURCE-COMPUTER paragraph}

\key{SOURCE-COMPUTER}.
\begin{0-1}
  \begin{1=}
    \computername
  \end{1=}\gnucobol{\ldots}
  \deleted{
    \begin{0-1}
      WITH \key{DEBUGGING} \key{MODE}
    \end{0-1}
  }
  .
\end{0-1}

\subsection{OBJECT-COMPUTER paragraph}
\key{OBJECT-COMPUTER}.

\begin{0-1}
  \begin{0-1}
    \begin{1=}
      \computername
    \end{1=}\gnucobol{\ldots}
  \end{0-1}\\\quad
  \begin{0-1}
    \deleted{
      \key{MEMORY} \key{SIZE} IS \integer
      \begin{1=}
        \key{CHARACTERS} \\
        \key{WORDS}
      \end{1=}
    } \\
    PROGRAM COLLATING \key{SEQUENCE} IS collating-sequence-1 \\
    \deleted{\key{SEGMENT-LIMIT} IS \integer} \\
    CHARACTER \key{CLASSIFICATION} IS
    \begin{1=}
      locale-name-1 \\
      \key{LOCALE} \\
      \key{SYSTEM-DEFAULT} \\
      \key{USER-DEFAULT}
    \end{1=}
  \end{0-1}\ldots\ {}.
\end{0-1}

\subsection{SPECIAL-NAMES paragraph}

\begin{0-1}
  \key{SPECIAL-NAMES}.
\end{0-1}

\begin{0-1}
  \begin{1=}
    mnemonic-name-clause \\
    alphabet-name-clause \\
    symbolic-characters-clause \\
    \key{LOCALE} locale-name-1 IS \literal \\

    \key{CLASS} class-name-1 IS
    \begin{1=}
      \literal
      \begin{0-1}
        \begin{1=}
          \key{THRU} \\
          \key{THROUGH}
        \end{1=}
        \literal
      \end{0-1}
    \end{1=}\ldots \\

    \key{CURRENCY} SIGN IS \literal
    \begin{0-1}
      \pending{WITH \key{PICTURE-SYMBOL} \literal}
    \end{0-1} \\

    \key{DECIMAL-POINT} IS \key{COMMA} \\
    \miscext{\key{NUMERIC} \key{SIGN} IS \key{TRAILING} \key{SEPARATE}} \\
    \key{CURSOR} IS \identifier \\
    \key{CRT} \key{STATUS} IS \identifier \\
    \miscext{\pending{\key{SCREEN-CONTROL} IS \identifier}} \\
    \miscext{\pending{\key{EVENT-STATUS} IS \identifier}}
  \end{1=}\ldots\ {}.
\end{0-1}\ldots

where mnemonic-name-clause is

\mnemonicname
\begin{1=}
  IS \key{CRT} \\
  \integer IS \systemname \\
  \begin{0-1}
    IS \switchname
  \end{0-1}
  \begin{1+}
      \key{ON} STATUS IS \switchstatusname \\
      \key{OFF} STATUS IS \switchstatusname
  \end{1+}
\end{1=}

where alphabet-name-clause is

\key{ALPHABET} alphabet-name-1 IS
\begin{1=}
  \key{ASCII} \\
  \key{EBCDIC} \\
  \key{NATIVE} \\
  \key{STANDARD-1} \\
  \key{STANDARD-2} \\
  \begin{1=}
    \literal
    \begin{0-1}
      \begin{1=}
        \key{THROUGH} \\
        \key{THRU}
      \end{1=}
      \literal \\
      \begin{1=}
        \key{ALSO} \literal
      \end{1=}\ldots
    \end{0-1}
  \end{1=}\ldots
\end{1=}

where symbolic-characters-clause is

\key{SYMBOLIC} CHARACTERS
\begin{1=}
  \begin{1=}
    symbolic-character-name-1
  \end{1=}\ldots
  \begin{1=}
    \key{IS} \\
    \key{ARE}
  \end{1=}
  \begin{1=}
    \integer
  \end{1=}\ldots
\end{1=}\ldots
\begin{0-1}
  \key{IN} \key{WORD}
\end{0-1}

\subsection{REPOSITORY paragraph}

\key{REPOSITORY}.

\begin{0-1}
  \begin{1=}
    \key{FUNCTION}
    \begin{1=}
      \begin{1=}
        \functionname
      \end{1=}\ldots \\

      \key{ALL}
    \end{1=}
    \key{INTRINSIC} \\

    \key{FUNCTION} \functionname
    \begin{0-1}
      \key{AS} \literal
    \end{0-1}
  \end{1=}\ldots\ {}.
\end{0-1}

\section{Input-output section}

\begin{0-1}
  \key{INPUT-OUTPUT} \key{SECTION}.
\end{0-1}\newline
\begin{0-1}
  file-control-paragraph
\end{0-1}\newline
\begin{0-1}
  i-o-control-paragraph
\end{0-1}

\subsection{FILE-CONTROL paragraph}

\begin{0-1}
  \key{FILE-CONTROL}.
\end{0-1}\newline
\begin{0-1}
  file-control-entry
\end{0-1} \ldots

\label{file-control-entry} where file-control-entry is

\key{SELECT}
\begin{0-1}
  \key{OPTIONAL} \\
  \miscext{\key{NOT} \key{OPTIONAL}}
\end{0-1}
\filename
\begin{0-1}
  assign-clause \\
  access-mode-clause \\
  alternative-record-key-clause \\
  collating-sequence-clause \\
  file-status-clause \\
  lock-mode-clause \\
  organization-clause \\
  padding-character-clause \\
  record-delimiter-clause \\
  record-key-clause \\
  relative-key-clause \\
  reserve-clause \\
  sharing-clause
\end{0-1}\ldots\ {}.

where assign-clause is

\ {}\newline
\key{ASSIGN}
\begin{0-1}
  \key{TO} \\
  \key{USING}
\end{0-1}
\miscext{
  \begin{0-1}
    \key{DYNAMIC} \\
    \key{EXTERNAL}
  \end{0-1}
}
\begin{1=}
  \miscext{
    \begin{0-1}
      \key{LINE} \key{ADVANCING} FILE
    \end{0-1}
  }
  \begin{1=}
    \literal \\
    \identifier
  \end{1=}\\

  \begin{1=}
    \key{DISC} \\
    \xopen{\key{DISK}} \\
    \miscext{\key{DISPLAY}} \\
    \miscext{\key{KEYBOARD}} \\
    \miscext{\key{PRINTER-1}} \\
    \xopen{\key{PRINTER}} \\
    \key{PRINT} \\
    \key{RANDOM} \\
    \key{TAPE}
  \end{1=}
  \begin{0-1}
    \literal \\
    \identifier
  \end{0-1}
\end{1=}

where access-mode-clause is

\key{ACCESS} MODE IS
\begin{1=}
  \key{SEQUENTIAL} \\
  \key{DYNAMIC} \\
  \key{RANDOM}
\end{1=}

where alternative-record-key-clause is

\key{ALTERNATE} RECORD KEY IS \identifier
\pending{
  \begin{0-1}
    \begin{1=}
      = \\
      \key{SOURCE} IS \\
    \end{1=}
    \begin{1=}
      \identifier
    \end{1=}\ldots
  \end{0-1}
}
\begin{0-1}
  WITH \key{DUPLICATES}
\end{0-1}
\pending{
  \begin{0-1}
    \key{SUPPRESS} \key{WHEN}
    \begin{1=}
      \key{ALL} \literal \\
      \key{SPACE} \\
      \key{ZERO}
    \end{1=}
  \end{0-1}
}

where collating-sequence-clause is

\pending{COLLATING \key{SEQUENCE} IS collating-sequence-name-1}


where file-status-clause is

\begin{0-1}
  \key{FILE} \\
  \key{SORT}
\end{0-1}
\key{STATUS} IS \identifier

where lock-mode-clause is

\key{LOCK} MODE IS
\begin{1=}
  \begin{1=}
    \begin{1=}
      \key{MANUAL} \\
      \key{AUTOMATIC}
    \end{1=}
    \key{WITH}
    \begin{1=}
      \key{LOCK} \key{ON} MULTIPLE
      \begin{1=}
        RECORD \\
        RECORDS
      \end{1=} \\
      \pending{\key{ROLLBACK}}
    \end{1=}
  \end{1=} \\
  \key{EXCLUSIVE}
\end{1=}

where organization-clause is

\begin{0-1}
  \begin{1=}
    \key{ORGANIZATION} \\
    \miscext{\key{ORGANISATION}}
  \end{1=}
  IS
\end{0-1}
\begin{1=}
  \key{INDEXED} \\
  \xopen{\key{LINE} \key{SEQUENTIAL}} \\
  \miscext{RECORD BINARY} \key{SEQUENTIAL} \\
  \key{RELATIVE}
\end{1=}

where padding-character-clause is

\pending{
  \key{PADDING} CHARACTER IS
  \begin{1=}
    \identifier \\
    \literal
  \end{1=}
}

where record-delimiter-clause is

\pending{\key{RECORD} \key{DELIMITER} IS \key{STANDARD-1}}

where record-key-clause is

\key{RECORD} KEY IS \identifier
\begin{0-1}
  \begin{1=}
    = \\
    \key{SOURCE} IS
  \end{1=}
  \begin{1=}
    \identifier
  \end{1=}\ldots
\end{0-1}

where relative-key-clause is

\key{RELATIVE} KEY IS \identifier

where reserve-clause is

\pending{
  \key{RESERVE}
  \begin{1=}
    \key{NO} \\
    \integer
  \end{1=}
  AREA
}

where sharing-clause is

\key{SHARING} WITH
\begin{1=}
  \key{ALL} OTHER \\
  \key{NO} OTHER \\
  \key{READ} \key{ONLY}
\end{1=}

\subsection{I-O-CONTROL paragraph}
\begin{0-1}
  \key{I-O-CONTROL}.
\end{0-1}\newline
\begin{1=}
  \key{SAME}
  \begin{0-1}
    \key{RECORD} \\
    \key{SORT} \\
    \key{SORT-MERGE}
  \end{0-1}
  AREA FOR
  \begin{1=}
    \filename
  \end{1=}\ldots \\

  \deleted{
    \key{MULTIPLE} FILE TAPE CONTAINS
    \begin{1=}
      \filename
      \begin{0-1}
        \key{POSITION} \integer
      \end{0-1}
    \end{1=}\ldots
  }
\end{1=}\ldots\ {}.

\chapter{Data division}

\begin{0-1}
  \key{DATA} \key{DIVISION}.
\end{0-1}\newline
\begin{0-1}
  file-section
\end{0-1}\newline
\begin{0-1}
  working-storage-section
\end{0-1}\newline
\begin{0-1}
  local-storage-section
\end{0-1}\newline
\begin{0-1}
  \pending{report-section}
\end{0-1}\newline
\begin{0-1}
  screen-section
\end{0-1}\newline

\section{File section}

\begin{0-1}
  \key{FILE} \key{SECTION}.
\end{0-1}\newline
\begin{1=}
  file-description-entry
  \begin{1=}
    record-description \\
    constant-definition
  \end{1=}\ldots
\end{1=}\ldots

\subsection{File description entry}

\begin{1=}
  \key{FD} \\
  \key{SD}
\end{1=}
\filename
\begin{0-1}
  block-clause \\

  \pending{
    \key{CODE-SET} IS alphabet-name-1
    \begin{0-1}
      \key{FOR}
      \begin{1=}
        \identifier
      \end{1=} \ldots
    \end{0-1}
  } \\

  \deleted{
    \key{DATA}
    \begin{1=}
      \key{RECORD} IS \\
      \key{RECORDS} ARE
    \end{1=}
    \begin{1=}
      \identifier
    \end{1=}\ldots
  } \\

  IS \key{EXTERNAL} \\
  IS \key{GLOBAL} \\

  \deleted{
    \key{LABEL}
    \begin{1=}
      \key{RECORD} IS \\
      \key{RECORDS} ARE
    \end{1=}
    \begin{1=}
      \key{STANDARD} \\
      \key{OMITTED}
    \end{1=}
  } \\

  linage-clause \\

  \deleted{
    \key{RECORDING} MODE IS
    \begin{1=}
      \begin{1=}
        \key{F} \\
        \key{FIXED} \\
      \end{1=} \\

      \begin{1=}
        \key{V} \\
        \key{VARIABLE} \\
      \end{1=} \\

      \key{U} \\
      \key{S}
    \end{1=}
  } \\

  \pending{
    \begin{1=}
      \key{REPORT} IS \\
      \key{REPORTS} ARE
    \end{1=}
    \begin{1=}
      \identifier
    \end{1=}\ldots
  } \\

  \deleted{
    \key{VALUE} \key{OF}
    \begin{1=}
      \key{FILE-ID} \\
      \key{ID} \\
      \identifier
    \end{1=}
    IS
    \begin{1=}
      \literal \\
      \identifier
    \end{1=}
  } \\

  record-clause
\end{0-1}\ldots\ {}.

\section{Working-storage section}
\key{WORKING-STORAGE} \key{SECTION}.\newline
\begin{0-1}
  constant-definition \\
  record-description
\end{0-1}\ldots

\section{Local-storage section}
\key{LOCAL-STORAGE} \key{SECTION}.\newline
\begin{0-1}
  constant-definition \\
  record-description
\end{0-1}\ldots

\section{Linkage section}
\key{LINKAGE} \key{SECTION}.\newline
\begin{0-1}
  constant-definition \\
  record-description
\end{0-1}\ldots

\section{Report section}
\pending{
  \key{REPORT} \key{SECTION}.\newline
  \begin{0-1}
    constant-definition \\
    report-description
  \end{0-1}\ldots
}

\subsection{Report description}

\key{RD} \reportname

\begin{0-1}
  IS \key{GLOBAL} \\

  \key{CODE} IS
  \begin{1=}
    \identifier
    \literal
  \end{1=} \\

  \begin{1=}
    \key{CONTROL} IS \\
    \key{CONTROLS} ARE
  \end{1=}
  FINAL
  \begin{1=}
    \identifier
  \end{1=}\ldots \\

  \key{PAGE}
  \begin{0-1}
    \key{LIMIT} IS \\
    \key{LIMITS} ARE
  \end{0-1}
  \integer
  \begin{0-1}
    \key{LINE} \\
    \key{LINES}
  \end{0-1}
  \begin{0-1}
    \integer
    \begin{1=}
      \key{COLUMNS} \\
      \key{COLS}
    \end{1=}
  \end{0-1}
\end{0-1}\ldots\ {}.\newline

\begin{1=}
  report-group-description-1
\end{1=}\ldots

\paragraph{Report group description}
\ {}\newline
\pending{
  level-number entry-name
  \begin{0-1}
    blank-clause \\
    column-clause \\
    \key{GROUP} INDICATE \\
    justified-clause \\
    line-clause \\
    next-group-clause \\
    picture-clause \\
    present-when-clause \\
    occurs-clause \\
    sign-clause \\
    source-clause \\
    sum-clause \\
    type-clause \\
    \key{USAGE} IS \key{DISPLAY} \\
    value-clause \\
    varying-clause
  \end{0-1}\ldots\ {}.
}

\section{Screen section}
\key{SCREEN} \key{SECTION}.\newline
\begin{0-1}
  constant-definition \\
  screen-description
\end{0-1}\ldots

\subsection{Screen description}
level-number entry-name
\begin{0-1}
  \begin{1=}
    \key{AUTO} \\
    \miscext{\key{AUTO-SKIP}} \\
    \miscext{\key{AUTOTERMINATE}} \\
  \end{1=} \\


  \key{BLANK}
  \begin{1=}
    \key{LINE} \\
    \key{SCREEN}
  \end{1=} \\

  \key{BLINK} \\

  column-clause \\

  \key{ERASE}
  \begin{1=}
    \key{EOL} \\
    \key{EOS} \\

    \begin{0-1}
      \key{END} OF
    \end{0-1}
    \begin{1=}
      \key{LINE} \\
      \key{SCREEN}
    \end{1=}
  \end{1=} \\

  \begin{1=}
    \key{FULL} \\
    \miscext{\key{LENGTH-CHECK}} \\
  \end{1=} \\

  IS \key{GLOBAL} \\
  \miscext{\pending{\key{GRID}}} \\

  \miscext{\key{INITIAL}} \\
  \miscext{\pending{\key{LEFTLINE}}} \\
  justified-clause \\
  line-clause \\

  occurs-clause \\
  \miscext{\pending{\key{OVERLINE}}} \\
  picture-clause \\

  \key{PROMPT}
  \begin{0-1}
    \key{CHARACTER} IS
    \begin{1=}
      \identifier \\
      \literal
    \end{1=}
  \end{0-1} \\
  
  \begin{1=}
    \key{REQUIRED} \\
    \miscext{\key{EMPTY-CHECK}}
  \end{1=} \\


  screen-attribute-clauses \\
  source-destination-clauses \\
  
  \key{SECURE} \\
  sign-clause \\

  usage-clause \\
  value-clause
\end{0-1}\ldots\ {}.

where screen-attribute-clauses is

\begin{0-1}
  \key{BELL} \\
  \key{BEEP}
\end{0-1}

\begin{0-1}
  \key{HIGHLIGHT}
\end{0-1}

\begin{0-1}
  \key{LOWLIGHT}
\end{0-1}

\begin{0-1}
  \key{REVERSE-VIDEO}
\end{0-1}

\begin{0-1}
  \key{UNDERLINE}
\end{0-1}


\begin{0-1}
  \begin{1=}
    \key{FOREGROUND-COLOR} \\
    \miscext{\key{FOREGROUND-COLOUR}}
  \end{1=}
  IS
  \begin{1=}
    \identifier \\
    \integer
  \end{1=}
\end{0-1}

\begin{0-1}
  \begin{1=}
    \key{BACKGROUND-COLOR} \\
    \miscext{\key{BACKGROUND-COLOUR}}
  \end{1=}
  IS
  \begin{1=}
    \identifier \\
    \integer
  \end{1=}
\end{0-1}

where source-destination-clauses is

\begin{0-1}
  \key{FROM}
  \begin{1=}
    \identifier \\
    \literal
  \end{1=} \\
\end{0-1}

\begin{0-1}
  \key{TO} \identifier
\end{0-1}

\begin{0-1}
  \key{USING} \identifier
\end{0-1}


\section{Record description}
level-number entry-name
\begin{0-1}
  \key{ANY}
  \begin{1=}
    \key{LENGTH} \\
    \gnucobol{\key{NUMERIC}}
  \end{1=} \\

  blank-clause \\

  IS \key{EXTERNAL}
  \begin{0-1}
    \key{AS} \literal
  \end{0-1} \\

  IS \key{GLOBAL} \\

  justified-clause \\

  occurs-clause \\

  \key{PICTURE} picture-string-1 \\

  \key{REDEFINES} \identifier \\

  \key{RENAMES} \identifier
  \begin{0-1}
    \begin{1=}
      \key{THROUGH} \\
      \key{THRU}
    \end{1=}
    \identifier
  \end{0-1} \\

  sign-clause \\

  \begin{1=}
    \key{SYNCHRONIZED} \\
    \miscext{\key{SYNCHRONISED}} \\
    \key{SYNC}
  \end{1=}
  \begin{0-1}
    \key{LEFT} \\
    \key{RIGHT}
  \end{0-1} \\

  usage-clause \\

  value-clause
\end{0-1}\ldots\ {}.

\section{Constant definition}

\begin{1=}
  1 \\
  01
\end{1=}
 \identifier \key{CONSTANT}
\begin{0-1}
  IS \key{GLOBAL}
\end{0-1}
\begin{1=}
  AS
  \begin{1=}
    \literal \\
    \begin{1=}
      \key{BYTE-LENGTH} \\
      \key{LENGTH}
    \end{1=}
    OF \identifier
  \end{1=} \\
  \pending{\key{FROM} \identifier}
\end{1=}.

\section{Data division clauses}

\subsection{BLANK WHEN ZERO clause}
\key{BLANK} WHEN \key{ZERO}

\subsection{BLOCK clause}
\ {}\newline
\key{BLOCK} CONTAINS \integer
\begin{0-1}
  \key{TO} \integer
\end{0-1}
\begin{0-1}
  \key{CHARACTERS} \\
  \key{RECORDS}
\end{0-1}\\

\subsection{COLUMN clause}

\format{report-section}
\begin{1=}
  \begin{1=}
    \key{COLUMN} \\
    \key{COL}
  \end{1=}
  NUMBERS
  \begin{0-1}
    \key{IS} \\
    \key{ARE}
  \end{0-1} \\

  \begin{1=}
    \key{COLUMNS} \\
    \key{COLS}
  \end{1=}
  ARE
\end{1=}
\begin{1=}
  \begin{0-1}
    \key{PLUS}
  \end{0-1}
\integer
\end{1=}
\ldots

\format{screen-section}
\begin{1=}
  \key{COLUMN} \\
  \key{COL}
\end{1=}
NUMBER IS
\begin{0-1}
  + \\
  - \\
  \key{PLUS} \\
  \key{MINUS}
\end{0-1}
\begin{1=}
  \identifier \\
  \integer
\end{1=}

\subsection{Entry name}
\begin{0-1}
  \key{FILLER} \\
  \identifier
\end{0-1}

\subsection{JUSTIFIED clause}
\key{JUSTIFIED} RIGHT

\subsection{LINAGE clause}
\key{LINAGE} IS
\begin{1=}
  \identifier \\
  \literal
\end{1=}
LINES
\begin{0-1}
  \begin{1=}
    \key{BOTTOM} \\
    \key{TOP} \\
    WITH \key{FOOTING} AT
  \end{1=}
  \begin{1=}
    \identifier \\
    \literal
  \end{1=} \\
\end{0-1}
\ldots \\

\subsection{LINE clause}

\format{report section}
\begin{1=}
  \key{LINE} NUMBERS
  \begin{0-1}
    \key{IS} \\
    \key{ARE}
  \end{0-1} \\

  \key{LINES} ARE
\end{1=}
\begin{1=}
  \begin{0-1}
    \key{PLUS}
  \end{0-1}
  \integer \\
  \key{NEXT} \key{PAGE}
\end{1=}\ldots

\format{screen section}
\key{LINE} NUMBER IS
\begin{0-1}
    + \\
    - \\
    \key{MINUS} \\
    \key{PLUS}
\end{0-1}
\begin{1=}
  \identifier \\
  \integer
\end{1=}

\subsection{Level-number}

A 1- or 2-digit integer having a value that is either between 1 and 49 or is 66, 77, \miscext{78} or 88.

\subsection{NEXT GROUP clause}
\key{NEXT} \key{GROUP} IS
\begin{1=}
  \begin{0-1}
    \key{PLUS}
  \end{0-1}
  \integer \\
  \key{NEXT} \key{PAGE}
\end{1=}

\subsection{OCCURS clause}

\format{usual}
\key{OCCURS}

\begin{1=}
  \integer
  \begin{0-1}
    \key{TO} \integer
  \end{0-1}
  TIMES
  \begin{0-1}
    \key{DEPENDING} ON \identifier
  \end{0-1} \\

  \pending{
    \key{DYNAMIC}
    \begin{0-1}
      \key{CAPACITY} IN \identifier
    \end{0-1}
    \begin{0-1}
      \key{FROM} \integer
    \end{0-1}
    \begin{0-1}
      \key{TO} \integer
    \end{0-1}
    \begin{0-1}
      \key{INITIALIZED}
    \end{0-1}
  }
\end{1=}

\begin{0-1}
    \begin{1=}
      \key{ASCENDING} \\
      \key{DESCENDING}
    \end{1=}
    KEY IS
    \begin{1=}
      \identifier
    \end{1=}\ldots
\end{0-1}\ldots

\begin{0-1}
  \key{INDEXED} BY
  \begin{1=}
    \cobolindexname
  \end{1=}\ldots
\end{0-1}

\format{report section}
\key{OCCURS} \integer
\begin{0-1}
  \key{TO} \integer
\end{0-1}
TIMES
\begin{0-1}
  \key{DEPENDING} ON \identifier
\end{0-1}
\begin{0-1}
  \key{STEP} \integer
\end{0-1}

\format{screen section}
\key{OCCURS} \integer TIMES

\subsection{PRESENT WHEN clause}
\key{PRESENT} \key{WHEN} \condition

\subsection{RECORD clause}
\key{RECORD}
\begin{1=}
  CONTAINS \integer
  \begin{0-1}
    \key{TO} \integer
  \end{0-1}
  CHARACTERS \\
  IS \key{VARYING} in size
  \begin{0-1}
    FROM \integer
  \end{0-1}
  \begin{0-1}
    \key{TO} \integer
  \end{0-1}
  CHARACTERS \\\qquad
  \key{DEPENDING} ON \identifier
\end{1=}

\subsection{SIGN clause}
SIGN IS
\begin{1=}
  \key{LEADING} \\
  \key{TRAILING}
\end{1=}
\begin{0-1}
  \key{SEPARATE} CHARACTER
\end{0-1}

\subsection{SOURCE clause}
\key{SOURCE} IS number-1
\begin{0-1}
  rounded-phrase
\end{0-1}

\subsection{SUM clause}
\key{SUM} OF
\begin{1=}
  number-2
\end{1=}\ldots
\begin{0-1}
  \key{RESET} ON
  \begin{1=}
    \identifier \\
    \key{FINAL}
  \end{1=}
\end{0-1}

\subsection{TYPE clause}
\key{TYPE} IS
\begin{1=}
  \begin{1=}
    \begin{1=}
      \key{CONTROL} \key{HEADING} \\
      \key{CH}
    \end{1=} \\
    \begin{1=}
      \key{CONTROL} \key{FOOTING} \\
      \key{CF}
    \end{1=}
  \end{1=}
  \begin{1=}
    \identifier \\
    \key{FINAL}
  \end{1=}
  \begin{0-1}
    \key{OR} \key{PAGE}
  \end{0-1} \\

  \begin{1=}
    \key{DETAIL} \\
    \key{DE}
  \end{1=} \\

  \begin{1=}
    \key{PAGE} \key{FOOTING} \\
    \key{PF}
  \end{1=} \\

  \begin{1=}
    \key{PAGE} \key{HEADING} \\
    \key{PH}
  \end{1=} \\

  \begin{1=}
    \key{REPORT} \key{FOOTING} \\
    \key{RF}
  \end{1=} \\

  \begin{1=}
    \key{REPORT} \key{HEADING} \\
    \key{RH}
  \end{1=}
\end{1=}

\subsection{USAGE clause}

\begin{0-1}
  \key{USAGE} IS
\end{0-1}
\begin{1=}
  \key{BINARY} \\

  fixed-length-integers \\
  computational-usages \\

  \key{DISPLAY} \\

  \pending{\key{FLOAT-BINARY-32}} \\
  \pending{\key{FLOAT-BINARY-64}} \\
  \pending{\key{FLOAT-BINARY-128}} \\
  \key{FLOAT-DECIMAL-16} \\
  \key{FLOAT-DECIMAL-34} \\
  \key{FLOAT-LONG} \\
  \key{FLOAT-SHORT} \\

  \key{INDEX} \\

  \pending{\key{NATIONAL}} \\

  \key{PACKED} \key{DECIMAL} \\
  \miscext{\key{POINTER}} \\
  \key{PROGRAM-POINTER} \\
\end{1=}

where fixed-length-integers is

\begin{1=}
  \begin{1=}
    \key{BINARY-CHAR} \\

    \begin{1=}
      \key{BINARY-LONG} \\
      \gnucobol{\key{BINARY-INT}}
    \end{1=} \\

    \gnucobol{\key{BINARY-C-LONG}} \\
    
    \begin{1=}
      \key{BINARY-DOUBLE} \\
      \gnucobol{\key{BINARY-LONG-LONG}} \\
    \end{1=} \\
  \end{1=}
  \begin{0-1}
    \key{SIGNED} \\
    \key{UNSIGNED}
  \end{0-1} \\

  \miscext{\key{SIGNED-SHORT}} \\
  \miscext{\key{SIGNED-INT}} \\
  \miscext{\key{SIGNED-LONG}} \\

  \miscext{\key{UNSIGNED-SHORT}} \\
  \miscext{\key{UNSIGNED-INT}} \\
  \miscext{\key{UNSIGNED-LONG}}
\end{1=}

where computation-usages is

\begin{1=}
  \begin{1=}
    \key{COMP} \\
    \key{COMPUTATIONAL}
  \end{1=} \\

  \miscext{
    \begin{1=}
      \key{COMP-1} \\
      \key{COMPUTATIONAL-1}
    \end{1=}
  } \\

  \miscext{
    \begin{1=}
      \key{COMP-2} \\
      \key{COMPUTATIONAL-2}
    \end{1=}
  } \\

  \xopen{
    \begin{1=}
      \key{COMP-3} \\
      \key{COMPUTATIONAL-3}
    \end{1=}
  } \\

  \miscext{
    \begin{1=}
      \key{COMP-4} \\
      \key{COMPUTATIONAL-4}
    \end{1=}
  } \\

  \xopen{
    \begin{1=}
      \key{COMP-5} \\
      \key{COMPUTATIONAL-5}
    \end{1=}
  } \\

  \miscext{
    \begin{1=}
      \key{COMP-6} \\
      \key{COMPUTATIONAL-6}
    \end{1=}
  } \\

  \miscext{
    \begin{1=}
      \key{COMP-X} \\
      \key{COMPUTATIONAL-X}
    \end{1=}
  } \\
\end{1=}

\subsection{VALUE clause}
\key{VALUE}
\begin{0-1}
  \key{IS} \\
  \key{ARE}
\end{0-1}
\begin{1=}
  \literal
  \begin{0-1}
    \begin{1=}
      \key{THROUGH} \\
      \key{THRU}
    \end{1=}
    \literal
  \end{0-1}
\end{1=}\ldots

\subsection{VARYING clause}
\key{VARYING} \identifier \key{FROM} number-1 \key{BY} number-2

\chapter{Procedure division}

\key{PROCEDURE} \key{DIVISION}
\begin{0-1}
using-chaining-clause
\end{0-1}
\begin{0-1}
  \key{RETURNING}
  \begin{1=}
    \identifier \\
    \key{OMITTED}
  \end{1=}
\end{0-1}.\newline
\begin{0-1}
  declaratives
\end{0-1}\newline
\begin{0-1}
  section-name-2 \key{SECTION}. \\
  paragraph-name-2. \\
  \begin{1=}
    statement
  \end{1=}\ldots
\end{0-1} \ldots

where using-chaining-clause is

\begin{1=}
  \key{USING} \\
  \miscext{\key{CHAINING}}
\end{1=}
\begin{1=}
  BY
  \begin{1=}
    \key{REFERENCE} \\
    \pending{\key{VALUE}}
  \end{1=}
  \miscext{
    \begin{0-1}
      \begin{0-1}
        \key{UNSIGNED}
      \end{0-1}
      \key{SIZE} IS
      \begin{1=}
        \key{AUTO} \\
        \integer
      \end{1=} \\

      \key{SIZE} IS \key{DEFAULT}
    \end{0-1}
  }

  \begin{0-1}
    \key{OPTIONAL}
  \end{0-1}
  \identifier
\end{1=}\ldots

where declaratives is

\key{DECLARATIVES}.

\begin{0-1}
  section-name-1 \key{SECTION}.
  use-statement
  \begin{0-1}
    \begin{0-1}
      paragraph-name-2.
    \end{0-1}
    \imperativestatement .
  \end{0-1} \ldots
\end{0-1}\ldots

\key{END} \key{DECLARATIVES}.

\section{Common phrases}

\subsection{ROUNDED phrase}

\key{ROUNDED}
\begin{0-1}
  \key{MODE} IS
  \begin{1=}
    \key{AWAY-FROM-ZERO} \\
    \key{NEAREST-AWAY-FROM-ZERO} \\
    \key{NEAREST-EVEN} \\
    \key{NEAREST-TOWARD-ZERO} \\
    \key{PROHIBITED} \\
    \key{TOWARD-GREATER} \\
    \key{TOWARD-LESSER} \\
    \key{TRUNCATION}
  \end{1=}
\end{0-1}

\subsection{SIZE phrase}

\gnucobol{
  \begin{1=}
    \key{SIZE} IS \key{AUTO} \\
    \key{SIZE} IS \key{DEFAULT} \\
    \key{SIZE} IS \integer \\
    \key{UNSIGNED} \key{SIZE} IS \key{AUTO} \\
    \key{UNSIGNED} \key{SIZE} IS \integer
  \end{1=}
}
\section{ACCEPT statement}

\format{user}
\key{ACCEPT}
\begin{1=}
  \identifier \\
  \miscext{\key{OMITTED}}
\end{1=}

\begin{0-1}
  \begin{1+}
    AT \key{LINE} NUMBER
    \begin{1=}
      \identifier \\
      \integer
    \end{1=} \\

    AT
    \begin{1=}
      \key{COLUMN} \\
      \key{COL} \\
      \key{POSITION}
    \end{1=}
    NUMBER
    \begin{1=}
      \identifier \\
      \integer
    \end{1=}
  \end{1+} \\

  \key{AT}
  \begin{1=}
    \identifier \\
    \integer
  \end{1=}
\end{0-1}

\begin{0-1}
\key{FROM} \key{CRT}
\end{0-1}

\begin{0-1}
\key{MODE} IS \key{BLOCK}
\end{0-1}


\begin{0-1}
WITH \key{AUTO}
\end{0-1}

\begin{0-1}
WITH \key{TAB}
\end{0-1}

\begin{0-1}
  WITH
  \begin{1=}
    \key{BELL} \\
    \key{BEEP}
  \end{1=}
\end{0-1}

\begin{0-1}
WITH \key{BLINK}
\end{0-1}

\begin{0-1}
WITH \key{CONVERSION}
\end{0-1}

\begin{0-1}
  WITH
  \begin{1=}
    \key{FULL} \\
    \key{LENGTH-CHECK} \\
  \end{1=}
\end{0-1}

\begin{0-1}
WITH
\begin{1=}
  \key{HIGHLIGHT} \\
  \key{LOWLIGHT}
\end{1=}
\end{0-1}

\begin{0-1}
WITH \key{LEFTLINE}
\end{0-1}

\begin{0-1}
WITH \key{LOWER}
\end{0-1}

\begin{0-1}
WITH \key{NO-ECHO}
\end{0-1}

\begin{0-1}
WITH \key{OVERLINE}
\end{0-1}

\begin{0-1}
  WITH \key{PROMPT}
  \begin{0-1}
    \key{CHARACTER} IS
    \begin{1=}
      \identifier \\
      \literal
    \end{1=}
  \end{0-1}
\end{0-1}

\begin{0-1}
  WITH
  \begin{1=}
    \key{REQUIRED} \\
    \key{EMPTY-CHECK}
  \end{1=}
\end{0-1}

\begin{0-1}
WITH \key{REVERSE-VIDEO}
\end{0-1}

\begin{0-1}
WITH \key{SECURE}
\end{0-1}

\begin{0-1}
  WITH \miscext{PROTECTED} \key{SIZE} IS
  \begin{1=}
    \identifier \\
    \integer
  \end{1=}
\end{0-1}

\begin{0-1}
WITH \key{UNDERLINE}
\end{0-1}


\begin{0-1}
  WITH
  \begin{0-1}
    \key{NO}
  \end{0-1}
  \begin{1=}
    \key{DEFAULT} \\
    \key{UPDATE}
  \end{1=}
\end{0-1}

\begin{0-1}
WITH \key{UPPER}
\end{0-1}


\begin{0-1}
  WITH
  \begin{1=}
    \key{FOREGROUND-COLOR} \\
    \key{FOREGROUND-COLOUR}
  \end{1=}
  IS
  \begin{1=}
    \identifier \\
    \integer
  \end{1=}
\end{0-1}

\begin{0-1}
  WITH
  \begin{1=}
    \key{BACKGROUND-COLOR} \\
    \key{BACKGROUND-COLOUR}
  \end{1=}
  IS
  \begin{1=}
    \identifier \\
    \integer
  \end{1=}
\end{0-1}

\begin{0-1}
  WITH \key{SCROLL} \key{UP}
  \begin{0-1}
    \identifier \\
    \integer
  \end{0-1}
  \begin{1=}
    \key{LINE} \\
    \key{LINES}
  \end{1=}
\end{0-1}

\begin{0-1}
  WITH \key{SCROLL} \key{DOWN}
  \begin{0-1}
    \identifier \\
    \integer
  \end{0-1}
  \begin{1=}
    \key{LINE} \\
    \key{LINES}
  \end{1=}
\end{0-1}

\begin{0-1}
  WITH
  \begin{1=}
    \key{TIMEOUT} \\
    \key{TIME-OUT} \\
  \end{1=}
  AFTER
  \begin{0-1}
    \identifier \\
    \integer
  \end{0-1}
\end{0-1}

\format{temporal}
\key{ACCEPT} \identifier \key{FROM}
\begin{1=}
  \key{DATE}
  \begin{0-1}
    \key{YYYYMMDD}
  \end{0-1} \\

  \key{DAY}
  \begin{0-1}
    \key{YYYYDDD}
  \end{0-1} \\

  \key{DAY-OF-WEEK} \\
  \key{TIME} \\
\end{1=}

\format{environment}
\miscext{
  \begin{minipage}[!h]{1.0\linewidth}
    \key{ACCEPT} \identifier \key{FROM}
    \begin{1=}
      \key{ARGUMENT-NUMBER} \\

      \begin{1=}
        \key{COLUMNS} \\
        \key{COLS}
      \end{1=} \\

      \key{COMMAND-LINE} \\
      \key{ESCAPE} \key{KEY} \\
      \key{EXCEPTION} \key{STATUS} \\

      \begin{1=}
        \key{LINES} \\
        \key{LINE} \key{NUMBER}
      \end{1=} \\

      \mnemonicname \\
      \key{USER} \key{NAME} \\
      \key{WORD}
    \end{1=}
  \end{minipage}
}


\format{environment-exception}
\miscext{
  \begin{minipage}[!h]{1.0\linewidth}
    \key{ACCEPT} \identifier \key{FROM}
    \begin{1=}
      \key{ARGUMENT-VALUE} \\
      \key{ENVIRONMENT}
      \begin{1=}
        \identifier \\
        \literal
      \end{1=} \\
      \key{ENVIRONMENT-VALUE} \\
    \end{1=}

    \begin{0+}
      ON
      \begin{1=}
        \key{EXCEPTION} \\
        \key{ESCAPE}
      \end{1=}
      \imperativestatement \\

      \key{NOT} ON
      \begin{1=}
        \key{EXCEPTION} \\
        \key{ESCAPE}
      \end{1=}
      \imperativestatement \\
    \end{0+}
  \end{minipage}
}
\section{ADD statement}

\format{simple}
\key{ADD}
\begin{1=}
  \identifier \\
  \literal \\
\end{1=} \ldots
\key{TO}
\begin{1=}
  \identifier
\end{1=} \ldots

\begin{0+}
  ON \key{SIZE} \key{ERROR} \imperativestatement \\
  \key{NOT} ON \key{SIZE} \key{ERROR} \imperativestatement
\end{0+}


\begin{0-1}
  \key{END-ADD}
\end{0-1}

\format{giving}
\key{ADD}
\begin{1=}
  \identifier \\
  \literal \\
\end{1=} \ldots
\begin{0-1}
  \key{TO}
  \begin{0-1}
    \identifier
  \end{0-1} \ldots
\end{0-1}

\key{GIVING}
\begin{1=}
  \identifier
  \begin{0-1}
    rounded-phrase
  \end{0-1}
\end{1=} \ldots

\begin{0+}
  ON \key{SIZE} \key{ERROR} \imperativestatement \\
  \key{NOT} ON \key{SIZE} \key{ERROR} \imperativestatement
\end{0+}

\begin{0-1}
  \key{END-ADD}
\end{0-1}

\format{corresponding}
\key{ADD}
\begin{1=}
  \key{CORRESPONDING} \\
  \key{CORR}
\end{1=}
\identifier \key{TO} \identifier
\begin{0-1}
  rounded-phrase
\end{0-1}

\begin{0+}
  ON \key{SIZE} \key{ERROR} \imperativestatement \\
  \key{NOT} ON \key{SIZE} \key{ERROR} \imperativestatement
\end{0+}

\begin{0-1}
  \key{END-ADD}
\end{0-1}


\section{ALLOCATE statement}

\key{ALLOCATE}
\begin{1=}
  \identifier
  \begin{0-1}
    \key{INITIALIZED}
  \end{0-1} \\
  \arithmeticexpression
  \begin{0-1}
    \key{INITIALIZED}
    \gnucobol{
      \begin{0-1}
        \key{TO}
        \begin{1=}
          \identifier \\
          \literal
        \end{1=}
      \end{0-1}
    }
  \end{0-1}
\end{1=}

\begin{0-1}
  \key{RETURNING} \identifier
\end{0-1} \\

\section{ALTER statement}

\deleted{
  \key{ALTER}
  \begin{1=}
    \procedurename TO PROCEED \key{TO} \procedurename
  \end{1=} \ldots
}

\section{CALL statement}

\key{CALL}
\miscext{
  \begin{0-1}
    \mnemonicname \\
    \key{STATIC} \\
    \key{STDCALL}
  \end{0-1}
}
\begin{1=}
  \identifier \\
  \literal \\
  \functionname
\end{1=}


\begin{0-1}
  \key{USING}
  \begin{1=}
    \begin{0-1}
      BY
      \begin{1=}
        \key{REFERENCE} \\
        \key{CONTENT} \\
        \key{VALUE}
      \end{1=}
    \end{0-1}
    \begin{1=}
      \key{OMITTED} \\

      \gnucobol{
        \begin{0-1}
          size-phrase
        \end{0-1}
      }
      \begin{1=}
        \identifier \\
        \literal
      \end{1=}
    \end{1=}
  \end{1=}\ldots
\end{0-1}

\begin{0-1}
  \begin{1=}
    \key{RETURNING} \\
    \miscext{\key{GIVING}}
  \end{1=}
  \begin{1=}
    INTO \identifier \\
    \key{ADDRESS} OF \identifier \\
    \gnucobol{\key{NOTHING}} \\
    \key{NULL} \\
    \key{OMITTED} \\
  \end{1=}
\end{0-1}

\begin{0+}
  ON
  \begin{1=}
    \key{EXCEPTION} \\
    \archaic{\key{OVERFLOW}}
  \end{1=}
  \imperativestatement \\
  \key{NOT} ON \key{EXCEPTION} \imperativestatement
\end{0+}

\begin{0-1}
  \key{END-CALL}
\end{0-1}

\section{CANCEL statement}

\key{CANCEL}
\begin{1=}
  \identifier \\
  \literal
\end{1=} \ldots

\section{CLOSE statement}

\key{CLOSE}
\begin{1=}
  \filename
  \begin{0-1}
    \begin{1=}
      \key{REEL} \\
      \key{UNIT}
    \end{1=}
    \begin{0-1}
      FOR \key{REMOVAL}
    \end{0-1} \\

    WITH \key{NO} \key{REWIND} \\
    WITH \key{LOCK}
  \end{0-1}
\end{1=} \ldots

\section{COMMIT statement}

\miscext{\key{COMMIT}}

\section{COMPUTE statement}

\key{COMPUTE}
\begin{1=}
  \identifier
  \begin{0-1}
    rounded-phrase
  \end{0-1}
\end{1=} \ldots
\begin{1=}
  = \\
  \miscext{\key{EQUAL}} \\
  \miscext{\key{EQUALS}}
\end{1=}
\arithmeticexpression

\begin{0+}
  ON \key{SIZE} \key{ERROR} \imperativestatement \\
  \key{NOT} ON \key{SIZE} \key{ERROR} \imperativestatement
\end{0+}

\begin{0-1}
  \key{END-COMPUTE}
\end{0-1}

\section{CONTINUE statement}

\key{CONTINUE}

\section{DELETE statement}

\format{record}
\key{DELETE} \filename RECORD

\begin{0+}
  \key{INVALID} KEY \imperativestatement \\
  \key{NOT} \key{INVALID} KEY \imperativestatement
\end{0+}

\begin{0-1}
  \key{END-DELETE}
\end{0-1}

\format{file}
\miscext{
  \key{DELETE} \key{FILE}
  \begin{1=}
    \filename
  \end{1=} \ldots
}
\gnucobol{\begin{0-1}
  \key{END-DELETE}
\end{0-1}}

\section{DISPLAY statement}

\format{device}
\key{DISPLAY}
\begin{1=}
  \identifier \\
  \literal
\end{1=} \ldots
\begin{0+}
  \key{UPON} \mnemonicname \\
  WITH \key{NO} \key{ADVANCING}
\end{0+}

\begin{0+}
  ON \key{EXCEPTION} \imperativestatement \\
  \key{NOT} ON \key{EXCEPTION} \imperativestatement \\
\end{0+}

\begin{0-1}
  \key{END-DISPLAY}
\end{0-1}

\format{environment}
\miscext{
  \begin{minipage}[!h]{1.0\linewidth}
    \key{DISPLAY}
    \begin{1=}
      \identifier \\
      \literal
    \end{1=}
    \key{UPON}
    \begin{1=}
      \key{ARGUMENT-NUMBER} \\
      \key{COMMAND-LINE} \\
      \key{ENVIRONMENT-NAME} \\
      \key{ENVIRONMENT-VALUE} \\
    \end{1=}

    \begin{0+}
      ON \key{EXCEPTION} \imperativestatement \\
      \key{NOT} ON \key{EXCEPTION} \imperativestatement \\
    \end{0+}

    \begin{0-1}
      \key{END-DISPLAY}
    \end{0-1}
  \end{minipage}
}
  
\format{screen}
\key{DISPLAY}
\begin{1=}
  \begin{1=}
    \identifier \\
    \miscext{\literal} \\
    \miscext{\key{OMITTED}}
  \end{1=}

  \miscext{
    \begin{0-1}
      \standard{position-clauses} \\
      
      \key{UPON}
      \begin{1=}
        \key{CRT} \\
        \key{CRT-UNDER}
      \end{1=} \\

      \key{MODE} IS \key{BLOCK} \\

      WITH
      \begin{1=}
        \key{BELL} \\
        \key{BEEP} \\
      \end{1=} \\

      WITH \key{BLANK}
      \begin{1=}
        \key{LINE} \\
        \key{SCREEN}
      \end{1=} \\

      WITH \key{BLINK} \\

      WITH \key{CONVERSION} \\

      WITH \key{ERASE}
      \begin{1=}
        \key{EOL} \\
        \key{EOS} \\
        
        \begin{0-1}
          \key{END} OF
        \end{0-1}
        \begin{1=}
          \key{LINE} \\
          \key{SCREEN}
        \end{1=}
      \end{1=} \\

      WITH
      \begin{1=}
        \key{HIGHLIGHT} \\
        \key{LOWLIGHT}
      \end{1=} \\

      WITH \key{OVERLINE} \\

      WITH \key{REVERSE-VIDEO} \\

      WITH \key{SIZE} IS
      \begin{1=}
        \identifier \\
        \literal
      \end{1=} \\

      WITH \key{UNDERLINE} \\

      WITH
      \begin{1=}
        \key{FOREGROUND-COLOR} \\
        \key{FOREGROUND-COLOUR}
      \end{1=}
      IS
      \begin{1=}
        \identifier \\
        \integer
      \end{1=} \\

      WITH
      \begin{1=}
        \key{BACKGROUND-COLOR} \\
        \key{BACKGROUND-COLOUR}
      \end{1=}
      IS
      \begin{1=}
        \identifier \\
        \integer
      \end{1=} \\

      WITH \key{SCROLL} \key{UP}
      \begin{0-1}
        \identifier \\
        \integer
      \end{0-1}
      \begin{1=}
        \key{LINE} \\
        \key{LINES}
      \end{1=} \\

      WITH \key{SCROLL} \key{DOWN}
      \begin{0-1}
        \identifier \\
        \integer
      \end{0-1}
      \begin{1=}
        \key{LINE} \\
        \key{LINES}
      \end{1=} \\
    \end{0-1}
  }
\end{1=} \miscext{\ldots}

\begin{0+}
  ON \key{EXCEPTION} \imperativestatement \\
  \key{NOT} ON \key{EXCEPTION} \imperativestatement \\
\end{0+}

\begin{0-1}
  \key{END-DISPLAY}
\end{0-1}

where position-clauses is

\begin{1=}
  \begin{1+}
    \key{LINE} NUMBER
    \begin{1=}
      \identifier \\
      \literal
    \end{1=} \\

    \key{AT}
    \begin{1=}
      \key{COLUMN} \\
      \key{COL} \\
      \miscext{\key{POSITION}}
    \end{1=}

    \begin{1=}
      \identifier \\
      \literal
    \end{1=}
  \end{1+} \\

  
  \miscext{
    \key{AT}
    \begin{1=}
      \identifier \\
      \literal
    \end{1=}
  }
\end{1=}

\section{DIVIDE statement}

\format{into}
\key{DIVIDE}
\begin{1=}
  \identifier \\
  \literal
\end{1=}
\key{INTO}
\begin{1=}
  \begin{1=}
    \identifier \\
    \literal
  \end{1=}
  \begin{0-1}
    rounded-phrase
  \end{0-1}
\end{1=} \ldots

\begin{0+}
  ON \key{SIZE} \key{ERROR} \imperativestatement \\
  \key{NOT} ON \key{SIZE} \key{ERROR} \imperativestatement
\end{0+}

\begin{0-1}
  \key{END-DIVIDE}
\end{0-1}

\format{giving}
\key{DIVIDE}
\begin{1=}
  \identifier \\
  \literal
\end{1=}
\begin{1=}
  \key{BY} \\
  \key{INTO}
\end{1=}
\begin{1=}
  \identifier \\
  \literal
\end{1=}

\key{GIVING}
\begin{1=}
  \begin{1=}
    \identifier \\
    \literal
  \end{1=}
  \begin{0-1}
    rounded-phrase
  \end{0-1}
\end{1=}
\ldots

\begin{0-1}
  \key{REMAINDER}
  \begin{1=}
    \identifier \\
    \literal
  \end{1=}
\end{0-1}

\begin{0+}
  ON \key{SIZE} \key{ERROR} \imperativestatement \\
  \key{NOT} ON \key{SIZE} \key{ERROR} \imperativestatement
\end{0+}

\begin{0-1}
  \key{END-DIVIDE}
\end{0-1}

\section{ENTRY statement}

\miscext{
  \begin{minipage}[!h]{1.0\linewidth}
    \key{ENTRY} \literal

    \begin{0-1}
      \key{USING}

      \begin{1=}
        \begin{0-1}
          BY
          \begin{1=}
            \key{REFERENCE} \\
            \gnucobol{\key{CONTENT}} \\
            \key{VALUE}
          \end{1=}
        \end{0-1}

        \begin{1=}
          \gnucobol{\key{OMITTED}} \\

          \begin{1=}
            \gnucobol{
              \begin{0-1}
                size-phrase
              \end{0-1}
            }
            \begin{1=}
              \identifier \\
              \literal
            \end{1=}
          \end{1=}
        \end{1=}\ldots
      \end{1=}
    \end{0-1}
  \end{minipage}
}

\section{EVALUATE statement}

\key{EVALUATE}
\begin{1=}
  \expression \\
  \key{TRUE} \\
  \key{FALSE}
\end{1=}
\begin{0-1}
  \key{ALSO}
  \begin{1=}
    \expression \\
    \key{TRUE} \\
    \key{FALSE}
  \end{1=}
\end{0-1} \ldots

\begin{1=}
  \key{WHEN}
  selection-object
  \begin{0-1}
    \key{ALSO} selection-object
  \end{0-1}\ldots\ {}
  \imperativestatement
\end{1=} \ldots

\begin{0-1}
  \key{WHEN} \key{OTHER} \imperativestatement
\end{0-1}

\begin{0-1}
  \key{END-EVALUATE}
\end{0-1}

where selection-object is

\begin{1=}
  partial-\expression
  \begin{0-1}
    \begin{1=}
      \key{THROUGH} \\
      \key{THRU}
    \end{1=}
    \expression
  \end{0-1} \\

  \key{ANY} \\
  \key{TRUE} \\
  \key{FALSE}
\end{1=}

\section{EXIT statement}

\key{EXIT}
\begin{0-1}
  \key{FUNCTION} \\
  \key{PARAGRAPH} \\

  \key{PERFORM}
  \begin{0-1}
    \key{CYCLE}
  \end{0-1} \\

  \key{PROGRAM}
  \miscext{
    \begin{0-1}
      \begin{1=}
        \key{RETURNING} \\
        \key{GIVING}
      \end{1=}
    \end{0-1}
    \begin{1=}
      \identifier \\
      \literal
    \end{1=}
  } \\

  \key{SECTION} \\
\end{0-1}

\section{FREE statement}

\key{FREE}
\begin{1=}
  \identifier
\end{1=} \ldots

\section{GENERATE statement}

\pending{\key{GENERATE} \reportname}

\section{GO TO statement}

\key{GO} TO
\begin{1=}
  \procedurename
\end{1=} \ldots
\begin{0-1}
  \key{DEPENDING} ON \identifier
\end{0-1}

\section{GOBACK statement}

\key{GOBACK}
\begin{0-1}
  \begin{1=}
    \key{RETURNING} \\
    \miscext{\key{GIVING}}
  \end{1=}
  \begin{1=}
    \identifier \\
    \literal
  \end{1=}
\end{0-1}

\section{IF statement}

\key{IF} condition THEN
\begin{1=}
  \imperativestatement \\
  \key{ELSE} \imperativestatement
\end{1=} \ldots

\begin{0-1}
  \key{END-IF}
\end{0-1}

\section{INITIALIZE statement}

\begin{1=}
  \key{INITIALIZE} \\
  \miscext{\key{INITIALISE}}
\end{1=}
\begin{1=}
  \identifier \\
  basic-\literal
\end{1=} \ldots
\begin{0-1}
  WITH \key{FILLER}
\end{0-1}

\begin{0-1}
  \begin{1=}
    \key{ALL} \\
    \key{ALPHABETIC} \\
    \key{ALPHANUMERIC} \\
    \key{ALPHANUMERIC-EDITED} \\
    \key{NATIONAL} \\
    \key{NATIONAL-EDITED} \\
    \key{NUMERIC} \\
    \key{NUMERIC-EDITED}
  \end{1=}
  TO \key{VALUE}
\end{0-1}

\begin{0-1}
  \key{REPLACING}
  \begin{1=}
    \begin{1=}
      \key{ALPHABETIC} \\
      \key{ALPHANUMERIC} \\
      \key{ALPHANUMERIC-EDITED} \\
      \key{NATIONAL} \\
      \key{NATIONAL-EDITED} \\
      \key{NUMERIC} \\
      \key{NUMERIC-EDITED}
    \end{1=}
    DATA \key{BY}
    \begin{1=}
      \identifier \\
      \literal
    \end{1=}
  \end{1=} \ldots
\end{0-1}

\begin{0-1}
  THEN TO \key{DEFAULT}
\end{0-1}

\section{INITIATE statement}
\pending{
  \key{INITIATE}
  \begin{1=}
    \reportname
  \end{1=} \ldots
}

\section{INSPECT statement}

\key{INSPECT}
\begin{1=}
  \identifier \\
  \literal \\
  \functionname
\end{1=}
\begin{1=}
  tallying-phrase
  \begin{0-1}
    replacing-phrase
  \end{0-1} \\

  replacing-phrase \\
  converting-phrase
\end{1=}

where tallying-phrase is

\key{TALLYING}
\begin{1=}
  \begin{1=}
    \begin{1=}
      \identifier \\
      \literal
    \end{1=}
    \key{FOR}
    \begin{1=}
      \key{CHARACTERS} \\

      \begin{1=}
        \key{ALL} \\
        \key{LEADING} \\
        \key{TRAILING}
      \end{1=}
      \begin{1=}
        \identifier \\
        \literal
      \end{1=}
    \end{1=}
  \end{1=}\ldots
  \begin{0-1}
    before-after-phrase
  \end{0-1}
\end{1=} \ldots

where replacing-phrase is

\key{REPLACING}
\begin{1=}
  \begin{1=}
    \key{CHARACTERS} \\

    \begin{0-1}
      \key{ALL} \\
      \key{LEADING} \\
      \key{FIRST} \\
      \key{TRAILING}
    \end{0-1}
    \begin{1=}
      \identifier \\
      \literal
    \end{1=}
  \end{1=}
  \key{BY}
  \begin{1=}
    \identifier \\
    \literal \\
  \end{1=}
  \begin{0-1}
    before-after-phrase
  \end{0-1} \\
\end{1=} \ldots

where converting-phrase is

\key{CONVERTING}
\begin{1=}
  \identifier \\
  \literal
\end{1=}
\key{TO}
\begin{1=}
  \identifier \\
  \literal
\end{1=}
\begin{0-1}
  before-after-phrase
\end{0-1}

where before-after-phrase is

\begin{0+}
  \key{BEFORE} INITIAL
  \begin{1=}
    \identifier \\
    \literal
  \end{1=} \\

  \key{AFTER} INITIAL
  \begin{1=}
    \identifier \\
    \literal
  \end{1=}
\end{0+}

\section{MERGE statement}

\key{MERGE} \identifier
\begin{0-1}
  ON
  \begin{1=}
    \key{ASCENDING} \\
    \key{DESCENDING}
  \end{1=}
  KEY
  \begin{0-1}
    \identifier
  \end{0-1}\ldots
\end{0-1} \ldots

\begin{0-1}
  WITH \key{DUPLICATES}
  \begin{0-1}
    IN \key{ORDER}
  \end{0-1}
\end{0-1}

\begin{0-1}
  COLLATING \key{SEQUENCE} IS \identifier
\end{0-1}

\begin{0-1}
  \key{USING}
  \begin{1=}
    \filename
  \end{1=}\ldots
\end{0-1}

\begin{0-1}
  \key{GIVING}
  \begin{1=}
    \filename
  \end{1=}\ldots \\

  \key{OUTPUT} \key{PROCEDURE} IS
  \procedurename
  \begin{0-1}
    \begin{1=}
      \key{THROUGH} \\
      \key{THRU}
    \end{1=}
    \procedurename
  \end{0-1}
\end{0-1}

\section{MOVE statement}

\key{MOVE}
\begin{0-1}
  \key{CORRESPONDING} \\
  \key{CORR}
\end{0-1}
\begin{1=}
  \identifier \\
  \literal
\end{1=}
\key{TO}
\begin{1=}
  \identifier
\end{1=} \ldots

\section{MULTIPLY statement}

\format{simple}
\key{MULTIPLY}
\begin{1=}
  \identifier \\
  \literal
\end{1=}
\key{BY}
\begin{1=}
  \begin{1=}
    \identifier \\
    \literal
  \end{1=}
  \begin{0-1}
    rounded-phrase
  \end{0-1}
\end{1=} \ldots

\begin{0+}
  ON \key{SIZE} \key{ERROR} \imperativestatement \\
  \key{NOT} ON \key{SIZE} \key{ERROR} \imperativestatement
\end{0+}

\begin{0-1}
  \key{END-MULTIPLY}
\end{0-1}

\format{giving}
\key{MULTIPLY}
\begin{1=}
  \identifier \\
  \literal
\end{1=}
\key{BY}
\begin{1=}
  \identifier \\
  \literal
\end{1=}

\key{GIVING}
\begin{1=}
  \begin{1=}
    \identifier \\
    \literal
  \end{1=}
  \begin{0-1}
    rounded-phrase
  \end{0-1}
\end{1=} \ldots

\begin{0+}
  ON \key{SIZE} \key{ERROR} \imperativestatement \\
  \key{NOT} ON \key{SIZE} \key{ERROR} \imperativestatement
\end{0+}

\begin{0-1}
  \key{END-MULTIPLY}
\end{0-1}


\section{NEXT SENTENCE statement}

\archaic{
  \key{NEXT} \key{SENTENCE}
}

\section{OPEN statement}

\key{OPEN}
\begin{1=}
  \key{INPUT} \\
  \key{OUTPUT} \\
  \key{I-O} \\
  \key{EXTEND}
\end{1=}
\begin{0-1}
  \key{SHARING} WITH
  \begin{1=}
    \key{ALL} OTHER \\
    \key{NO} OTHER \\
    \key{READ} \key{ONLY}
  \end{1=}
\end{0-1}
\begin{1=}
  \filename
\end{1=} \ldots
\begin{0-1}
  WITH \key{NO} \key{REWIND} \\
  WITH \key{LOCK} \\
  \deleted{\key{REVERSED}}
\end{0-1}

\section{PERFORM statement}

\format{procedure}
\key{PERFORM} \procedurename
\begin{0-1}
  \begin{1=}
    \key{THROUGH} \\
    \key{THRU}
  \end{1=}
  \procedurename
\end{0-1}
\begin{0-1}
  \gnucobol{\key{FOREVER}} \\
  times-phrase \\
  until-phrase \\
  varying-phrase
\end{0-1}

\format{inline}
\key{PERFORM}
\begin{0-1}
  \gnucobol{\key{FOREVER}} \\
  times-phrase \\
  until-phrase \\
  varying-phrase
\end{0-1}
\imperativestatement
\begin{0-1}
  \key{END-PERFORM}
\end{0-1}

where times-phrase is

\begin{1=}
  \identifier \\
  \literal \\
  \functionname
\end{1=}
\key{TIMES} \\

where until-phrase is

\begin{0-1}
  WITH \key{TEST}
  \begin{1=}
    \key{BEFORE} \\
    \key{AFTER} \\
  \end{1=}
\end{0-1}
\key{UNTIL}
\begin{1=}
  \condition \\
  \gnucobol{\key{EXIT}}
\end{1=} \\

and where varying-phrase is

\begin{0-1}
  WITH \key{TEST}
  \begin{1=}
    \key{BEFORE} \\
    \key{AFTER} \\
  \end{1=}
\end{0-1}

\key{VARYING} \identifier \key{FROM}
\begin{1=}
  \identifier \\
  \literal
\end{1=}
\key{BY}
\begin{1=}
  \identifier \\
  \literal
\end{1=}
\key{UNTIL}
\condition

\begin{0-1}
  \key{AFTER} \identifier \key{FROM}
  \begin{1=}
    \identifier \\
    \literal
  \end{1=}
  \key{BY}
  \begin{1=}
    \identifier \\
    \literal
  \end{1=}

  \key{UNTIL}
  \condition
\end{0-1} \ldots


% \begin{0-1}
%   \key{\begin{0-1}
%   \key{END-PERFORM}
% \end{0-1}} \\
%   .
% \end{0-1}

% TO-DO: Improve

\section{READ statement}

\key{READ} \filename
\begin{0-1}
  \key{NEXT} \\
  \key{PREVIOUS}
\end{0-1}
RECORD
\begin{0-1}
  \key{INTO} \identifier
\end{0-1}

\begin{0-1}
  \key{IGNORING} \key{LOCK} \\

  WITH
  \begin{1=}
    \key{KEPT} \\
    \key{NO} \\
    \key{IGNORE} \\
  \end{1=}
  \key{LOCK} \\

  WITH \key{WAIT}
\end{0-1}

\begin{0-1}
  \key{KEY} IS \identifier
\end{0-1}

\begin{0-1}
  \begin{1+}
    \key{INVALID} \key{KEY} \imperativestatement \\
    \key{NOT} \key{INVALID} \key{KEY} \imperativestatement
  \end{1+} \\

  \begin{1+}
    AT \key{END} \imperativestatement \\
    \key{NOT} AT \key{END} \imperativestatement
  \end{1+}
\end{0-1}

\begin{0-1}
  \key{END-READ}
\end{0-1}

\section{READY statement}

\miscext{\key{READY} \key{TRACE}}

\section{RELEASE statement}

\key{RELEASE} \identifier
\begin{0-1}
  \key{FROM}
  \begin{1=}
    \identifier \\
    \literal \\
    function-call-1
  \end{1=}
\end{0-1}

\section{RESET statement}

\miscext{\key{RESET} \key{TRACE}}

\section{RETURN statement}

\key{RETURN} \filename RECORD
\begin{0-1}
  \key{INTO} \identifier
\end{0-1}

AT \key{END} \imperativestatement

\begin{0-1}
  \key{NOT} AT \key{END} \imperativestatement
\end{0-1}

\begin{0-1}
  \key{END-RETURN}
\end{0-1}

\section{REWRITE statement}

\key{REWRITE}
\recordname
\begin{0-1}
  \key{FROM}
  \begin{1=}
    \identifier \\
    \literal \\
    \functionname
  \end{1=}
\end{0-1}
\begin{0-1}
  WITH
  \begin{0-1}
    \key{NO}
  \end{0-1}
  \key{LOCK}
\end{0-1}

\begin{0+}
  \key{INVALID} \key{KEY} \imperativestatement \\
  \key{NOT} \key{INVALID} \key{KEY} \imperativestatement
\end{0+} \\

\begin{0-1}
  \key{END-REWRITE}
\end{0-1}

\section{ROLLBACK statement}

\miscext{\key{ROLLBACK}}

\section{SEARCH statement}

\format{simple}
\key{SEARCH} \identifier
\begin{0-1}
  \key{VARYING} \identifier
\end{0-1}

\begin{0-1}
  AT \key{END} \imperativestatement
\end{0-1}

\begin{1=}
  \key{WHEN} \condition \imperativestatement
\end{1=} \ldots

\begin{0-1}
  \key{END-SEARCH}
\end{0-1}

\format{all}
\key{SEARCH} \key{ALL} \identifier

\begin{0-1}
  AT \key{END} \imperativestatement
\end{0-1}

\key{WHEN} \expression \imperativestatement

\begin{0-1}
  \key{END-SEARCH}
\end{0-1}


\section{SET statement}

\format{simple}
\key{SET} \identifier \key{TO}
\begin{1=}
  \identifier \\
  \literal \\
  \arithmeticexpression
\end{1=}

\format{entry}
\gnucobol{
  \key{SET} \identifier \key{TO} \key{ENTRY}
  \begin{1=}
    \identifier \\
    \literal
  \end{1=}
}

\format{environment}
\miscext{
  \key{SET} \key{ENVIRONMENT}
  \begin{1=}
    \identifier \\
    \literal
  \end{1=}
  \key{TO}
  \begin{1=}
    \identifier \\
    \literal
  \end{1=}
}

\format{attribute}
\key{SET} \identifier \key{ATTRIBUTE}
\begin{1=}
  \begin{1=}
    \begin{1=}
      \key{BELL} \\
      \key{BEEP}
    \end{1=} \\

    \key{BLINK} \\
    \key{HIGHLIGHT} \\
    \key{LOWLIGHT} \\
    \key{REVERSE-VIDEO} \\
    \key{UNDERLINE} \\
    \key{LEFTLINE} \\
    \key{OVERLINE}
  \end{1=}
  \begin{1=}
    \key{ON} \\
    \key{OFF}
  \end{1=}
\end{1=}\ldots

\format{arithmetic}
\key{SET}
\begin{1=}
  \cobolindexname
\end{1=}\ldots
\begin{1=}
  \key{UP} \\
  \key{DOWN}
\end{1=}
\key{BY}
\arithmeticexpression

\format{on\slash{}off}
\key{SET}
\begin{1=}
  \begin{1=}
    \mnemonicname
  \end{1=}\ldots
  \key{TO}
  \begin{1=}
    \key{ON} \\
    \key{OFF}
  \end{1=}
\end{1=} \ldots

\format{true\slash{}false}
\key{SET}
\begin{1=}
  \begin{1=}
    \conditionname
  \end{1=}\ldots
  \key{TO}
  \begin{1=}
    \key{TRUE} \\
    \key{FALSE}
  \end{1=}
\end{1=} \ldots

\format{exception}
\key{SET} \key{LAST} \key{EXCEPTION} \key{TO} \key{OFF}

\section{SORT statement}

\key{SORT} \identifier
\begin{0-1}
  ON
  \begin{1=}
    \key{ASCENDING} \\
    \key{DESCENDING}
  \end{1=}
  KEY
  \begin{0-1}
    \identifier
  \end{0-1}\ldots
\end{0-1} \ldots

\begin{0-1}
  WITH \key{DUPLICATES}
  \begin{0-1}
    IN \key{ORDER}
  \end{0-1}
\end{0-1}

\begin{0-1}
  COLLATING \key{SEQUENCE} IS \identifier
\end{0-1}

\begin{0-1}
  \key{USING}
  \begin{1=}
    \filename
  \end{1=}\ldots \\

  \key{INPUT} \key{PROCEDURE} IS
  \procedurename
  \begin{0-1}
    \begin{1=}
      \key{THROUGH} \\
      \key{THRU}
    \end{1=}
    \procedurename
  \end{0-1}
\end{0-1}

\begin{0-1}
  \key{GIVING}
  \begin{1=}
    \filename
  \end{1=}\ldots \\

  \key{OUTPUT} \key{PROCEDURE} IS
  \procedurename
  \begin{0-1}
    \begin{1=}
      \key{THROUGH} \\
      \key{THRU}
    \end{1=}
    \procedurename
  \end{0-1}
\end{0-1}

\section{START statement}

\key{START} \filename
\begin{0-1}
  \key{FIRST} \\

  \key{KEY} IS
  relational-operator
  \identifier\\

  \key{LAST}
\end{0-1}

\begin{0-1}
  WITH
  \begin{1=}
    \key{SIZE} \\
    \gnucobol{\key{LENGTH}}
  \end{1=}
  \arithmeticexpression
\end{0-1}

\begin{0+}
  \key{INVALID} KEY \imperativestatement \\
  \key{NOT} \key{INVALID} KEY \imperativestatement
\end{0+}

\begin{0-1}
  \key{END-START}
\end{0-1}

\section{STOP statement}

\format{standard}
\ {}\newline
\key{STOP} \key{RUN}
\begin{0-1}
  \begin{1=}
    \key{RETURNING} \\
    \miscext{\key{GIVING}}
  \end{1=}
  \begin{1=}
    \identifier \\
    \literal
  \end{1=} \\

  WITH
  \begin{1=}
    \key{ERROR} \\
    \key{NORMAL}
  \end{1=}
  STATUS
  \begin{0-1}
    \identifier \\
    \literal
  \end{0-1}
\end{0-1}

\format{literal}
\ {}\newline
\deleted{\key{STOP} \literal}

\format{ACUCOBOL}
\ {}\newline
\miscext{
  \key{STOP} \key{RUN}
  \begin{1=}
    \identifier \\
    \literal
  \end{1=}
}

\section{STRING statement}

\key{STRING}
\begin{1=}
  \begin{1=}
    \identifier \\
    \literal
  \end{1=} \\

  \key{DELIMITED} BY
  \begin{1=}
    \key{SIZE} \\
    \identifier \\
    \literal
  \end{1=}
\end{1=} \ldots
\key{INTO} \identifier

\begin{0-1}
  WITH \key{POINTER} IS \identifier
\end{0-1}

\begin{0+}
  ON \key{OVERFLOW} \imperativestatement \\
  \key{NOT} ON \key{OVERFLOW} \imperativestatement
\end{0+}

\section{SUBTRACT statement}

\format{simple}
\key{SUBTRACT}
\begin{1=}
  \identifier \\
  \literal
\end{1=} \ldots
\key{FROM}
\begin{1=}
  \begin{1=}
    \identifier \\
    \literal
  \end{1=}
  \begin{0-1}
    rounded-phrase
  \end{0-1}
\end{1=} \ldots

\begin{0+}
  ON \key{SIZE} \key{ERROR} \imperativestatement \\
  \key{NOT} ON \key{SIZE} \key{ERROR} \imperativestatement
\end{0+}

\begin{0-1}
  \key{END-SUBTRACT}
\end{0-1}

\format{giving}
\key{SUBTRACT}
\begin{1=}
  \identifier \\
  \literal
\end{1=} \ldots
\key{FROM}
\begin{1=}
  \identifier \\
  \literal
\end{1=}

\key{GIVING}
\begin{1=}
  \begin{1=}
    \identifier \\
    \literal
  \end{1=}
  \begin{0-1}
    rounded-phrase
  \end{0-1}
\end{1=} \ldots

\begin{0+}
  ON \key{SIZE} \key{ERROR} \imperativestatement \\
  \key{NOT} ON \key{SIZE} \key{ERROR} \imperativestatement
\end{0+}

\begin{0-1}
  \key{END-SUBTRACT}
\end{0-1}

\format{corresponding}
\key{SUBTRACT}
\begin{1=}
  \key{CORR} \\
  \key{CORRESPONDING}
\end{1=}
\identifier{} \key{FROM} \identifier
\begin{0-1}
  rounded-phrase
\end{0-1}

\begin{0+}
  ON \key{SIZE} \key{ERROR} \imperativestatement \\
  \key{NOT} ON \key{SIZE} \key{ERROR} \imperativestatement
\end{0+}

\begin{0-1}
  \key{END-SUBTRACT}
\end{0-1}

\section{SUPPRESS statement}

\pending{
  \key{SUPPRESS} PRINTING
}

\section{TERMINATE statement}

\pending{
  \key{TERMINATE}
  \begin{1=}
    \reportname
  \end{1=} \ldots
}

\section{TRANSFORM statement}

\deleted{
  \key{TRANSFORM} \identifier \key{FROM}
  \begin{1=}
    \identifier \\
    \literal
  \end{1=}
  \key{TO}
  \begin{1=}
    \identifier \\
    \literal
  \end{1=}
}

\section{UNLOCK statement}

\key{UNLOCK} \filename
\begin{0-1}
  \key{RECORD} \\
  \key{RECORDS}
\end{0-1}

\section{UNSTRING statement}

\key{UNSTRING} \identifier

\begin{0-1}
  \key{DELIMITED} BY
  \begin{0-1}
    \key{ALL}
  \end{0-1}
  \begin{1=}
    \identifier \\
    \literal
  \end{1=}
  \begin{1=}
    \key{OR}
    \begin{0-1}
      \key{ALL}
    \end{0-1}
    \begin{1=}
      \identifier \\
      \literal
    \end{1=}
  \end{1=} \ldots
\end{0-1}

\key{INTO}
\begin{1=}
  \identifier
  \begin{0-1}
    \key{DELIMITER} IN \identifier
  \end{0-1}
  \begin{0-1}
    \key{COUNT} IN \identifier
  \end{0-1}
\end{1=} \ldots

\begin{0-1}
  WITH \key{POINTER} IS \identifier
\end{0-1}

\begin{0-1}
  \key{TALLYING} IN \identifier
\end{0-1}

\begin{0+}
  ON \key{OVERFLOW} \imperativestatement \\
  \key{NOT} ON \key{OVERFLOW} \imperativestatement
\end{0+}

\begin{0-1}
  \key{END-OVERFLOW}
\end{0-1}

\section{USE statement}

\format{file exception}
\key{USE}
\begin{0-1}
  \key{GLOBAL}
\end{0-1}
AFTER STANDARD
\begin{1=}
  \key{EXCEPTION} \\
  \key{ERROR}
\end{1=}
PROCEDURE ON

\begin{1=}
  \begin{1=}
    \filename
  \end{1=} \ldots
  \begin{0+}
    \key{INPUT} \\
    \key{OUTPUT} \\
    \key{I-O} \\
    \key{EXTEND}
  \end{0+} \ldots
\end{1=}

\format{debugging}
\deleted{
  \key{USE} FOR \key{DEBUGGING} ON
  \begin{1=}
    \procedurename \\
    \key{ALL} \key{PROCEDURES} \\
    \key{ALL} REFERENCES OF \identifier
  \end{1=} \ldots
}

\format{start\slash{}end}
\miscext{\pending{
  \key{USE} AT \key{PROGRAM}
  \begin{1=}
    \key{START} \\
    \key{END}
  \end{1=}
}}

\format{reporting}
\key{USE}
\begin{0-1}
  \key{GLOBAL}
\end{0-1}
\key{BEFORE} \key{REPORTING} \identifier

\format{exception}
\pending{
  \key{USE}
  \begin{1=}
    \key{EXCEPTION-CONDITION} \\
    \key{EC}
  \end{1=}
}

\section{WRITE statement}

\format{sequential}
\key{WRITE} \recordname
\begin{0-1}
  \key{FROM}
  \begin{1=}
    \identifier \\
    \literal \\
    \functionname
  \end{1=}
\end{0-1}

\begin{0-1}
  \begin{1=}
    \key{BEFORE} \\
    \key{AFTER}
  \end{1=}
  ADVANCING
  \begin{1=}
    \begin{1=}
      \identifier \\
      \literal
    \end{1=}
    \begin{0-1}
      \key{LINE} \\
      \key{LINES}
    \end{0-1} \\

    \mnemonicname \\

    \key{PAGE}
  \end{1=}
\end{0-1}

\begin{0-1}
  WITH
  \begin{0-1}
    \key{NO}
  \end{0-1}
  \key{LOCK}
\end{0-1}

\begin{0+}
  AT
  \begin{1=}
    \key{END-OF-PAGE} \\
    \key{EOP}
  \end{1=}
  \imperativestatement \\

  \key{NOT} AT
  \begin{1=}
    \key{END-OF-PAGE} \\
    \key{EOP}
  \end{1=}
  \imperativestatement
\end{0+}

\begin{0-1}
  \key{END-WRITE}
\end{0-1}

\format{other}
\key{WRITE} \recordname
\begin{0-1}
  \key{FROM}
  \begin{1=}
    \identifier \\
    \literal \\
    \functionname
  \end{1=}
\end{0-1}

\begin{0-1}
  \begin{1=}
    \key{BEFORE} \\
    \key{AFTER}
  \end{1=}
  ADVANCING
  \begin{1=}
    \begin{1=}
      \identifier \\
      \literal
    \end{1=}
    \begin{0-1}
      \key{LINE} \\
      \key{LINES}
    \end{0-1} \\

    \mnemonicname \\

    \key{PAGE}
  \end{1=}
\end{0-1}

\begin{0-1}
  WITH
  \begin{0-1}
    \key{NO}
  \end{0-1}
  \key{LOCK}
\end{0-1}

\begin{0+}
  \key{INVALID} \key{KEY} \imperativestatement \\
  \key{NOT} \key{INVALID} \key{KEY} \imperativestatement
\end{0+} \\

\begin{0-1}
  \key{END-WRITE}
\end{0-1}

\chapter{Intrinsic functions}

\section{ABS function}

\key{FUNCTION} \key{ABS} ( \argument )

\section{ACOS function}

\key{FUNCTION} \key{ACOS} ( \argument)

\section{ANNUITY function}

\key{FUNCTION} \key{ANNUITY} ( \argument \argument )

\section{ASIN function}

\key{FUNCTION} \key{ASIN} ( \argument )

\section{ATAN function}

\key{FUNCTION} \key{ATAN} ( \argument )

\section{BOOLEAN-OF-INTEGER function}

\pending{
  \key{FUNCTION} \key{BOOLEAN-OF-INTEGER} ( \argument \argument )
}

\section{BYTE-LENGTH function}

\key{FUNCTION} \key{BYTE-LENGTH} ( \argument )

\section{CHAR function}

\key{FUNCTION} \key{CHAR} ( \argument )

\section{CHAR-NATIONAL function}

\pending{
  \key{FUNCTION} \key{CHAR-NATIONAL} ( \argument )
}

\section{COMBINED-DATETIME function}

\key{FUNCTION} \key{COMBINED-DATETIME} ( \argument \argument )

\section{CONCATENATE function}

\key{FUNCTION} \key{CONCATENATE} (
\begin{1=}
  \argument
\end{1=}
\ldots
\ {})

\section{COS function}

\key{FUNCTION} \key{COS} ( \argument )

\section{CURRENCY-SYMBOL function}

\gnucobol{
  \key{FUNCTION} \key{CURRENCY-SYMBOL}
}

\section{CURRENT-DATE function}

\key{FUNCTION} \key{CURRENT-DATE}

\section{DATE-OF-INTEGER function}

\key{FUNCTION} \key{DATE-OF-INTEGER} ( \argument )

\section{DATE-TO-YYYYMMDD function}

\key{FUNCTION} \key{DATE-TO-YYYYMMDD} (
\argument
\begin{0-1}
  \argument
  \begin{0-1}
    \argument
  \end{0-1}
\end{0-1}
\ {})

\section{DAY-OF-INTEGER function}

\key{FUNCTION} \key{DAY-OF-INTEGER} ( \argument )

\section{DAY-TO-YYYYDDD function}

\key{FUNCTION} \key{DAY-TO-YYYYDDD} (
\argument
\begin{0-1}
  \argument
  \begin{0-1}
    \argument
  \end{0-1}
\end{0-1}
\ {})

\section{DISPLAY-OF function}

\pending{
  \key{FUNCTION} \key{DISPLAY-OF} ( \argument )
}

\section{E function}

\key{FUNCTION} \key{E}

\section{EXCEPTION-FILE function}

\key{FUNCTION} \key{EXCEPTION-FILE}

\section{EXCEPTION-FILE-N function}

\pending{
  \key{FUNCTION} \key{EXCEPTION-FILE-N}
}

\section{EXCEPTION-LOCATION function}

\key{FUNCTION} \key{EXCEPTION-LOCATION}

\section{EXCEPTION-LOCATION-N function}

\pending{
  \key{FUNCTION} \key{EXCEPTION-LOCATION-N}
}

\section{EXCEPTION-STATEMENT function}

\key{FUNCTION} \key{EXCEPTION-STATEMENT}

\section{EXCEPTION-STATUS function}

\key{FUNCTION} \key{EXCEPTION-STATUS}

\section{EXP function}

\key{FUNCTION} \key{EXP} ( \argument )

\section{EXP10 function}

\key{FUNCTION} \key{EXP10} ( \argument )

\section{FACTORIAL function}

\key{FUNCTION} \key{FACTORIAL} ( \argument )

\section{FORMATTED-CURRENT-DATE function}

\key{FUNCTION} \key{FORMATTED-CURRENT-DATE} ( \argument )

\section{FORMATTED-DATE function}

\key{FUNCTION} \key{FORMATTED-DATE} ( \argument \argument)

\section{FORMATTED-DATETIME function}

\key{FUNCTION} \key{FORMATTED-DATETIME}

( \argument \argument \argument
\begin{0-1}
  \argument \\
  \gnucobol{\key{SYSTEM-OFFSET}}
\end{0-1}
)

\section{FORMATTED-TIME function}

\key{FUNCTION} \key{FORMATTED-TIME} ( \argument \argument
\begin{0-1}
  \argument \\
  \gnucobol{\key{SYSTEM-OFFSET}}
\end{0-1}
)

\section{FRACTION-PART function}

\key{FUNCTION} \key{FRACTION-PART} ( \argument )

\section{HIGHEST-ALGEBRAIC function}

\key{FUNCTION} \key{HIGHEST-ALGEBRAIC} ( \argument )

\section{INTEGER function}

\key{FUNCTION} \key{INTEGER} ( \argument )

\section{INTEGER-OF-BOOLEAN function}

\pending{
  \key{FUNCTION} \key{INTEGER-OF-BOOLEAN} ( \argument )
}

\section{INTEGER-OF-DATE function}

\key{FUNCTION} \key{INTEGER-OF-DATE} ( \argument )

\section{INTEGER-OF-DAY function}

\key{FUNCTION} \key{INTEGER-OF-DAY} ( \argument )

\section{INTEGER-OF-FORMATTED-DATE function}

\key{FUNCTION} \key{INTEGER-OF-FORMATTED-DATE} ( \argument \argument )

\section{INTEGER-PART function}

\key{FUNCTION} \key{INTEGER-PART} ( \argument )

\section{LENGTH function}

\key{FUNCTION} \key{LENGTH} ( \argument )

\section{LENGTH-AN function}

\miscext{
  \key{FUNCTION} \key{LENGTH-AN} ( \argument )
}

\section{LOCALE-COMPARE function}

\key{FUNCTION} \key{LOCALE-COMPARE} ( \argument \argument
\begin{0-1}
  \argument
\end{0-1}
)

\section{LOCALE-DATE function}

\key{FUNCTION} \key{LOCALE-DATE} ( \argument
\begin{0-1}
  \argument
\end{0-1}
)

\section{LOCALE-TIME function}

\key{FUNCTION} \key{LOCALE-TIME} ( \argument
\begin{0-1}
  \argument
\end{0-1}
)

\section{LOCALE-TIME-FROM-SECONDS function}

\key{FUNCTION} \key{LOCALE-TIME-FROM-SECONDS} ( \argument
\begin{0-1}
  \argument
\end{0-1}
)

\section{LOG function}

\key{FUNCTION} \key{LOG} ( \argument )

\section{LOG10 function}

\key{FUNCTION} \key{LOG10} ( \argument )

\section{LOWER-CASE function}

\key{FUNCTION} \key{LOWER-CASE} ( \argument )

\section{LOWEST-ALGEBRAIC function}

\key{FUNCTION} \key{LOWEST-ALGEBRAIC} ( \argument )

\section{MAX function}

\key{FUNCTION} \key{MAX} (
\begin{1=}
  \argument
\end{1=}\ldots
\ {})

\section{MEAN function}

\key{FUNCTION} \key{MEAN} (
\begin{1=}
  \argument
\end{1=}\ldots
\ {})

\section{MEDIAN function}

\key{FUNCTION} \key{MEDIAN} (
\begin{1=}
  \argument
\end{1=}\ldots
\ {})

\section{MIDRANGE function}

\key{FUNCTION} \key{MIDRANGE} (
\begin{1=}
  \argument
\end{1=}\ldots
\ {})

\section{MIN function}

\key{FUNCTION} \key{MIN} (
\begin{1=}
  \argument
\end{1=}\ldots
\ {})

\section{MOD function}

\key{FUNCTION} \key{MOD} ( \argument \argument )

\section{MODULE-CALLER-ID function}

\gnucobol{
  \key{FUNCTION} \key{MODULE-CALLER-ID}
}

\section{MODULE-DATE function}

\gnucobol{
  \key{FUNCTION} \key{MODULE-DATE}
}

\section{MODULE-FORMATTED-DATE function}

\gnucobol{
  \key{FUNCTION} \key{MODULE-FORMATTED-DATE}
}

\section{MODULE-ID function}

\gnucobol{
  \key{FUNCTION} \key{MODULE-ID}
}

\section{MODULE-PATH function}

\gnucobol{
  \key{FUNCTION} \key{MODULE-PATH}
}

\section{MODULE-SOURCE function}

\gnucobol{
  \key{FUNCTION} \key{MODULE-SOURCE}
}

\section{MODULE-TIME function}

\gnucobol{
  \key{FUNCTION} \key{MODULE-TIME}
}

\section{MONETARY-DECIMAL-POINT function}

\gnucobol{
  \key{FUNCTION} \key{MONETARY-DECIMAL-POINT}
}

\section{MONETARY-THOUSANDS-SEPARATOR function}

\gnucobol{
  \key{FUNCTION} \key{MONETARY-THOUSANDS-SEPARATOR}
}

\section{NATIONAL-OF function}

\pending{
  \key{FUNCTION} \key{NATIONAL-OF} ( \argument
  \begin{0-1}
    \argument
  \end{0-1}
  )
}

\section{NUMERIC-DECIMAL-POINT function}

\gnucobol{
  \key{FUNCTION} \key{NUMERIC-DECIMAL-POINT}
}

\section{NUMERIC-THOUSANDS-SEPARATOR function}

\gnucobol{
  \key{FUNCTION} \key{NUMERIC-THOUSANDS-SEPARATOR}
}

\section{NUMVAL function}

\key{FUNCTION} \key{NUMVAL} ( \argument )

\section{NUMVAL-C function}

\key{FUNCTION} \key{NUMVAL-C} ( \argument
\begin{0-1}
  \argument
\end{0-1}
)

\section{NUMVAL-F function}

\key{FUNCTION} \key{NUMVAL-F} ( \argument )

\section{ORD function}

\key{FUNCTION} \key{ORD} ( \argument )

\section{ORD-MAX function}

\key{FUNCTION} \key{ORD-MAX} (
\begin{1=}
  \argument
\end{1=} \ldots
\ {})

\section{ORD-MIN function}

\key{FUNCTION} \key{ORD-MIN} (
\begin{1=}
  \argument
\end{1=} \ldots
\ {})

\section{PI function}

\key{FUNCTION} \key{PI}

\section{PRESENT-VALUE function}

\key{FUNCTION} \key{PRESENT-VALUE} (
\begin{1=}
  \argument
\end{1=} \ldots
\ {})

\section{RANDOM function}

\key{FUNCTION} \key{RANDOM}
\begin{0-1}
  (
  \begin{0-1}
    \argument
  \end{0-1} \gnucobol{\ldots}\ {}
  )
\end{0-1}

\section{RANGE function}

\key{FUNCTION} \key{RANGE} (
\begin{1=}
  \argument
\end{1=}\ldots
\ {})

\section{REM function}

\key{FUNCTION} \key{REM} ( \argument \argument )

\section{REVERSE function}

\key{FUNCTION} \key{REVERSE} ( \argument )

\section{SECONDS-FROM-FORMATTED-TIME function}

\key{FUNCTION} \key{SECONDS-FROM-FORMATTED-TIME}

( \argument \argument)

\section{SECONDS-PAST-MIDNIGHT function}

\key{FUNCTION} \key{SECONDS-PAST-MIDNIGHT} ( \argument )

\section{SIGN function}

\key{FUNCTION} \key{SIGN} ( \argument )

\section{SIN function}

\key{FUNCTION} \key{SIN} ( \argument )

\section{SQRT function}

\key{FUNCTION} \key{SQRT} ( \argument )

\section{STANDARD-COMPARE function}

\pending{
  \key{FUNCTION} \key{STANDARD-COMPARE}
}

\pending{
  ( \argument \argument
  \begin{0-1}
    \argument
  \end{0-1}
  \begin{0-1}
    \argument
  \end{0-1}
  )
}

\section{STANDARD-DEVIATION function}

\key{FUNCTION} \key{STANDARD-DEVIATION} (
\begin{1=}
  \argument
\end{1=}\ldots
\ {})

\section{STORED-CHAR-LENGTH function}

\gnucobol{
  \key{FUNCTION} \key{STORED-CHAR-LENGTH} ( \argument )
}

\section{SUBSTITUTE function}

\gnucobol{
  \key{FUNCTION} \key{SUBSTITUTE} ( \argument
  \begin{1=}
    \argument \argument
  \end{1=}\ldots\ {}
  )
}

\section{SUBSTITUTE-CASE function}

\gnucobol{
  \key{FUNCTION} \key{SUBSTITUTE-CASE} ( \argument
  \begin{1=}
    \argument \argument
  \end{1=}\ldots\ {}
  )
}

\section{SUM function}

\key{FUNCTION} \key{SUM} (
\begin{1=}
  \argument
\end{1=}\ldots
\ {})

\section{TAN function}

\key{FUNCTION} \key{TAN} ( \argument )

\section{TEST-DATE-YYYYMMDD function}

\key{FUNCTION} \key{TEST-DATE-YYYYMMDD} ( \argument )

\section{TEST-DAY-YYYYDDD function}

\key{FUNCTION} \key{TEST-DAY-YYYYDDD} ( \argument )

\section{TEST-FORMATTED-DATETIME function}

\key{FUNCTION} \key{TEST-FORMATTED-DATETIME} ( \argument \argument )

\section{TEST-NUMVAL function}

\key{FUNCTION} \key{TEST-NUMVAL} ( \argument )

\section{TEST-NUMVAL-C function}

\key{FUNCTION} \key{TEST-NUMVAL-C} ( \argument \argument )

\section{TEST-NUMVAL-F function}

\key{FUNCTION} \key{TEST-NUMVAL-F} ( \argument )

\section{TRIM function}

\key{FUNCTION} \key{TRIM} ( \argument
\begin{0-1}
  \key{LEADING} \\
  \key{TRAILING}
\end{0-1}
)

\section{UPPER-CASE function}

\key{FUNCTION} \key{UPPER-CASE} ( \argument )

\section{VARIANCE function}

\key{FUNCTION} \key{VARIANCE} (
\begin{1=}
  \argument
\end{1=}\ldots
\ {})

\section{WHEN-COMPILED function}

\key{FUNCTION} \key{WHEN-COMPILED}

\section{YEAR-TO-YYYY function}

\key{FUNCTION} \key{YEAR-TO-YYYY} ( \argument
\begin{0-1}
  \argument
  \begin{0-1}
    \argument
  \end{0-1}
\end{0-1}
)

\begin{appendices}

  \include{gfdl}
\end{appendices}

\end{document}
