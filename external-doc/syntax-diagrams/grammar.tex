\documentclass[a4paper,oneside,svgnames]{scrbook}

\usepackage{microtype}
\usepackage[toc,page]{appendix}
\usepackage{booktabs}
\usepackage{textcomp}

% for arithmetic expression table
\usepackage{pifont}
\newcommand{\tick}{\ding{51}}
\usepackage{multirow}

% Erewhon is for math mode.
\usepackage[proportional,scaled=1.064]{erewhon}
\usepackage[erewhon,vvarbb,bigdelims]{newtxmath}
\usepackage{roboto}
\usepackage[T1]{fontenc}
\renewcommand*\oldstylenums[1]{\textosf{#1}}

\usepackage[dvipsnames]{xcolor}
\usepackage{afterpage}
\usepackage{tikz}
\usepackage[printwatermark]{xwatermark}
\usepackage[tocindentauto]{tocstyle}
\usetocstyle{standard}
\usepackage[most]{tcolorbox} % for coloured boxes and warning boxes
\usepackage[pdfauthor={Edward Hart},
            pdftitle={The GnuCOBOL 2.0 Grammar},
            pdfkeywords={GnuCOBOL,COBOL,manual,grammar,guide},
            colorlinks]{hyperref}

\newcommand{\clearpageifnotfirst}[1]{%
  \ifnum\value{#1}>0 \clearpage{} \fi%
}

\newtcolorbox{streak}[1][]{frame code={}
  halign title=flush center,
  left=40mm,
  right=40mm,
  top=10pt,
  bottom=10pt,
  colback=Maroon,
  colupper=white,
  left skip=40mm,
  grow to left by=40mm,
  right skip=40mm,
  grow to right by=40mm,
  width=\paperwidth,
  text width=\textwidth,
  arc=0pt,outer arc=0pt,
  #1
}

% \newtcolorbox{syntaxbox}[1][]{enhanced,
%   before skip=2mm,after skip=3mm,
%   boxrule=0.4pt,left=5mm,right=2mm,top=1mm,bottom=1mm,
%   colback=yellow!50,
%   colframe=yellow!20!black,
%   sharp corners,rounded corners=southeast,arc is angular,arc=3mm,
%   underlay={%
%     \path[fill=tcbcol@back!80!black] ([yshift=3mm]interior.south east)--++(-0.4,-0.1)--++(0.1,-0.2);
%     \path[draw=tcbcol@frame,shorten <=-0.05mm,shorten >=-0.05mm] ([yshift=3mm]interior.south east)--++(-0.4,-0.1)--++(0.1,-0.2);
%     \path[fill=yellow!50!black,draw=none] (interior.south west) rectangle node[white]{\Huge\bfseries !} ([xshift=4mm]interior.north west);
%     },
%   drop fuzzy shadow}

\newtcolorbox{syntaxbox}[1][]{enhanced,
  boxrule=0.4pt,left=1mm,right=1mm,top=1mm,bottom=1mm,
  sharp corners,% Rectangular shape
  halign=flush left,% To prevent spread out lines
  parbox=false,% To give normal paragraph indentation.
  before upper={\parskip3pt\parindent15pt}%
}

\renewcommand{\arraystretch}{1.2}
\def\defaultparskip{\parskip}

\newenvironment{syntax}{%
  \begin{syntaxbox}\small\selectfont\setlength{\parskip}{\baselineskip}}{%
  \setlength{\parskip}{\defaultparskip}\normalsize\selectfont\end{syntaxbox}}

\renewcommand{\familydefault}{\sfdefault}

\newcommand{\metaelement}[1]{%
  \textit{#1}%
}

\newcommand{\newlexicalelement}[1]{%
  \newcounter{#1}[subsection]%
  \expandafter\newcommand\csname #1\endcsname{%
    \stepcounter{#1}%
    \metaelement{#1-\arabic{#1}}\ {}%
  }%
}

\newcommand{\newlexicalelementwithname}[2]{%
  \newcounter{#1}[subsection]%
  \expandafter\newcommand\csname #1\endcsname{%
    \stepcounter{#1}%
    \metaelement{#2-\arabic{#1}}\ {}%
  }%
}

\newcounter{format}[subsection]
\newcommand{\format}[1]{%
  \stepcounter{format}%
  \paragraph{Format \arabic{format}\vspace{1em}\ {}(#1)}\ {}\newline%
}

\newcommand{\key}[1]{\underline{#1}}
\newcommand{\directiveindicator}[0]{>{}>}
\newcommand{\deleted}[1]{%
  \colorbox{red!75}{#1}}
\newcommand{\archaic}[1]{%
  \colorbox{pink}{#1}}
\newcommand{\obsolete}[1]{%
  \colorbox{red!60}{#1}}
\newcommand{\xopen}[1]{%
  \colorbox{green!75}{#1}}
\newcommand{\gnucobol}[1]{%
  \colorbox{orange!75}{#1}}
\newcommand{\miscext}[1]{%
  \colorbox{blue!50}{#1}}
\newcommand{\pending}[1]{%
  \textcolor{gray!75}{#1}}
\newcommand{\standard}[1]{%
  \colorbox{white}{#1}}

\newenvironment{0-1}{$\left[ \begin{tabular}{@{}l@{}}}{\end{tabular} \right]$}
\newenvironment{0+}{$\left[\left| \begin{tabular}{@{}l@{}}}{\end{tabular} \right|\right]$}
\newenvironment{1=}{$\left\{ \begin{tabular}{@{}l@{}}}{\end{tabular} \right\}$}
\newenvironment{1+}{$\left\{\left| \begin{tabular}{@{}l@{}}}{\end{tabular} \right|\right\}$}

\begin{document}

\newsavebox\mybox
\savebox\mybox{\tikz[color=gray,opacity=0.3]\node{DRAFT};}
% \newwatermark*[
%   allpages,
%   angle=45,
%   scale=12,
%   xpos=-42,
%   ypos=39
% ]{\usebox\mybox}

\newlexicalelementwithname{arithmeticexpression}{arithmetic-expression}
\newlexicalelementwithname{conditionname}{condition-name}
\newlexicalelementwithname{filename}{file-name}
\newlexicalelementwithname{functionname}{function-name}
\newlexicalelementwithname{imperativestatement}{imperative-statement}
\newlexicalelementwithname{cobolindexname}{index-name}
\newlexicalelementwithname{libraryname}{library-name}
\newlexicalelementwithname{mnemonicname}{mnemonic-name}
\newlexicalelementwithname{procedurename}{procedure-name}
\newlexicalelementwithname{pseudotext}{pseudo-text}
\newlexicalelementwithname{recordname}{record-name}
\newlexicalelementwithname{reportname}{report-name}
\newlexicalelementwithname{sourcetext}{source-text}
\newlexicalelementwithname{textname}{text-name}
\newlexicalelement{argument}
\newlexicalelement{condition}
\newlexicalelement{expression}
\newlexicalelement{identifier}
\newlexicalelement{integer}
\newlexicalelement{literal}
\newlexicalelementwithname{switchstatusname}{switch-status-name}
\newlexicalelementwithname{switchname}{switch-name}
\newlexicalelementwithname{systemname}{system-name}
\newlexicalelementwithname{computername}{computer-name}

\frontmatter

\begin{titlepage}
  \pagecolor{Maroon!75}\afterpage{\nopagecolor}

  \centering

  \vfill

  \begin{streak}[bottom=7.5pt]
    {\centering\fontsize{32pt}{0cm}\bfseries
      The GnuCOBOL 2.0 Grammar \par
    }
    \vspace{1em}
    {\centering\scshape\LARGE for r1141 \par}
  \end{streak}

  \vspace{3cm}
  \begin{streak}
    {\LARGE\itshape Edward Hart\par}
    \vspace{5pt}
    {\large\href{mailto:edward.dan.hart@gmail.com}{\color{white}{edward.dan.hart@gmail.com}}}
  \end{streak}

  \vfill

  % Bottom of the page
  \begin{streak}[top=3pt,bottom=3pt]
    {\Large \today\par}
  \end{streak}
\end{titlepage}

\topskip0pt
\vspace*{\fill}

COBOL is an industry language and is not the property of any company or group of companies, or of any organisation or group of organisations.

No warranty, expressed or implied, is made by any contributor, or by the CODASYL COBOL Committee,\footnote{The CODASYL COBOL committee was dissolved in 1992. Its work was continued by ANSI X3J4 and then INCITS PL22.4, which was itself dissolved in 2015.} as to the accuracy and functioning of the programming system and language. Moreover, no responsibility is assumed by any contributor, or by the committee, in connection therewith.

The authors and copyright holders of the copyrighted materials used herein are:
\begin{itemize}
\item FLOW-MATIC (trademark of Sperry Rand Corporation), Programming for the UNIVAC \textregistered{} I and II, Data Automation Systems, copyrighted 1958, 1959 by Sperry Rand Corporation,\footnote{Sperry Rand's computer business is now part of Unisys.}
\item IBM Commercial Translator, Form No. F28-8013, copyrighted 1959 by IBM, and
\item FACT, DSI 27A5260-2760, copyrighted 1960 by Minneapolis-Honeywell.
\end{itemize}

They have specially authorised the use of this material, in whole or in part, in the COBOL specifications. Such authorisation extends to the reproduction and use of COBOL specifications in programming manuals or similar publications.

\vfill

\begin{center}
  This work is typeset in Roboto.
\end{center}

\vfill

\begin{center}
  Copyright \textcopyright{} \the\year{} Edward Hart

  \vspace{5pt}

  Permission is granted to copy, distribute and\slash{}or modify this document under the terms of the GNU Free Documentation License, Version 1.3 or any later version published by the Free Software Foundation; with no Invariant Sections, no Front-Cover Texts, and no Back-Cover Texts. Your attention is drawn to the copy of the license in Appendix \ref{label_fdl}.
  \vspace{5pt}

  The moral rights of the author have been asserted.
\end{center}

\tableofcontents

\chapter{Foreword}

This document describes the syntax of COBOL as supported by GnuCOBOL. It is hoped it will complement Gary Cutler's 2013 \textit{GnuCOBOL Programmer's Guide} which does not document recent features added to GnuCOBOL. It is also formatted in \LaTeX, so that everything looks a bit prettier.

The syntax diagrams were transcribed from GnuCOBOL's parsers. It thus replicates some unusual syntax rules and misses some syntax rules implemented outside the parser. For example, the obsolete identification division comment paragraphs are allowed in any order and the syntax of \hyperref[file-control-entry]{a file-control entry} does not distinguish between SEQUENTIAL, INDEXED and RELATIVE organisations.

This is a draft and so has many flaws. Designed to document features the \textit{Programmer's Guide} does not, it strangely lacks a list of these new features. Important syntax rules which cannot be contained in syntax diagrams are missing. There are no definitions of fundamental objects such as conditions and identifiers. If people find this document useful, I will try to fix these shortcomings.

%%% Local Variables:
%%% mode: latex
%%% TeX-master: "grammar.tex"
%%% End:

\chapter{Changelog}

\section{GnuCOBOL 3.0-dev}

\paragraph{General}
\begin{itemize}
\item \textbf{C API (configuration)}: added cob\_set\_runtime\_option and cob\_get\_runtime\_option for specifying the file handle which printer output or trace output should go to.
\item \textbf{cobc command-line options}: added -fdump to dump data of all modules on abend of program.
\item \textbf{cobc command-line options}: disabled -fif-cutoff pending removal.
\item \textbf{cobc command-line options}: -debug no longer implies -ftrace (\bug{449}).
\item \textbf{cobc command-line options}: added -O0 for disabling optimisations (\fr{255}).
\item \textbf{cobc command-line options}: added -Wno-dialect to suppress dialect-specific warnings.
\item \textbf{cobc command-line options}: added -Wother to enable miscellaneous warnings.
\item \textbf{cobcrun command-line options}: renamed -runtime-conf to -runtime-config.
\item \textbf{Copyfiles}: fixed segfault when GnuCOBOL tried and failed to open a copyfile (\bug{458}).
\item \textbf{Copyfiles}: GnuCOBOL now checks for recursive copyfiles (\bug{467}).
\item \textbf{Debugging}: Added debug logging for compiler developers (--enable-debug-log ./configure option).
\item \textbf{DJGGP support}: added minimal support for DJGGP.
\item \textbf{Expressions}: fixed \bug{431}, where decimal constants were not initialised in INITIAL programs.
\item \textbf{IBM i support}: fixed segfault involving filepaths.
\item \textbf{Listings}: fixed the substitution of tokens in quotes in copyfiles (\bug{494}).
\item \textbf{Modules}: enabled running of modules generated with older versions of GnuCOBOL (\fr{239}).
\item \textbf{Nested programs}: fixed \bug{435}, where identifiers in the containing program where not propogated to the contained program, causing a segfault.
\item \textbf{Recursion}: fixed \bug{222}, where the return value was lost when returning from a recursive call.
\item \textbf{Reserved words}: allow users to make default reserved words aliases for other words.
\item \textbf{Signals}: added error message for SIGFPE (see \bug{434}).
\item \textbf{Solaris support}: fixed errors and warnings when compiling in Solaris 10.
\item \textbf{Tracing}: added new options for controlling tracing (see \fr{242}).
\end{itemize}

\paragraph{Configuration options}
\begin{itemize}
\item \textbf{IBM dialect}: reserved words updated to Enterprise COBOL V6.2.
\item \textbf{incorrect-conf-sec-order option}: changed to ``ok'' from ``error'' in mf and default dialects.
\item \textbf{New compiler configuration options}: binary-comp-1, display-special-fig-consts, free-redefines-position (\fr{211}), line-col-zero-default, move-figurative-space-to-numeric, move-non-numeric-lit-to-numeric-is-zero, missing-statement, perform-without-varying-by, record-delimiter, record-delim-with-fixed-recs, screen-section-rules, sequential-delimiters
\item \textbf{New runtime configuration options}: col\_just\_lrc (for enabling\slash{}disabling the LEFT\slash{}RIGHT\slash{}CENTER phrases of the report COLUMN clause), printer and display\_print\_pipe (for specifying what command should be executed before DISPLAY UPON PRINTER; similar to Micro Focus' COBPRINTER option); display\_print\_file (name of file which DISPLAY UPON PRINTER will append its output to), trace\_format, dump\_file and dump\_width.
\item \textbf{Registers}: disabling a register also removes the reserved word with the same name, if it exists (\fr{278}).
\end{itemize}

\paragraph{Compiler directives}
\begin{itemize}
\item \textbf{New directives}: ADDRSV, ADDSYN, MAKESYN, OVERRIDE and REMOVE (\fr{210}); COMP1.
\end{itemize}

\paragraph{Identification division}
\begin{itemize}
\item \textbf{OPTIONS paragraph}: added support for ARITHMETIC clause.
\end{itemize}

\paragraph{Environment division}
\begin{itemize}
\item \textbf{ALTERNATE RECORD KEY clause}: implemented split keys (SOURCE IS; \fr{23}) and sparse keys (SUPPRESS WHEN; \fr{281}).
\item \textbf{ASSIGN clause}: re-enabled use of linkage section or BASED items (\bug{421}).
\item \textbf{CLASS phrase}: added recognition of ALPHANUMERIC\slash{}NATIONAL and alphabet phrases.
\item \textbf{CURRENCY phrase}: changed to emit error when CURRENCY SIGN other than ``\$'' is entered.
\item \textbf{CURRENCY phrase}: improved error messages for invalid currency signs.
\item \textbf{EXTERN clause}: improved error messages (\bug{446}).
\item \textbf{FILE STATUS clause}: added detection of VSAM secondary status identifier (\fr{51}).
\item \textbf{OCCURS clause}: correctly implemented nested OCCURS DEPENDING tables (ODOSLIDE).
\item \textbf{PASSWORD clause}: added detection of.
\item \textbf{RECORD DELIMITER phrase}:  added support for BINARY-SEQUENTIAL and LINE-SEQUENTIAL phrases.
\item \textbf{RECORD DELIMITER phrase}: improved syntax checks (see \bug{442}).
\item \textbf{RECORD KEY clause}: implemented sparse keys (SOURCE IS; \fr{23}).
\end{itemize}

\paragraph{Data division}
\begin{itemize}
\item \textbf{ANY LENGTH clause}: fixed \bug{487}, where literals moved to ANY LENGTH items were incorrectly truncated.
\item \textbf{Screen section}: made some COBOL words context-sensitive to screen section.
\item \textbf{Screen description}: the rules on which clauses must be specified when now depend on the dialect (see \bug{382}).
\item \textbf{SYNCHRONIZED clause}: deactivated RIGHT phrase, pending correct implementation (previously SYNCHRONIZED RIGHT was the same as SYNCHRONIZED LEFT).
\item \textbf{USAGE clause}: added option to make COMP-1 mean a 16-bit signed integer (\fr{272}).
\end{itemize}

\paragraph{Procedure division}
\begin{itemize}
\item \textbf{ACCEPT statement (screen)}: fixed ACCEPT WITH UPDATE and not working (\bug{423}).
\item \textbf{ACCEPT statement (screen)}: fixed \bug{426}, where backspacing at the start of a field moved the cursor to the second-to-last character of the preceding field.
\item \textbf{ACCEPT statement (screen)}: added detection of CONTROL KEY clause.
\item \textbf{ACCEPT statement (screen)}: allow numeric-edited fields to contain spaces (see \bug{491}).
\item \textbf{ACCEPT statement (screen)}: fixed error when filling in a one character field with insert mode on (\bug{498}).
\item \textbf{ACCEPT statement (temporal)}: fixed \bug{469}, where ACCEPT FROM DAY was off by $-1$.
\item \textbf{CALL statement}: fixed ON EXCEPTION not working properly with -fstatic (\bug{462}).
\item \textbf{DISPLAY statement (printer)}: allow for redirecting DISPLAY UPON PRINTER(-1) to files based on the runtime configuration.
\item \textbf{DISPLAY statement (screen)}: fixed DISPLAY LOW-VALUE not setting position of cursor for next DISPLAY statement (\bug{423}).
\item \textbf{DISPLAY statement (screen)}: added configuration option to disable Micro Focus' special behaviour with some figurative constants (see \bug{423}).
\item \textbf{DISPLAY statement (screen)}: fixed \bug{428}, where DISPLAY ALL ``x'' WITH SIZE only displayed ``x'' once.
\item \textbf{Exceptions}: the exception status is now only reset by SET LAST EXCEPTION TO OFF.
\item \textbf{Exceptions}: fixed EC-SIZE-OVERFLOW being raised when EC-SIZE-ZERO-DIVIDE is active (\bug{223}).
\item \textbf{File I-O}: fixed \bug{457}, where file status 30 was set instead of the correct permanent error status (34, 35 or 37).
\item \textbf{File I-O}: added support for VBISAM 2.1.1.
\item \textbf{Floating-point arithmetic}: fixed incorrect SIZE ERROR exceptions (see \bug{470}).
\item \textbf{Floating-point arithmetic}: fixed \bug{478}, where calculations with the SIZE ERROR phrase have different results to calculations without it.
\item \textbf{Floating-point arithmetic}: now activate the SIZE ERROR handler when a floating-point variable is set to a non-finite value (see \bug{122}).
\item \textbf{INQUIRE statement}: added detection of.
\item \textbf{LENGTH OF phrase}: fixed error message when used on group fields (\bug{175}).
\item \textbf{PERFORM statement}: added check for non-zero item in BY phrase (\fr{268}).
\item \textbf{PERFORM statement}: allow the BY phrase to be omitted, which is the same as specifying BY 1 (\fr{158}).
\item \textbf{MODIFY statement}: added detection of.
\item \textbf{MOVE statement}: added options to interpret moving non-numeric values to numeric items as moving zero to numeric items.
\item \textbf{Reference modification}: improved out-of-bounds error message (\bug{445}).
\item \textbf{OPEN statement}: fail with file status 39 when the first record of an indexed file is larger than specified in the FD.
\item \textbf{READ statement (indexed)}: fail with file status 43 when the read record is larger than specified in the FD.
\item \textbf{Reference modification}: added warnings for invalid reference modifications involving variables.
\item \textbf{Report writer}: implemented, including features from COBOL 2002 and the IBM Report Writer.
\item \textbf{SET statement}: fixed \bug{225}, where an invalid SET statement caused an error saying ``invalid MOVE statement''.
\item \textbf{STOP statement}: fixed the NORMAL phrase causing a compilation error (\bug{433}).
\item \textbf{VALIDATE statement}: added detection of.
\end{itemize}

\paragraph{Intrinsic functions}
\begin{itemize}
\item \textbf{MOD function}: fixed EC-SIZE-ZERO-DIVIDE being raised instead of EC-ARGUMENT-FUNCTION when zero was provided as an argument.
\item \textbf{REM function}: fixed EC-SIZE-ZERO-DIVIDE being raised instead of EC-ARGUMENT-FUNCTION when zero was provided as an argument.
\item \textbf{WHEN-COMPILED function}: fixed timezone, which was missing its sign and contained nonsense when negative (\bug{436}).
\end{itemize}

\paragraph{Built-in subprograms}
\begin{itemize}
\item \textbf{C\$GETPID}: fixed wrong process ID being returned after forking (\bug{451}).
\item \textbf{C\$SLEEP}: if the requested time is too large, sleep for the maximum possible time instead of not at all.
\item \textbf{CBL\_READ\_KBD\_CHAR}: this now works (\bug{500}).
\item \textbf{SYSTEM}: added workaround of buggy system() implementation on Windows, which removes leading and trailing quotes.
\end{itemize}

\section{GnuCOBOL 2.2}

This list tracks changes made from 23 November 2013. This excludes many changes made in 2009--2013 which would be pertinent to those upgrading from a 2009 build of OpenCOBOL 1.1 found in many package repositories.

\paragraph{General}
\begin{itemize}
\item \textbf{64-bit numbers}: fixed bugs in handling of 64-bit numbers (e.g. \bug{229}).
\item \textbf{ACUCOBOL windows}: added detection of ACUCOBOL's window\slash{}message box GUI syntax.
\item \textbf{C API (data)}: added several functions for getting and setting cob\_field items.
\item \textbf{C API (files)}: added cob\_file\_external\_addr, cob\_file\_malloc and cob\_file\_free.
\item \textbf{C API (screen)}: added several functions from Micro Focus' C to COBOL API: cob\_display\_text, cob\_sys\_get\_char, cob\_get\_char, cob\_get\_text, cob\_display\_formatted\_text, cob\_sys\_get\_csr\_pos, cob\_sys\_set\_csr\_pos, cobmove, cobaddstrc, cobprintf and cobgetch (feature requests \frnoname{148} and~\frnoname{187}).
\item \textbf{C API (signals)}: added cob\_raise to send signal to signal handlers.
\item \textbf{C compiler support}: fixed errors in compilers without designated initializers.
\item \textbf{cobc command-line options}: added -O3 to enable more optimisations.
\item \textbf{cobc command-line options}: added -Wfatal-error to make the compiler abort on the first error.
\item \textbf{cobc command-line options}: added -Wpossible-overlap to warn items that \emph{may} overlap (-Woverlap only warns if items definitely overlap).
\item \textbf{cobc command-line options}: added -fmax-errors to set number of errors at which the compiler aborts.
\item \textbf{cobc command-line options}: added -fwinmain to output WinMain instead of main (\fr{194}).
\item \textbf{cobc command-line options}: added -t and -T for complete listing support (-t for 80-characters wide listings and -T for 132-characters wide) which includes cross-references (thanks to Dave Pitts).
\item \textbf{cobc command-line options}: added -vvv (like -vv but passes verbose option to the linker as well) and -\#\#\# (like -v but commands are not executed).
\item \textbf{cobc command-line options}: allow -, i.e. stdin, as a source file.
\item \textbf{COBOL-85 NIST testsuite}: tests now refer to \$COBC, \$COBCRUN and \$COBCRUN\_DIRECT environment variables instead of directly calling cobc and cobcrun, allowing the testsuite to run in conjunction with tools like valgrind.
\item \textbf{COBOL-85 NIST testsuite}: tests for obsolete feature flagging are now executed, if possible.
\item \textbf{Comments}: added ACUCOBOL comments: \$ as synonym for * in indicator area and $\vert$ as synonym for floating comment indicator *>.
\item \textbf{Communication facility}: added detection of communication facility syntax.
\item \textbf{configure}: added useful error message when help2man, bison and flex are missing when they are needed.
\item \textbf{curses}: fixed compilation errors when configured without curses (\bug{90}).
\item \textbf{Error messages}: error messages are now lowercase, in line with the GNU Coding Standards (\bug{198}).
\item \textbf{Error messages}: segfaults in the compiler now cause an error message to be displayed.
\item \textbf{Error messages}: replaced instances of ``ODO'' by the clearer ``OCCURS DEPENDING ON''.
\item \textbf{Expressions}: resolve constant expressions and optimise constant decimals at compile time.
\item \textbf{Expressions}: added support for IBM OS/VS COBOL's arithmetic.
\item \textbf{Expressions}: improved error messages for malformed expressions.
\item \textbf{Indicators}: invalid indicators no longer cause compilation to immediately terminate (\fr{126}).
\item \textbf{Information}: output compiler version used to build GnuCOBOL and any mathematical libraries used (\fr{169}).
\item \textbf{Information}: output what a reserved word is an alias for in the --list-reserved output (\fr{214}).
\item \textbf{Manpage}: added manpage generation and installation.
\item \textbf{Nested programs}: Nested programs no longer need to have END PROGRAM.
\item \textbf{National literals}: added basic support for national literals.
\item \textbf{Numeric literals}: added ACUCOBOL numeric literals: B\#\ldots\, for binary, O\#\ldots\, for octal, and X\#\ldots\ and H\#\ldots\, for hexadecimal.
\item \textbf{Literals}: fixed heap corruptions caused by uncommon literals (\bug{195}).
\item \textbf{Literals}: allow concatenation of literal and Boolean literals.
\item \textbf{Memory management}: all memory belonging to the parsers and lexers is freed upon a compiler abend.
\item \textbf{Memory management}: fixed memory leaks due to recursive CALLs.
\item \textbf{Microsoft Visual C++}: output when compiling with cl.exe is now filtered and temporary files are deleted.
\item \textbf{MinGW}: fixed use of wrong directory separator.
\item \textbf{Signals}: removed error message on SIGPIPE.
\item \textbf{Signals}: added error message for SIGBUS.
\item \textbf{Translations}: updated, with new support for German and Italian.
\item \textbf{User-defined functions}: function definitions must now end with END FUNCTION.
\item \textbf{User-defined functions}: function definitions may no longer be nested in programs (\bug{255}).
\item \textbf{Windows support}: allow linking with asm files.
\item \textbf{Windows support}: added support for DISAM in the batch file which creates distributables.
\item \textbf{Windows support}: fixed environment-setting batch files not working with Microsoft Visual Studio 2017.
\item \textbf{Windows support}: fixed 64-bit environment-setting batch files not checking the correct directories for binaries and libraries.
\end{itemize}

\paragraph{Configuration options}
\begin{itemize}
\item \textbf{Deleted compiler configuration options}: eject-statement, cobol85-reserved.
\item \textbf{New compiler configurations}: all dialects have been split into standard and strict dialects, with strict dialects maintaining source compatibility with the dialect's compiler(s).
\item \textbf{New compiler configurations}: acu for ACUCOBOL, cobol2014 for COBOL 2014, rm for RM-COBOL, xopen for X\slash{}Open.
\item \textbf{New compiler configuration options}: accept-display-extensions, accept-update, accept-auto, acu-literals, arithmetic-osvs, call-overflow, console-is-crt, constant-01, constant-78, constant-folding, define-constant-directive, hexadecimal-boolean, hexadecimal-national-literals, incorrect-conf-sec-order, intrinsic-function, listing-statements, literal-length, move-figurative-constant-to-numeric, move-figurative-quote-to-numeric, move-ibm, national-literals, no-echo-means-secure, not-exception-before-exception, numeric-boolean, numeric-literal-length, numeric-value-for-edited-item, pic-length, program-name-redefinition, program-prototypes, reference-out-of-declaratives (\fr{179}), register, renames-uncommon-levels, reserved, reserved-words, stop-identifier, system-name, title-statement, use-for-debugging, word-length (\fr{43}). % TO-DO: expand
\item \textbf{Registers}: compiler configurations can now specify all the registers to generate.
\item \textbf{Registers}: added registers not yet implemented by GnuCOBOL as reserved words.
\item \textbf{Renamed compiler configuration options}: debugging-line to debugging-mode, relaxed-syntax-check to relax-syntax-checks.
\item \textbf{Reserved words}: compiler configurations can now specify all the reserved words and context-sensitive words permitted.
\item \textbf{Reserved words}: compiler configurations can now specify whether a reserved word is an alias for another reserved word. % TO-DO: Improve
\item \textbf{Runtime configuration}: added ability to configure some libcob features at runtime.
\item \textbf{Support options}: options which specify if a feature is supported can now take a ``+'' before their argument to indicate it takes effect only if the current level of support is less strict than ``ok''.
\end{itemize}

\paragraph{Compiler directives}
\begin{itemize}
\item \textbf{\$ indicator character}: added \$ as an indicator for compiler directive lines.
\item \textbf{\directiveindicator{}IF directive}: fixed \bug{263}, where nested \directiveindicator{}IF directives were not handled correctly.
\item \textbf{New constants}: GCCOMP, GNUCOBOL.
\item \textbf{New directives}: \directiveindicator{}CALL-CONVENTION, \directiveindicator{}LISTING, \directiveindicator{}PAGE. % TO-DO: Expand
\item \textbf{New directives (detection only)}: *CBL, *CONTROL, TITLE.
\item \textbf{New \directiveindicator{}SET phrase}: SOURCEFORMAT.
\end{itemize}

\paragraph{Identification division}
\begin{itemize}
\item \textbf{Comment paragraphs}: fixed invalid parsing of quote characters inside comment paragraphs (\bug{297}).
\item \textbf{FUNCTION-ID}: added checks for redefinition of function-names.
\item \textbf{INITIAL phrase}: fixed premature deallocation of INITIAL programs (\bug{52}).
\item \textbf{OPTIONS paragraph}: added with implementation of DEFAULT ROUNDED MODE and ENTRY-CONVENTION phrases and recognition of INTERMEDIATE ROUNDING phrase.
\item \textbf{PROGRAM-ID}: added checks for redefinition of program-names.
\item \textbf{PROGRAM-ID phrases}: permit INITIAL or RECURSIVE before COMMON (\bug{244}).
\item \textbf{Program\slash{}function-names}: warn if program\slash{}function-names contain spaces.
\end{itemize}

\paragraph{Environment division}
\begin{itemize}
\item \textbf{ASSIGN clause}: missing ASSIGN clauses are now detected at compile-time.
\item \textbf{ASSIGN clause}: added PRINTER and PRINTER-1 device-names for writing to a printer.
\item \textbf{ASSIGN clause}: added CARD-PUNCH, CARD-READER, CASSETTE, INPUT, INPUT-OUTPUT, MAGNETIC-TAPE and OUTPUT device-names for line sequential devices.
\item \textbf{ASSIGN clause}: temporarily prohibit BASED and linkage items in ASSIGN USING due to \bug{421}.
\item \textbf{CALL-CONVENTION phrase}: statically calling functions with CALL-CONVENTION 74 no longer causes linker errors (\bug{316}).
\item \textbf{CURRENCY phrase}: fixed \bug{182}, where a preceding SWITCH phrase caused an incorrect duplicate CURRENCY clause error.
\item \textbf{File-control entry}: fixed \bug{71}, where referring to a global constant caused an internal error.
\item \textbf{File-control entry}: fixed \bug{331}, where using an identifier in a file record qualified with the file's name caused an error.
\item \textbf{FUNCTION phrase}: added checks for redefinition of function-(prototype-)names.
\item \textbf{FUNCTION phrase}: compiler will no longer stop when it encounters a syntax error.
\item \textbf{LOCK MODE clause}: fixed combination of LOCK MODE IS AUTOMATIC/MANUAL with LOCK ON MULTIPLE.
\item \textbf{PROGRAM phrase}: added support for program-prototype-names.
\item \textbf{SIGN clause}: improved syntax checks.
\item \textbf{SWITCH phrase}: added check for duplicate on\slash{}off clauses (\bug{136}).
\item \textbf{SWITCH phrase}: added new switch names: SWITCH-16 through to SWITCH-36 (\fr{65}), ``SWITCH 1'' to ``SWITCH 26'' (and their aliases  ``SWITCH A'' to ``SWITCH Z''), UPSI-0 to UPSI-8 (equivalent to ``SWITCH 0'' to ``SWITCH 8'') and USW-0 to USW-31 (equivalent to ``SWITCH 0'' to ``SWITCH 31'').
\end{itemize}

\paragraph{Data division}
\begin{itemize}
\item \textbf{78-level items}: strengthened syntax checks.
\item \textbf{88-level items}: strengthened syntax checks.
\item \textbf{ANY NUMERIC clause}: ANY NUMERIC items must now have PIC 9.
\item \textbf{ANY LENGTH clause}: ANY LENGTH items may no longer be BY VALUE parameters (see \bug{219}).
\item \textbf{ANY LENGTH clause}: ANY LENGTH items must now have PIC X or PIC N.
\item \textbf{BLANK clause}: fixed \bug{143}, where BLANK LINE\slash{}SCREEN did not colour line\slash{}screen.
\item \textbf{BLANK WHEN ZERO clause}: added checks that BLANK WHEN ZERO is not specified with PICTURE clauses containing S.
\item \textbf{Constant items}: expressions in VALUE clauses now permitted.
\item \textbf{Data description}: added a maximum record length.
\item \textbf{Data description}: increased maximum size of non-indexed file record to 64 MiB (maximum size of an indexed file record is 65535 bytes).
\item \textbf{ERASE clause}: fixed \bug{186}, where ERASE EOL and ERASE EOS could be specified simultaneously.
\item \textbf{FULL clause}: added warning for useless FULL clauses on numeric items (\fr{209}).
\item \textbf{HIGHLIGHT and LOWLIGHT clauses}: added checks that HIGHLIGHT and LOWLIGHT are not specified simultaneously.
\item \textbf{Local-storage section}: fixed \bug{78}, where local-storage items where initialised after file section items.
\item \textbf{LOWLIGHT clause}: implemented.
\item \textbf{OCCURS clause}: fixed internal compiler when used with SYNC (\bug{155}).
\item \textbf{OCCURS clause}: allow KEY phrase and INDEXED phrase in any order.
\item \textbf{OCCURS clause}: fixed \bug{167}, where overly large numeric literals where accepted in the OCCURS clause.
\item \textbf{OCCURS clause (depending)}: require the minimum length to be less than the maximum length (\fr{99}).
\item \textbf{OCCURS clause (depending)}: disabled nested OCCURS DEPENDING tables due to bugs.
\item \textbf{OCCURS clause (screen-section)}: require relative LINE\slash{}COLUMN clauses in OCCURS entries (\bug{83}).
\item \textbf{OCCURS clause (unbounded)}: added by Frank Swarbrick (\patch{50}).
\item \textbf{PICTURE clause}: restricted number of permitted PICTURE strings (\bug{232}).
\item \textbf{PICTURE clause}: improved checks of constant-names referenced in PICTURE strings.
\item \textbf{RENAMES items}: strengthened syntax checks.
\item \textbf{RESERVE clause}: allow the optional word AREAS.
\item \textbf{Screen description}: permit figurative constants in screen items (\bug{108}).
\item \textbf{TALLY special register}: added.
\item \textbf{USAGE clause}: added ACUCOBOL's HANDLE phrases (see \fr{77}).
\item \textbf{VALUE clause}: VALUE clauses in REDEFINES entries now cause warnings, not errors, for compatibility.
\item \textbf{Variable records}: added checks that the minimum size of a variable record is large enough to contain the record key.
\end{itemize}

\paragraph{Procedure division}
\begin{itemize}
\item \textbf{ACCEPT statement}: added ESCAPE as synonym for EXCEPTION.
\item \textbf{ACCEPT statement}: permit clauses in any order.
\item \textbf{ACCEPT statement}: allow WITH before every screen attribute clause.
\item \textbf{ACCEPT statement}: entering control-C now terminates the program.
\item \textbf{ACCEPT statement (screen)}: fixed failed ACCEPTs caused by a buffer overflow.
\item \textbf{ACCEPT statement (screen)}: enhanced support for special keys (insert, tab, delete, alt-delete, etc.).
\item \textbf{ACCEPT statement (screen)}: fixed \bug{161} where screens terminated after entering a few characters in a field.
\item \textbf{ACCEPT statement (screen)}: added DEFAULT as synonym for UPDATE.
\item \textbf{ACCEPT statement (screen)}: ERASE and BLANK clauses in screens are now ignored (\bug{192}).
\item \textbf{ACCEPT statement (screen)}: fixed \bug{160} where ACCEPT statement LINE\slash{}COLUMN clauses did not work.
\item \textbf{ACCEPT statement (screen)}: fixed segfault on ACCEPT OMITTED (\bug{300}).
\item \textbf{ACCEPT statement (screen)}: added checks that screen attributes are not specified multiple times or after conflicting attributes.
\item \textbf{ACCEPT statement (screen)}: fixed some phrases not being recognised without being preceded by WITH (\bug{402}).
\item \textbf{ACCEPT statement (screen)}: fixed the backspace and delete keys not working and the insert key not toggling between insertion and overwriting.
\item \textbf{ACCEPT statement (screen)}: cursor now changes with insertion\slash{}overwrite mode (if supported by the terminal).
\item \textbf{ACCEPT statement (screen)}: a beep is emitted on attempts to \emph{insert} data into a full field.
\item \textbf{ADD statement (corresponding)}: restricted to numeric items (\bug{235}).
\item \textbf{ADD statement (table)}: added detection of ADD TABLE.
\item \textbf{Addition of COMP-3 numbers}: fixed bug where COMP-3 addition failed.
\item \textbf{Addition of floating-point numbers}: fixed incorrect addition of floating-point numbers.
\item \textbf{CALL statement}: implemented \fr{101}, allowing more arguments to be provided.
\item \textbf{CALL statement}: fixed behaviour when calling cancelled modules.
\item \textbf{CALL statement}: added RETURNING NOTHING. % TO-DO: What does this do?
\item \textbf{CALL statement}: the generation of C function declarations for static CALLs can now be disabled.
\item \textbf{CALL statement}: added checks for static CALLs referring to C macros.
\item \textbf{CALL statement}: warn if a literal containing the program-name contains spaces.
\item \textbf{CALL statement}: added detection of NESTED phrase.
\item \textbf{CANCEL statement}: fixed crash caused by cancelling a cancelled module.
\item \textbf{Conditions}: restricted use of IS (\bug{321}).
\item \textbf{Conditions}: added warnings for always true\slash{}false conditions (including the reason why it is always true\slash{}false).
\item \textbf{DESTROY statement}: added detection of DESTROY.
\item \textbf{DISPLAY statement}: permit clauses in any order.
\item \textbf{DISPLAY statement}: allow WITH before every screen attribute clause.
\item \textbf{DISPLAY statement (screen)}: fixed bug where EC-SCREEN exceptions did not trigger ON EXCEPTION handler (\bug{243}).
\item \textbf{DISPLAY statement (screen)}: fixed bugs in DISPLAY SPACES\slash{}ALL X``02''\slash{}ALL X``07''.
\item \textbf{DISPLAY statement (screen)}: added checks that screen attributes are not specified multiple times or after conflicting attributes.
\item \textbf{DISPLAY statement (screen)}: DISPLAY OMITTED marked as unfinished; currently equivalent to DISPLAY LOW-VALUE.
\item \textbf{DISPLAY statement (screen)}: fixed some phrases not being recognised without being preceded by WITH (\bug{402}).
\item \textbf{END DECLARATIVES phrase}: fixed \bug{88}, where an erroneous unreachable code warning was emitted for code without a main procedure.
\item \textbf{ENTRY statement}: suppress incorrect unreachable code warnings.
\item \textbf{Exception handlers}: permit NOT ON EXCEPTION\slash{}END-OF-PAGE\slash{}etc. before ON EXCEPTION\slash{}END-OF-PAGE\slash{}etc.
\item \textbf{EXIT statement}: added extension RETURNING\slash{}GIVING clause for PROGRAM phrase.
\item \textbf{File I-O}: added detection of and handling for error when no disc space is available for output files.
\item \textbf{File I-O}: added RETRY and ADVANCING ON LOCK as pending features.
\item \textbf{File I-O}: fixed detection of DISAM file handler.
\item \textbf{FREE statement}: NULL addresses no longer cause an exception.
\item \textbf{GOBACK statement}: added extension RETURNING\slash{}GIVING clause.
\item \textbf{INITIALIZE statement}: fixed \bug{84}, where literals could be passed to INITIALIZE.
\item \textbf{INITIALIZE statement}: fixed \bug{287}, where reference-modified group items were not treated like elementary items.
\item \textbf{INSPECT statement}: fixed \bug{47}, where clauses were permitted in invalid orders.
\item \textbf{LENGTH OF phrase}: fixed \bug{89}, where the length of REDEFINES item where calculated incorrectly.
\item \textbf{LENGTH OF phrase}: fixed \bug{110}, where LENGTH OF was not allowed in the UNTIL phrase of a PERFORM statement or in a VALUE clause.
\item \textbf{MOVE statement}: added more checks for overlapping MOVE statements.
\item \textbf{MOVE statement}: fixed truncation of COMP numbers not conforming to the binary-truncate setting (\bug{69}).
\item \textbf{MOVE statement}: fixed \bug{344}, where trying to MOVE to a procedure-name caused a segfault.
\item \textbf{MOVE statement}: added support for IBM's character-by-character MOVE.
\item \textbf{PERFORM statement}: fixed \bug{368}, where the compiler segfaulted when there was a PERFORM statement with an empty body and DEBUGGING MODE was specified.
\item \textbf{Procedure division header}: fixed \bug{55}, where a user-defined function without parameters failed to compile.
\item \textbf{Procedure division header}: disabled the BY VALUE phrase, pending a working implementation.
\item \textbf{Procedure division header}: fixed \bug{349}, where BY VALUE pointer parameters lead to code that couldn't be compiled by older versions of Microsoft Visual C++ (patched by Mario Matos).
\item \textbf{Procedure division header}: RETURNING items must now be declared in the linkage section.
\item \textbf{Procedure division header}: added RETURNING OMITTED.
\item \textbf{Procedure division header}: added entry-convention specifiers.
\item \textbf{Procedure division header}: now mandatory in function definitions (see \bug{271}).
\item \textbf{Procedure division header}: CHAINING programs may no longer be called by other programs (\bug{354}), per the ACUCOBOL implementation.
\item \textbf{Reference modification}: fixed \bug{146}, where the length of reference-modified item in an OCCURS DEPENDING table was too long because it was assumed the OCCURS DEPENDING table was at its maximum size.
\item \textbf{READ statement}: a failed second READ of a missing OPTIONAL file now results in a file status of 46, not 23.
\item \textbf{REWRITE statement}: added REWRITE FILE (\fr{170}).
\item \textbf{Screen I-O}: added detection of situations which raise EC-SCREEN-LINE-NUMBER, EC-SCREEN-STARTING-COLUMN and EC-SCREEN-ITEM-TRUNCATED.
\item \textbf{Screen I-O}: added support for the LINE 0 and COL 0 extensions.
\item \textbf{Screen I-O}: added some ACUCOBOL synonyms (NO ECHO, OFF, REVERSED, REVERSE, etc.).
\item \textbf{Screen I-O}: added detection of ACUCOBOL's non-standard clauses like TAB, NO-ECHO, STANDARD, BACKGROUND-HIGH, BACKGROUND-LOW, BACKGROUND-STANDARD and SIZE.
\item \textbf{SEARCH statement (ALL)}: fixed \bug{314}, where SEARCH ALL with an empty OCCURS DEPENDING table did not exit as soon as possible.
\item \textbf{Segment numbers}: added syntax checks.
\item \textbf{SET statement (address)}: disallowed changing address of non-01\slash{}77-level item (\bug{366}).
\item \textbf{SET statement (attribute)}: made HIGHLIGHT ON imply LOWLIGHT OFF and vice versa.
\item \textbf{SET statement (exception)}: added.
\item \textbf{SET statement (thread)}: added detection of ACUCOBOL extension.
\item \textbf{STOP statement (identifier)}: added (see \bug{320}).
\item \textbf{STOP statement (literal)}: fixed segfault.
\item \textbf{STOP statement (thread)}: added detection of ACUCOBOL extension.
\item \textbf{STRING statement}: strengthened syntax checks (\bug{259}).
\item \textbf{SUBTRACT statement (corresponding)}: restricted to numeric items (\bug{235}).
\item \textbf{SUBTRACT statement (table)}: added detection of SUBTRACT TABLE.
\item \textbf{Tracing}: fixed \bug{216}, where a segfault occurred with a program made from modules some of which had been compiled with tracing and physical CANCEL % TO-DO: Define "physical CANCEL"
enabled and some of which hadn't.
\item \textbf{UNSTRING statement}: fixed \bug{54}, where the POINTER value was calculated incorrectly when the delimiter was longer than one character.
\item \textbf{UNSTRING statement}: allow a literal to be the subject of an UNSTRING.
\item \textbf{WRITE statement}: added WRITE FILE (\fr{170}).
\end{itemize}

\paragraph{Intrinsic functions}
\begin{itemize}
\item \textbf{New functions (ACUCOBOL)}: ABSOLUTE-VALUE (synonym for ABS).
\item \textbf{New functions (COBOL 2014)}: FORMATTED-CURRENT-DATE, FORMATTED-DATE, FORMATTED-DATETIME, FORMATTED-TIME, INTEGER-OF-FORMATTED-DATE, TEST-FORMATTED-DATETIME.
\item \textbf{ISO-8601-date-handling functions}: added extension SYSTEM-OFFSET as replacement for last optional argument.
\item \textbf{ISO-8601-date-handling functions}: added EC-IMP-UTC-UNKNOWN if a time format ending in Z is provided but the timezone cannot be found.
\item \textbf{LENGTH function}: added detection of PHYSICAL phrase.
\item \textbf{RANDOM function}: fixed non-random number generation.
\end{itemize}

\paragraph{Built-in subprograms}
\begin{itemize}
\item \textbf{CBL\_GC\_FORK}: added.
\item \textbf{CBL\_GC\_PRINTABLE}: renamed from C\$PRINTABLE.
\item \textbf{CBL\_GC\_WAITPID}: added.
\item \textbf{CBL\_SET\_CSR\_POS}: added (feature requests \frnoname{148} and~\frnoname{187}).
\item \textbf{CBL\_READ\_KBD\_CHAR}: added (feature requests \frnoname{148} and~\frnoname{187}).
\end{itemize}

%%% Local Variables:
%%% mode: latex
%%% TeX-master: "grammar.tex"
%%% End:


\mainmatter

\chapter{Key}

\begin{table}[!h]
  \centering
  \begin{tabular}[!h]{p{0.4\textwidth} p{0.5\textwidth}}
    \toprule
    Element & Notes \\ \midrule
    Braces, $\left\{\ {}\right\}$ & One element within the braces must be selected. \\
    Brackets, $\left[\ {}\right]$ & One or zero elements within the brackets must be selected. \\
    Vertical lines, $\left|\ {}\right|$ & Each element may be selected once and in any order; if within braces, at least one element must be selected. \\
    Ellipsis, \ldots & The preceding element may be repeated any number of times. \\
    OPTIONAL-RESERVED-WORD & \\
    \key{MANDATORY-RESERVED-WORD} & Mandatory reserved words in brackets are often used instead of optional reserved words to indicate an optional feature. \\
    \deleted{Deleted element} & These elements were previously in the COBOL standard but have since been deleted. Their use is strongly discouraged. \\
    \archaic{Archaic element} & These elements remain in the standard, but their use is considered poor style and is strongly discouraged. \\
    \obsolete{Obsolete element} & These elements are slated to be deleted from the standard. Their use is strongly discouraged. \\
    \xopen{X\slash{}Open extension} & \\
    \gnucobol{GnuCOBOL-only extension} & \\
    \miscext{Miscellaneous extension} & An extension which may have come from COBOL dialects by Micro Focus, IBM, AcuCorp, Ryan-McFarland, Fujitsu or Microsoft. \\
    \pending{Unimplemented element} & These elements are recognised by GnuCOBOL, but result in errors. \\ \bottomrule
  \end{tabular}
\end{table}

%%% Local Variables:
%%% mode: latex
%%% TeX-master: "grammar.tex"
%%% End:


\preto\section{\clearpageifnotfirst{section}}

\chapter{Language fundamentals}

\section{Lexical elements}

\subsection{COBOL words}

\subsection{User-defined words}

\subsection{Reserved words}

\subsection{Literals}

\subsubsection{Alphanumeric literals}

\format{standard}
\begin{syntax}
  \begin{1=}
    \textquotesingle
    \begin{0-1}
      \character
    \end{0-1}\ldots
    \textquotesingle \\
    
    \textquotedbl
    \begin{0-1}
      \character
    \end{0-1}\ldots
    \textquotedbl
  \end{1=}
\end{syntax}

\format{hexadecimal}
\begin{syntax}
  \begin{1=}
    X\textquotesingle
    \begin{0-1}
      \hexcharacter
    \end{0-1}\ldots
    \textquotesingle \\
    
    X\textquotedbl
    \begin{0-1}
      \hexcharacter
    \end{0-1}\ldots
    \textquotedbl
  \end{1=}  
\end{syntax}

\format{null-terminated}
\begin{syntax}[\gnucobolcolour]
  \begin{1=}
    Z\textquotesingle
    \begin{0-1}
      \character
    \end{0-1}\ldots
    \textquotesingle \\
    
    Z\textquotedbl
    \begin{0-1}
      \character
    \end{0-1}\ldots
    \textquotedbl
  \end{1=}
\end{syntax}

\format{raw-C-string}
\begin{syntax}[\gnucobolcolour] % TO-DO: Check not ACUCOBOL extension.
  \begin{1=}
    L\textquotesingle
    \begin{0-1}
      \character
    \end{0-1}\ldots
    \textquotesingle \\
    
    L\textquotedbl
    \begin{0-1}
      \character
    \end{0-1}\ldots
    \textquotedbl
  \end{1=}
\end{syntax}

\subsubsection{Numeric literals}

\format{integer}
\begin{syntax}
  \begin{0-1}
    + \\
    -
  \end{0-1}
  \begin{1=}
    0 \\
    1 \\
    2 \\
    3 \\
    4 \\
    5 \\
    6 \\
    7 \\
    8 \\
    9
  \end{1=}\ldots
\end{syntax}

\format{fixed-point}
\begin{syntax}
  \begin{0-1}
    + \\
    -
  \end{0-1}
  \begin{0-1}
    0 \\
    1 \\
    2 \\
    3 \\
    4 \\
    5 \\
    6 \\
    7 \\
    8 \\
    9
  \end{0-1}\ldots
  \begin{1=}
    . \\
    ,
  \end{1=}
  \begin{1=}
    0 \\
    1 \\
    2 \\
    3 \\
    4 \\
    5 \\
    6 \\
    7 \\
    8 \\
    9
  \end{1=}\ldots
\end{syntax}

\format{floating-point}
\begin{syntax}
  \begin{0-1}
    + \\
    -
  \end{0-1}
  \begin{0-1}
    0 \\
    1 \\
    2 \\
    3 \\
    4 \\
    5 \\
    6 \\
    7 \\
    8 \\
    9
  \end{0-1}\ldots
  \begin{1=}
    . \\
    ,
  \end{1=}
  \begin{1=}
    0 \\
    1 \\
    2 \\
    3 \\
    4 \\
    5 \\
    6 \\
    7 \\
    8 \\
    9
  \end{1=}\ldots
  E
  \begin{0-1}
    + \\
    -
  \end{0-1}
  \begin{1=}
    0 \\
    1 \\
    2 \\
    3 \\
    4 \\
    5 \\
    6 \\
    7 \\
    8 \\
    9    
  \end{1=}\ldots
\end{syntax}

\format{binary}
\begin{syntax}[\miscextcolour]
  B\#
  \begin{1=}
    0 \\
    1
  \end{1=}\ldots
\end{syntax}

\format{octal}
\begin{syntax}[\miscextcolour]
  O\#
  \begin{1=}
    0 \\
    1 \\
    2 \\
    3 \\
    4 \\
    5 \\
    6 \\
    7
  \end{1=}\ldots
\end{syntax}

\format{hexadecimal-number}
\begin{syntax}[\miscextcolour]
  \begin{1=}
    H\# \\
    X\#
  \end{1=}
  \begin{1=}
    0 \\
    1 \\
    2 \\
    3 \\
    4 \\
    5 \\
    6 \\
    7
  \end{1=}\ldots
\end{syntax}

\format{hexadecimal-string}
\begin{syntax}[\gnucobolcolour] % TO-DO: Check not ACUCOBOL extension.
  \begin{1=}
    H\textquotesingle
    \begin{0-1}
      \character
    \end{0-1}\ldots
    \textquotesingle \\
    
    H\textquotedbl
    \begin{0-1}
      \character
    \end{0-1}\ldots
    \textquotedbl
  \end{1=}
\end{syntax}

\subsubsection{Boolean literals}

\format{standard}
\begin{syntax}
  \begin{1=}
    B\textquotesingle
    \begin{0-1}
      \character
    \end{0-1}\ldots
    \textquotesingle \\
    
    B\textquotedbl
    \begin{0-1}
      \character
    \end{0-1}\ldots
    \textquotedbl
  \end{1=}
\end{syntax}

\format{hexadecimal}
\begin{syntax}
  \begin{1=}
    BX\textquotesingle
    \begin{0-1}
      \hexcharacter
    \end{0-1}\ldots
    \textquotesingle \\
    
    BX\textquotedbl
    \begin{0-1}
      \hexcharacter
    \end{0-1}\ldots
    \textquotedbl
  \end{1=}
\end{syntax}

\subsubsection{National literals}

\format{standard}
\begin{syntax}
  \begin{1=}
    N\textquotesingle
    \begin{0-1}
      \character
    \end{0-1}\ldots
    \textquotesingle \\
    
    N\textquotedbl
    \begin{0-1}
      \character
    \end{0-1}\ldots
    \textquotedbl
  \end{1=}
\end{syntax}

\format{hexadecimal}
\begin{syntax}
  \begin{1=}
    NX\textquotesingle
    \begin{0-1}
      \hexcharacter
    \end{0-1}\ldots
    \textquotesingle \\
    
    NX\textquotedbl
    \begin{0-1}
      \hexcharacter
    \end{0-1}\ldots
    \textquotedbl
  \end{1=}  
\end{syntax}

\subsubsection{Figurative constants}

\format{zero}
\begin{syntax}
  ALL
  \begin{1=}
    \key{ZERO} \\
    \key{ZEROES} \\
    \key{ZEROS}
  \end{1=}
\end{syntax}

\format{space}
\begin{syntax}
  ALL
  \begin{1=}
    \key{SPACE} \\
    \key{SPACES}
  \end{1=}
\end{syntax}

\format{high-value}
\begin{syntax}
  ALL
  \begin{1=}
    \key{HIGH-VALUE} \\
    \key{HIGH-VALUES}
  \end{1=}
\end{syntax}

\format{low-value}
\begin{syntax}
  ALL
  \begin{1=}
    \key{LOW-VALUE} \\
    \key{LOW-VALUES}
  \end{1=}
\end{syntax}

\format{quote}
\begin{syntax}
  ALL
  \begin{1=}
    \key{QUOTE} \\
    \key{QUOTES}
  \end{1=}
\end{syntax}

\format{null}
\begin{syntax}[\miscextcolour]
  ALL
  \begin{1=}
    \key{NULL} \\
    \key{NULLS}
  \end{1=}
\end{syntax}

\format{literal}
\begin{syntax}
  \key{ALL} \literal
\end{syntax}

\format{symbolic-character}
\begin{syntax}
  ALL \symboliccharacter
\end{syntax}

\section{References}

\section{Expressions}

\subsection{Arithmetic expressions}

Arithmetic expressions may contain the following operators:

\begin{table}[!h]
  \begin{tabular}[!h]{l l l}
    \toprule
    \textbf{Binary operators} & \textbf{Purpose} & \textbf{Precedence} \\
    + & addition & 1 \\
    -- & subtraction & 1\\
    * & multiplication & 2\\
    / & division & 2 \\
    ** & exponentiation & 3 \\
    \gnucobol{\^{}} & \gnucobol{exponentiation} & 3 \\ \midrule
    \textbf{Unary operators} \\
    + & no effect & 4 \\
    -- & multiplication by $-1$ & 4 \\ \bottomrule
  \end{tabular}
\end{table}

Binary operators must have a numeric item or expression to both their left and right. Unary operators must have a numeric item or expression to their right only.

Operators with greatest precedence are evaluated first. If an expression contains multiple operators of equal precedence, they are evaluated from left to right.

Arithmetic expressions may contain arithmetic expressions surrounded by parentheses. These nested expressions are evaluated first, before any of the operators of the outer expression.

\begin{table}[!h]
  \centering
  \begin{tabular}[!h]{c c c c c c}
    \toprule
    \multirow{2}{*}{\textbf{First symbol}} & \multicolumn{5}{c}{\textbf{Second symbol}} \\
    \cmidrule(lr){2-6}
                          & Identifier or literal & Binary operator & Unary operator & (     & ) \\ \midrule
    Identifier or literal &                       & \tick           &                &       & \tick \\
    Binary operator       & \tick                 &                 & \tick          & \tick & \\
    Unary operator        & \tick                 &                 &                & \tick & \\
    (                     & \tick                 &                 & \tick          & \tick & \\
    )                     &                       & \tick           &                &       & \tick \\
    \bottomrule
  \end{tabular}
\end{table}

\subsection{Concatenation expressions}

\begin{syntax}
  \begin{1=}
    \literal \\
    \metaelement{concatenation-expression-1}
  \end{1=}
  \& \literal
\end{syntax}

\subsection{Conditional expressions}

\begin{table}[!h]
  \begin{tabular}[!h]{l l l}
    \toprule
    \textbf{Binary operators} & \textbf{Purpose} & \textbf{Precedence} \\
    AND & logical and & 1 \\
    OR & logical or & 2 \\ \midrule
    \textbf{Unary operator} \\
    NOT & logical not & 3 \\ \bottomrule
  \end{tabular}
\end{table}

%%% Local Variables:
%%% mode: latex
%%% TeX-master: "grammar.tex"
%%% End:


\preto\subsection{\clearpageifnotfirst{subsection}}

\chapter{Compiler directives}

\section{D directive}

\gnucobol{\directiveindicator\key{D} \sourcetext}

\section{COPY statement}

\begin{1=}
  \key{COPY} \\
  \deleted{\key{INCLUDE}}
\end{1=}
\begin{1=}
  \literal \\
  \textname \\
\end{1=}
\begin{0-1}
  \begin{1=}
    \key{IN} \\
    \key{OF}
  \end{1=}
  \begin{1=}
    \literal \\
    \libraryname
  \end{1=}
\end{0-1}

\begin{0-1}
  \key{SUPPRESS} PRINTING
\end{0-1}

\begin{0-1}
  \key{REPLACING}
  \begin{1=}
    \begin{1=}
      == \pseudotext == \\
      \identifier \\
      \literal
    \end{1=}
    \key{BY}
    \begin{1=}
      == \pseudotext == \\
      \identifier \\
      \literal
    \end{1=} \\

    \begin{1=}
      \key{LEADING} \\
      \key{TRAILING}
    \end{1=}
    == partial-word-1 ==
    \key{BY}
    == partial-word-2 ==
  \end{1=}\ldots
\end{0-1}
.

\section{DEFINE directive}

\begin{1=}
  \directiveindicator \\
  \miscext{\textdollar}
\end{1=}
\key{DEFINE}
\gnucobol{
  \begin{0-1}
    \key{CONSTANT}
  \end{0-1}
}
compilation-variable-1 AS
\begin{1=}
  \begin{1=}
      \literal \\
      \key{PARAMETER}
  \end{1=}
  \begin{0-1}
    \key{OVERRIDE}
  \end{0-1} \\
  \key{OFF}
\end{1=}

\section{DISPLAY directive}

\miscext{
  \begin{1=}
    \directiveindicator \\
    \textdollar
  \end{1=}
  \key{DISPLAY} \sourcetext
}

\section{IF directive}

\begin{1=}
  \directiveindicator \\
  \miscext{\textdollar}
\end{1=}
\key{IF} compilation-variable-1 IS NOT
\begin{1=}
  \key{DEFINED} \\
  \key{SET} \\
  relation compilation-variable-2
\end{1=}

\sourcetext


\miscext{
  \begin{0-1}
    \begin{1=}
      \gnucobol{\directiveindicator} \\
      \textdollar
    \end{1=}
    \begin{1=}
      \key{ELIF} \\
      \key{ELSE-IF}
    \end{1=}
    \condition
    \sourcetext
  \end{0-1} \ldots
}

\begin{0-1}
  \begin{1=}
    \directiveindicator \\
    \miscext{\textdollar}
  \end{1=}
  \key{ELSE} \sourcetext
\end{0-1}

\begin{0-1}
  \directiveindicator\key{END-IF} \\
  \miscext{\textdollar\key{END}}
\end{0-1}


\section{LEAP-SECOND directive}

\pending{\directiveindicator\key{LEAP-SECOND}}

\section{LISTING directive}

\directiveindicator\key{LISTING}
\begin{1=}
  \key{ON} \\
  \key{OFF}
\end{1=}

\section{PAGE directive}

\directiveindicator\key{PAGE}
\begin{0-1}
  comment-text
\end{0-1}

\section{REPLACE statement}

\format{on}
\key{REPLACE}
\begin{0-1}
  \key{ALSO}
\end{0-1}
\begin{1=}
  \begin{1=}
    == \pseudotext == \\
    \identifier
  \end{1=}
  \key{BY}
  \begin{1=}
    == \pseudotext == \\
    \identifier
  \end{1=} \\

  \begin{1=}
    \key{LEADING} \\
    \key{TRAILING}
  \end{1=}
  == partial-word-1 ==
  \key{BY}
  == partial-word-2 ==
\end{1=}\ldots .

\format{off}
\key{REPLACE}
\begin{0-1}
  \key{LAST}
\end{0-1}
\key{OFF}.

\section{SET directive}

\miscext{
  \begin{1=}
    \gnucobol{\directiveindicator} \\
    \textdollar
  \end{1=}
  \key{SET}
  \begin{1=}
    \gnucobol{
      \key{CONSTANT} compilation-variable-1 AS \literal
    } \\

    \gnucobol{
      compilation-variable-2
      \begin{0-1}
        AS \literal
      \end{0-1}
    } \\

    \key{SOURCEFORMAT} AS \literal \\

    \begin{1=}
      \key{NO-FOLD-COPY-NAME} \\
      \key{NOFOLDCOPYNAME}
    \end{1=} \\

    \begin{1=}
    \key{FOLD-COPY-NAME} \\
    \key{FOLDCOPYNAME}
  \end{1=}
  AS \literal
  \end{1=} \ldots
}

\section{SOURCE directive}

\directiveindicator\key{SOURCE} FORMAT IS
\begin{1=}
  \key{FIXED} \\
  \key{FREE}
\end{1=}

\section{TURN directive}

\pending{
  \directiveindicator\key{TURN}
  \begin{1=}
    exception-name-1
  \end{1=} \ldots
  \begin{0-1}
    \key{ON} \\
    \key{OFF}
  \end{0-1}
  \begin{0-1}
    WITH \key{LOCATION}
  \end{0-1}
}

\section{Miscellaneous directives}

\begin{itemize}
\item \miscext{EJECT}
\item \miscext{\pending{PROCESS}}
\item \miscext{SKIP1}
\item \miscext{SKIP2}
\item \miscext{SKIP3}
\end{itemize}

%%% Local Variables:
%%% mode: latex
%%% TeX-master: "grammar.tex"
%%% End:

\chapter{Compilation group}

\begin{0-1}
  program-definition \\
  function-definition
\end{0-1} \ldots

where program-definition is\vspace{1em}\newline
\begin{0-1}
  \begin{1=}
    \key{IDENTIFICATION} \\
    \miscext{\key{ID}}
  \end{1=}
  \key{DIVISION}.
\end{0-1} \newline
\key{PROGRAM-ID}.
\begin{1=}
  program-name-1 \\
  \miscext{\literal}
\end{1=}
\begin{0-1} \key{AS} \literal \end{0-1}
\begin{0-1} IS
  \begin{1=}
    \begin{1+}
      \key{COMMON} \\

      \begin{1=}
        \key{INITIAL} \\
        \key{RECURSIVE}
      \end{1=}
    \end{1+} \\

    \miscext{\pending{\key{EXTERNAL}}}
  \end{1=}
  PROGRAM
\end{0-1}. \newline
\deleted{
  \begin{0-1}
    comment-paragraphs
  \end{0-1}
} \newline
\begin{0-1}
  environment-division
\end{0-1} \newline
\begin{0-1}
  data-division
\end{0-1} \newline
\begin{0-1}
  procedure-division
  \begin{0-1}
    program-definition
  \end{0-1} \ldots
\end{0-1} \newline
\begin{0-1}
  \key{END} \key{PROGRAM}
  \begin{1=}
    program-name-1 \\
    \miscext{literal-1}
  \end{1=}.
\end{0-1}

where function-definition is\vspace{1em}\newline
\begin{0-1}
  \begin{1=}
    \key{IDENTIFICATION} \\
    \miscext{\key{ID}}
  \end{1=}
  \key{DIVISION}.
\end{0-1} \newline
\key{FUNCTION-ID}.
\begin{1=}
  program-name-1 \\
  \miscext{\literal}
\end{1=}
\begin{0-1} \key{AS} \literal \end{0-1}.\newline
\deleted{
  \begin{0-1}
    comment-paragraphs
  \end{0-1}
} \newline
\begin{0-1}
  environment-division
\end{0-1} \newline
\begin{0-1}
  data-division
\end{0-1} \newline
\begin{0-1}
  procedure-division
\end{0-1} \newline
\key{END} \key{FUNCTION}
\begin{1=}
  program-name-1 \\
  \miscext{literal-1}
\end{1=}.

%%% Local Variables:
%%% mode: latex
%%% TeX-master: "grammar.tex"
%%% End:
\chapter{Identification division}

\newcommand{\commenttext}{\metaelement{comment-text}}

\begin{syntax}
  \begin{0-1}
    \begin{1=}
      \key{IDENTIFICATION} \\
      \miscext{\key{ID}}
    \end{1=}
    DIVISION.
  \end{0-1}
  \newline
  \begin{1=}
    \metaelement{function-id-paragraph} \\
    \metaelement{program-id-paragraph}
  \end{1=}
  \newline
  \begin{0-1}
    \metaelement{options-paragraph} % TO-DO: Can this be mixed up with the comment paragraphs?
  \end{0-1}
  \newline
  \deleted{
    \begin{0+}
      \key{AUTHOR}. \commenttext. \\
      \key{DATE-WRITTEN}. \commenttext. \\
      \key{DATE-MODIFIED}. \commenttext. \\
      \key{DATE-COMPILED}. \commenttext. \\
      \key{INSTALLATION}. \commenttext. \\
      \key{REMARKS}. \commenttext. \\
      \key{SECURITY}. \commenttext. \\
    \end{0+} \gnucobol{\ldots}
  }
\end{syntax}

\subsubsection{Syntax rules}

\subsubsection{General rules}

\section{PROGRAM-ID paragraph}

\begin{syntax}
  \key{PROGRAM-ID}.
  \begin{1=}
    \metaelement{program-name-1} \\
    \literal
  \end{1=}
  \begin{0-1} \key{AS} \literal \end{0-1}

  \begin{0-1} IS
    \begin{1=}
      \begin{1+}
        \key{COMMON} \\

        \begin{1=}
          \key{INITIAL} \\
          \key{RECURSIVE}
        \end{1=}
      \end{1+} \\

      \miscext{\pending{\key{EXTERNAL}}}
    \end{1=}
    PROGRAM
  \end{0-1}.
\end{syntax}

\subsubsection{Syntax rules}

\subsubsection{General rules}

\section{FUNCTION-ID paragraph}

\begin{syntax}
  \key{FUNCTION-ID}.
  \begin{1=}
    \functionname \\
    \literal
  \end{1=}
  \begin{0-1} \key{AS} \literal \end{0-1}.
\end{syntax}

\subsubsection{Syntax rules}

\subsubsection{General rules}

\section{OPTIONS paragraph}

\begin{syntax}
  \key{OPTIONS}.

  \begin{0-1}
    \key{DEFAULT} \key{ROUNDED} MODE IS
    \begin{1=}
      \key{AWAY-FROM-ZERO} \\
      \key{NEAREST-AWAY-FROM-ZERO} \\
      \key{NEAREST-EVEN} \\
      \key{NEAREST-TOWARD-ZERO} \\
      \key{PROHIBITED} \\
      \key{TOWARD-GREATER} \\
      \key{TOWARD-LESSER} \\
      \key{TRUNCATION}
    \end{1=}
  \end{0-1}

  \begin{0-1}
    \key{ENTRY-CONVENTION} IS
    \begin{1=}
      \key{COBOL} \\
      \key{EXTERN} \\
      \key{STDCALL}
    \end{1=}
  \end{0-1}

  \pending{
    \begin{0-1}
      \key{INTERMEDIATE} \key{ROUNDING} IS
      \begin{1=}
        \key{NEAREST-AWAY-FROM-ZERO} \\
        \key{NEAREST-EVEN} \\
        \key{PROHIBITED} \\
        \key{TRUNCATION}
      \end{1=}
    \end{0-1}
  }.
\end{syntax}

\subsubsection{Syntax rules}

\subsubsection{General rules}


%%% Local Variables:
%%% mode: latex
%%% TeX-master: "grammar.tex"
%%% End:

\chapter{Environment division}
\begin{syntax}
  \begin{0-1}
    \key{ENVIRONMENT} \key{DIVISION}.
  \end{0-1}
  \newline
  \begin{0-1}
    \metaelement{configuration-section}
  \end{0-1}
  \newline
  \begin{0-1}
    \metaelement{input-output-section}
  \end{0-1}
\end{syntax}

\subsubsection{Syntax rules}

\subsubsection{General rules}

\section{Configuration section}

\format{standard}

\begin{syntax}
  \key{CONFIGURATION} \key{SECTION}.
  \newline
  \begin{0-1}
    \metaelement{source-computer-paragraph}
  \end{0-1}
  \newline
  \begin{0-1}
    \metaelement{object-computer-paragraph}
  \end{0-1}
  \newline
  \begin{0-1}
    \metaelement{special-names-header}
    \begin{0-1}
      \metaelement{special-names-entry}
    \end{0-1} \ldots
  \end{0-1}
  \newline
  \begin{0-1}
    \metaelement{repository-paragraph}
  \end{0-1}
\end{syntax}

\format{micro-focus-and-gnucobol}

\begin{syntax}[\miscextcolour]
  \begin{0-1}
    \key{CONFIGURATION} \key{SECTION}.
  \end{0-1}
  \newline
  \begin{0+}
    \metaelement{source-computer-paragraph} \\
    \metaelement{object-computer-paragraph} \\
    \metaelement{special-names-header} \\
    \metaelement{special-names-entry} \\
    \metaelement{repository-paragraph}
  \end{0+}
\end{syntax}

\subsubsection{Syntax rules}

\subsubsection{General rules}

\subsection{SOURCE-COMPUTER paragraph}

The SOURCE-COMPUTER paragraph identifies the computer on which the compilation unit should be compiled.

\begin{syntax}
  \key{SOURCE-COMPUTER}.
  \begin{0-1}
    \begin{1=}
      \computername
    \end{1=}\gnucobol{\ldots}
    \deleted{
      \begin{0-1}
        WITH \key{DEBUGGING} \key{MODE}
      \end{0-1}
    }
    .
  \end{0-1}
\end{syntax}

\subsubsection{Syntax rules}

\subsubsection{General rules}

\subsection{OBJECT-COMPUTER paragraph}

The OBJECT-COMPUTER paragraph identifies the computer on which the runtime module should be run.

\begin{syntax}
  \key{OBJECT-COMPUTER}.

  \begin{0-1}
    \begin{0-1}
      \begin{1=}
        \computername
      \end{1=}\gnucobol{\ldots}
    \end{0-1}\\\quad
    \begin{0-1}
      \deleted{
        \key{MEMORY} SIZE IS \integer
        \begin{1=}
          \key{CHARACTERS} \\
          \key{WORDS} \\
          \key{MODULES}
        \end{1=}
      } \\

      PROGRAM COLLATING \key{SEQUENCE} IS \defdmetaelement{collating-sequences}\\

      \deleted{\key{SEGMENT-LIMIT} IS \integer} \\
      CHARACTER \key{CLASSIFICATION} IS
      \begin{1=}
        \metaelement{locale-name-1} \\
        \key{LOCALE} \\
        \key{SYSTEM-DEFAULT} \\
        \key{USER-DEFAULT}
      \end{1=}
    \end{0-1}\ldots\ {}.
  \end{0-1}
\end{syntax}

where \defnmetaelement{collating-sequences} is:

\begin{syntax}
  \begin{1=}
    IS \metaelement{alphanumeric-collating-sequence}
    \pending{
      \begin{0-1}
        \metaelement{national-collating-sequence}
      \end{0-1}
    } \\

    \begin{1+}
      FOR \key{ALPHANUMERIC} IS \metaelement{alphanumeric-collating-sequence} \\
      \pending{FOR \key{NATIONAL} IS \metaelement{national-collating-sequence}}
    \end{1+}
  \end{1=} \\
\end{syntax}

\subsubsection{Syntax rules}

\subsubsection{General rules}

\subsection{SPECIAL-NAMES paragraph}

\begin{syntax}
  \begin{0-1}
    \key{SPECIAL-NAMES}.
  \end{0-1}

  \begin{0-1}
    \begin{1=}
      \defdmetaelement{mnemonic-name-clause} \\
      \defdmetaelement{alphabet-name-clause} \\
      \defdmetaelement{symbolic-characters-clause} \\
      \miscext{\defdmetaelement{symbolic-constant-class}} \\
      \key{LOCALE} \metaelement{locale-name-1} IS \literal \\
      \defdmetaelement{class-clause} \\

      \key{CURRENCY} SIGN IS \literal
      \begin{0-1}
        \pending{WITH \key{PICTURE-SYMBOL} \literal}
      \end{0-1} \\

      \key{DECIMAL-POINT} IS \key{COMMA} \\
      \miscext{\key{NUMERIC} \key{SIGN} IS \key{TRAILING} \key{SEPARATE}} \\
      \key{CURSOR} IS \identifier \\
      \key{CRT} \key{STATUS} IS \identifier \\
      \miscext{\pending{\key{SCREEN-CONTROL} IS \identifier}} \\
      \miscext{\pending{\key{EVENT-STATUS} IS \identifier}}
    \end{1=}\ldots\ {}.
  \end{0-1}\ldots
\end{syntax}

where \defnmetaelement{mnemonic-name-clause} is

\begin{syntax}
  \mnemonicname
  \begin{1=}
    IS \key{CRT} \\
    \integer IS \systemname \\
    \begin{0-1}
      IS \switchname
    \end{0-1}
    \begin{1+}
      \key{ON} STATUS IS \switchstatusname \\
      \key{OFF} STATUS IS \switchstatusname
    \end{1+}
  \end{1=}
\end{syntax}

where \defnmetaelement{alphabet-name-clause} is

\begin{syntax}
  \key{ALPHABET} \metaelement{alphabet-name-1} IS
  \begin{1=}
    \key{ASCII} \\
    \key{EBCDIC} \\
    \key{NATIVE} \\
    \key{STANDARD-1} \\
    \key{STANDARD-2} \\
    \begin{1=}
      \literal
      \begin{0-1}
        \begin{1=}
          \key{THROUGH} \\
          \key{THRU}
        \end{1=}
        \literal \\
        \begin{1=}
          \key{ALSO} \literal
        \end{1=}\ldots
      \end{0-1}
    \end{1=}\ldots
  \end{1=}
\end{syntax}

where \defnmetaelement{symbolic-characters-clause} is

\begin{syntax}
  \key{SYMBOLIC} CHARACTERS
  \begin{1=}
    \begin{1=}
      \metaelement{symbolic-character-name-1}
    \end{1=}\ldots
    \begin{1=}
      \key{IS} \\
      \key{ARE}
    \end{1=}
    \begin{1=}
      \integer
    \end{1=}\ldots
  \end{1=}\ldots
  \begin{0-1}
    \key{IN} \key{WORD}
  \end{0-1}
\end{syntax}

where \defnmetaelement{symbolic-constant-clause} is

\begin{syntax}[\miscextcolour]
  \key{SYMBOLIC} \key{CONSTANT}
  \begin{1=}
    \identifier IS \literal
  \end{1=}\ldots
\end{syntax}

where \defnmetaelement{class-clause} is

\begin{syntax}
  \key{CLASS}
  \begin{0-1}
    FOR \key{ALPHANUMERIC} \\
    \pending{FOR \key{NATIONAL}} \\
  \end{0-1}
  \metaelement{class-name-1} IS
  \begin{1=}
    \literal
    \begin{0-1}
      \begin{1=}
        \key{THRU} \\
        \key{THROUGH}
      \end{1=}
      \literal
    \end{0-1}
  \end{1=}\ldots

  \pending{
    \begin{0-1}
      \key{IN} \metaelement{alphabet-name-1}
    \end{0-1}
  }
\end{syntax}

\subsubsection{Syntax rules}

\subsubsection{General rules}

\subsection{REPOSITORY paragraph}

\begin{syntax}
  \key{REPOSITORY}.

  \begin{0-1}
    \begin{1=}
      \key{FUNCTION}
      \begin{1=}
        \begin{1=}
          \functionname
        \end{1=}\ldots \\

        \key{ALL}
      \end{1=}
      \key{INTRINSIC} \\

      \key{FUNCTION} \functionname
      \begin{0-1}
        \key{AS} \literal
      \end{0-1} \\

      \key{PROGRAM} \programname
      \begin{0-1}
        \key{AS} \literal
      \end{0-1}
    \end{1=}\ldots\ {}.
  \end{0-1}
\end{syntax}

\subsubsection{Syntax rules}

\subsubsection{General rules}

\section{Input-output section}

\begin{syntax}
  \begin{0-1}
    \key{INPUT-OUTPUT} \key{SECTION}.
  \end{0-1}\newline
  \begin{0-1}
    \metaelement{file-control-paragraph}
  \end{0-1}\newline
  \begin{0-1}
    \metaelement{i-o-control-paragraph}
  \end{0-1}
\end{syntax}

\subsubsection{Syntax rules}

\subsubsection{General rules}

\subsection{FILE-CONTROL paragraph}

\begin{syntax}
  \begin{0-1}
    \key{FILE-CONTROL}.
  \end{0-1}\newline
  \begin{0-1}
    \defdmetaelement{file-control-entry}
  \end{0-1} \ldots
\end{syntax}

where \defnmetaelement{file-control-entry} is

\begin{syntax}
  \key{SELECT}
  \begin{0-1}
    \key{OPTIONAL} \\
    \miscext{\key{NOT} \key{OPTIONAL}}
  \end{0-1}
  \filename
  \begin{0-1}
    \defdmetaelement{assign-clause} \\
    \defdmetaelement{access-mode-clause} \\
    \defdmetaelement{alternate-record-key-clause} \\
    \defdmetaelement{collating-sequence-clause} \\
    \defdmetaelement{file-status-clause} \\
    \defdmetaelement{lock-mode-clause} \\
    \defdmetaelement{organization-clause} \\
    \defdmetaelement{padding-character-clause} \\
    \defdmetaelement{password-clause} \\
    \defdmetaelement{record-delimiter-clause} \\
    \defdmetaelement{record-key-clause} \\
    \defdmetaelement{relative-key-clause} \\
    \defdmetaelement{reserve-clause} \\
    \defdmetaelement{sharing-clause}
  \end{0-1}\ldots\ {}.
\end{syntax}

where \defnmetaelement{assign-clause} is

\ {}\newline
\begin{syntax}
  \key{ASSIGN}
  \begin{0-1}
    \key{TO} \\
    \key{USING}
  \end{0-1}
  \miscext{
    \begin{0-1}
      \key{DYNAMIC} \\
      \key{EXTERNAL}
    \end{0-1}
  }
  \begin{1=}
    \miscext{
      \begin{0-1}
        \key{LINE} \key{ADVANCING} FILE
      \end{0-1}
    }
    \begin{1=}
      \literal \\
      \identifier
    \end{1=}\\

    \begin{1=}
      \miscext{\key{CARD-PUNCH}} \\
      \miscext{\key{CARD-READER}} \\
      \miscext{\key{CASSETTE}} \\
      \key{DISC} \\
      \xopen{\key{DISK}} \\
      \miscext{\key{DISPLAY}} \\
      \miscext{\key{INPUT}} \\
      \miscext{\key{INPUT-OUTPUT}} \\
      \miscext{\key{KEYBOARD}} \\
      \miscext{\key{MAGNETIC-TAPE}} \\
      \miscext{\key{OUTPUT}} \\
      \miscext{\key{PRINTER-1}} \\
      \xopen{\key{PRINTER}} \\
      \key{PRINT} \\
      \key{RANDOM} \\
      \key{TAPE}
    \end{1=}
    \begin{0-1}
      \literal \\
      \identifier
    \end{0-1}
  \end{1=}
\end{syntax}

where \defnmetaelement{access-mode-clause} is

\begin{syntax}
  \key{ACCESS} MODE IS
  \begin{1=}
    \key{SEQUENTIAL} \\
    \key{DYNAMIC} \\
    \key{RANDOM}
  \end{1=}
\end{syntax}

where \defnmetaelement{alternate-record-key-clause} is

\begin{syntax}
  \key{ALTERNATE} RECORD KEY IS \identifier
  \begin{0-1}
    \begin{1=}
      = \\
      \key{SOURCE} IS
    \end{1=}
    \begin{1=}
      \identifier
    \end{1=}\ldots
  \end{0-1}
  \begin{0-1}
    WITH
    \begin{0-1}
      \key{NO}
    \end{0-1}
    \key{DUPLICATES}
  \end{0-1}
  \miscext{
    \pending{
      \begin{0-1}
        \key{PASSWORD} IS \identifier
      \end{0-1}
    }
  }
  \begin{0-1}
    \key{SUPPRESS} \key{WHEN}
    \begin{1=}
      \key{ALL} \literal \\
      \key{SPACE} \\
      \key{ZERO}
    \end{1=}
  \end{0-1}
\end{syntax}

where \defnmetaelement{collating-sequence-clause} is

\begin{syntax}
  \pending{COLLATING \key{SEQUENCE} IS \metaelement{collating-sequence-name-1}}
\end{syntax}

where \defnmetaelement{file-status-clause} is

\begin{syntax}
  \begin{0-1}
    \key{FILE} \\
    \key{SORT}
  \end{0-1}
  \key{STATUS} IS \identifier
  \miscext{
    \begin{0-1}
      \identifier
    \end{0-1}
  }
\end{syntax}

where \defnmetaelement{lock-mode-clause} is

\begin{syntax}
  \key{LOCK} MODE IS
  \begin{1=}
    \begin{1=}
      \begin{1=}
        \key{MANUAL} \\
        \key{AUTOMATIC}
      \end{1=}
      \begin{0-1}
        \key{WITH} LOCK \key{ON}
        \begin{0-1}
          \key{MULTIPLE}
        \end{0-1}
        \begin{1=}
          \key{RECORD} \\
          \key{RECORDS}
        \end{1=}
      \end{0-1}
      \begin{0-1}
        \miscext{\pending{WITH \key{ROLLBACK}}}
      \end{0-1}
    \end{1=} \\
    
    \miscext{
      \key{EXCLUSIVE}
      \pending{
        \begin{0-1}
          WITH \key{MASS-UPDATE}
        \end{0-1}
      }
    }
  \end{1=}
\end{syntax}

where \defnmetaelement{organization-clause} is

\begin{syntax}
  \begin{0-1}
    \begin{1=}
      \key{ORGANIZATION} \\
      \miscext{\key{ORGANISATION}}
    \end{1=}
    IS
  \end{0-1}
  \begin{1=}
    \key{INDEXED} \\
    \xopen{\key{LINE} \key{SEQUENTIAL}} \\
    \miscext{RECORD BINARY} \key{SEQUENTIAL} \\
    \key{RELATIVE}
  \end{1=}
\end{syntax}

where \defnmetaelement{padding-character-clause} is

\begin{syntax}
  \pending{
    \key{PADDING} CHARACTER IS
    \begin{1=}
      \identifier \\
      \literal
    \end{1=}
  }
\end{syntax}

where \defnmetaelement{password-clause} is

\begin{syntax}[\miscextcolour]
  \pending{
    \key{PASSWORD} IS \identifier
  }
\end{syntax}

where \defnmetaelement{record-delimiter-clause} is

\begin{syntax}
  \key{RECORD} \key{DELIMITER} IS
  \begin{1=}
    \pending{\key{STANDARD-1}} \\
    \key{LINE-SEQUENTIAL} \\
    \key{BINARY-SEQUENTIAL}
  \end{1=}
\end{syntax}

where \defnmetaelement{record-key-clause} is

\begin{syntax}
  \key{RECORD} KEY IS \identifier
  \begin{0-1}
    \begin{1=}
      \miscext{=} \\
      \key{SOURCE} IS
    \end{1=}
    \begin{1=}
      \identifier
    \end{1=}\ldots
  \end{0-1}

  \miscext{
    \pending{
      \begin{0-1}
        \key{PASSWORD} IS \identifier
      \end{0-1}
    }
  }

  \miscext{
    \begin{0-1}
      WITH
      \begin{0-1}
        \key{NO}
      \end{0-1}
      \key{DUPLICATES}
    \end{0-1}
  }
\end{syntax}

where \defnmetaelement{relative-key-clause} is

\begin{syntax}
  \key{RELATIVE} KEY IS \identifier
\end{syntax}

where \defnmetaelement{reserve-clause} is

\begin{syntax}
  \pending{
    \key{RESERVE}
    \begin{1=}
      \key{NO} \\
      \integer
    \end{1=}
    \begin{0-1}
      AREA \\
      AREAS
    \end{0-1}
  }
\end{syntax}

where \defnmetaelement{sharing-clause} is

\begin{syntax}
  \key{SHARING} WITH
  \begin{1=}
    \key{ALL} OTHER \\
    \key{NO} OTHER \\
    \key{READ} \key{ONLY}
  \end{1=}
\end{syntax}

\subsubsection{Syntax rules}

\subsubsection{General rules}

\subsection{I-O-CONTROL paragraph}
\begin{syntax}
  \begin{0-1}
    \key{I-O-CONTROL}.
  \end{0-1}\newline
  \begin{1=}
    \key{SAME}
    \begin{0-1}
      \key{RECORD} \\
      \key{SORT} \\
      \key{SORT-MERGE}
    \end{0-1}
    AREA FOR
    \begin{1=}
      \filename
    \end{1=}\ldots \\

    \deleted{
      \key{MULTIPLE} FILE TAPE CONTAINS
      \begin{1=}
        \filename
        \begin{0-1}
          \key{POSITION} \integer
        \end{0-1}
      \end{1=}\ldots
    }
  \end{1=}\ldots\ {}.
\end{syntax}

\subsubsection{Syntax rules}

\subsubsection{General rules}

%%% Local Variables:
%%% mode: latex
%%% TeX-master: "grammar.tex"
%%% End:

\chapter{Data division}

\begin{syntax}
  \begin{0-1}
    \key{DATA} \key{DIVISION}.
  \end{0-1}\newline
  \begin{0-1}
    \metaelement{file-section}
  \end{0-1}\newline
  \begin{0-1}
    \metaelement{working-storage-section}
  \end{0-1}\newline
  \pending{
    \deleted{
      \begin{0-1}
        \metaelement{communication-section}
      \end{0-1}
    }
  }\newline
  \begin{0-1}
    \metaelement{local-storage-section}
  \end{0-1}\newline
  \begin{0-1}
    \pending{\metaelement{report-section}}
  \end{0-1}\newline
  \begin{0-1}
    \metaelement{screen-section}
  \end{0-1}
\end{syntax}

\subsubsection{Syntax rules}

\subsubsection{General rules}

\section{File section}

\begin{syntax}
  \begin{0-1}
    \key{FILE} \key{SECTION}.
  \end{0-1}\newline
  \begin{1=}
    \metaelement{file-description-entry}
    \begin{1=}
      \metaelement{record-description} \\
      \metaelement{constant-definition}
    \end{1=}\ldots
  \end{1=}\ldots
\end{syntax}

\subsubsection{Syntax rules}

\subsubsection{General rules}

\subsection{File description entry}

\begin{syntax}
  \begin{1=}
    \key{FD} \\
    \key{SD}
  \end{1=}
  \filename
  \begin{0-1}
    \metaelement{block-clause} \\

    \pending{
      \key{CODE-SET} IS \metaelement{alphabet-name-1}
      \begin{0-1}
        \key{FOR}
        \begin{1=}
          \identifier
        \end{1=} \ldots
      \end{0-1}
    } \\

    \deleted{
      \key{DATA}
      \begin{1=}
        \key{RECORD} IS \\
        \key{RECORDS} ARE
      \end{1=}
      \begin{1=}
        \identifier
      \end{1=}\ldots
    } \\

    IS \key{EXTERNAL} \\
    IS \key{GLOBAL} \\

    \deleted{
      \key{LABEL}
      \begin{1=}
        \key{RECORD} IS \\
        \key{RECORDS} ARE
      \end{1=}
      \begin{1=}
        \key{STANDARD} \\
        \key{OMITTED}
      \end{1=}
    } \\

    \metaelement{linage-clause} \\

    \deleted{
      \key{RECORDING} MODE IS
      \begin{1=}
        \begin{1=}
          \key{F} \\
          \key{FIXED} \\
        \end{1=} \\

        \begin{1=}
          \key{V} \\
          \key{VARIABLE} \\
        \end{1=} \\

        \key{U} \\
        \key{S}
      \end{1=}
    } \\

    \pending{
      \begin{1=}
        \key{REPORT} IS \\
        \key{REPORTS} ARE
      \end{1=}
      \begin{1=}
        \identifier
      \end{1=}\ldots
    } \\

    \deleted{
      \key{VALUE} \key{OF}
      \begin{1=}
        \key{FILE-ID} \\
        \key{ID} \\
        \identifier
      \end{1=}
      IS
      \begin{1=}
        \literal \\
        \identifier
      \end{1=}
    } \\

    \metaelement{record-clause}
  \end{0-1}\ldots\ {}.
\end{syntax}

\subsubsection{Syntax rules}

\subsubsection{General rules}

\section{Working-storage section}
\begin{syntax}
  \key{WORKING-STORAGE} \key{SECTION}.\newline
  \begin{0-1}
    \metaelement{constant-definition} \\
    \metaelement{record-description}
  \end{0-1}\ldots
\end{syntax}

\subsubsection{Syntax rules}

\subsubsection{General rules}

\section{Communication section}
\begin{syntax}[\deletedcolour]
  \pending{
    \key{COMMUNICATION} \key{SECTION}.\newline
    \begin{0-1}
      \metaelement{communication-description-entry}
      \begin{0-1}
        \metaelement{record-description} \\
        \metaelement{constant-definition}
      \end{0-1}\ldots
    \end{0-1}\ldots
  }
\end{syntax}

\subsection{Communication description entry}

% TO-DO: Format more neatly.
\format{input}
\begin{syntax}[\deletedcolour]
  \key{CD} \metaelement{entry-name} FOR
  \begin{0-1}
    \key{INITIAL}
  \end{0-1}
  \key{INPUT}
  \begin{0-1}
    \begin{1+}
      SYMBOLIC \key{QUEUE} IS \identifier \\
      SYMBOLIC \key{SUB-QUEUE-1} IS \identifier \\
      SYMBOLIC \key{SUB-QUEUE-2} IS \identifier \\
      SYMBOLIC \key{SUB-QUEUE-3} IS \identifier \\
      \key{MESSAGE} \key{DATE} IS \identifier \\
      \key{MESSAGE} \key{TIME} IS \identifier \\
      SYMBOLIC \key{SOURCE} IS \identifier \\
      \key{TEXT} \key{LENGTH} IS \identifier \\
      \key{END} \key{KEY} IS \identifier \\
      \key{STATUS} \key{KEY} IS \identifier \\
      MESSAGE \key{COUNT} IS \identifier
    \end{1+} \\

    \hspace{1em}
    \begin{minipage}[!h]{0.4\textwidth}
      \hspace{-2.15em} \identifier \identifier \identifier \identifier \identifier \identifier \identifier \identifier \identifier \identifier \identifier
    \end{minipage}
  \end{0-1}
  .
\end{syntax}

\format{output}
\begin{syntax}[\deletedcolour]
  \key{CD} \metaelement{entry-name} FOR \key{OUTPUT}
  \begin{0+}
    \key{DESTINATION} \key{COUNT} IS \identifier \\
    \key{TEXT} \key{LENGTH} IS \identifier \\
    \key{STATUS} \key{KEY} IS \identifier \\
    \key{DESTINATION} \key{TABLE} \key{OCCURS} \integer TIMES
    \begin{0-1}
      \key{INDEXED} BY
      \begin{1=}
        \cobolindexname
      \end{1=}\ldots
    \end{0-1} \\
    \key{ERROR} \key{KEY} IS \identifier \\
    \key{DESTINATION} IS \identifier \\
    \key{SYMBOLIC} \key{DESTINATION} IS \identifier
  \end{0+}
  .
\end{syntax}

\format{I-O}
\begin{syntax}[\deletedcolour]
  \key{CD} \metaelement{entry-name} FOR INITIAL \key{I-O}
  \begin{0-1}
    \begin{1+}
      \key{MESSAGE} \key{DATE} IS \identifier \\
      \key{MESSAGE} \key{TIME} IS \identifier \\
      SYMBOLIC \key{TERMINAL} IS \identifier \\
      \key{TEXT} \key{LENGTH} IS \identifier \\
      \key{END} \key{KEY} IS \identifier \\
      \key{STATUS} \key{KEY} IS \identifier
    \end{1+} \\

    \identifier \identifier \identifier \identifier \identifier \identifier
  \end{0-1}
  .
\end{syntax}

\section{Local-storage section}
\begin{syntax}
  \key{LOCAL-STORAGE} \key{SECTION}.\newline
  \begin{0-1}
    \metaelement{constant-definition} \\
    \metaelement{record-description}
  \end{0-1}\ldots
\end{syntax}

\subsubsection{Syntax rules}

\subsubsection{General rules}

\section{Linkage section}
\begin{syntax}
  \key{LINKAGE} \key{SECTION}.\newline
  \begin{0-1}
    \metaelement{constant-definition} \\
    \metaelement{record-description}
  \end{0-1}\ldots
\end{syntax}

\subsubsection{Syntax rules}

\subsubsection{General rules}

\section{Report section}
\begin{syntax}
  \pending{
    \key{REPORT} \key{SECTION}.\newline
    \begin{0-1}
      \metaelement{constant-definition} \\
      \metaelement{report-description}
    \end{0-1}\ldots
  }
\end{syntax}

\subsubsection{Syntax rules}

\subsubsection{General rules}

\subsection{Report description}

\begin{syntax}
  \key{RD} \reportname

  \begin{0-1}
    IS \key{GLOBAL} \\

    \key{CODE} IS
    \begin{1=}
      \identifier \\
      \literal
    \end{1=} \\

    \begin{1=}
      \key{CONTROL} IS \\
      \key{CONTROLS} ARE
    \end{1=}
    FINAL
    \begin{1=}
      \identifier
    \end{1=}\ldots \\

    \key{PAGE}
    \begin{0-1}
      \key{LIMIT} IS \\
      \key{LIMITS} ARE
    \end{0-1}
    \integer
    \begin{0-1}
      \key{LINE} \\
      \key{LINES}
    \end{0-1}
    \begin{0-1}
      \integer
      \begin{1=}
        \key{COLUMNS} \\
        \key{COLS}
      \end{1=}
    \end{0-1}
  \end{0-1}\ldots\ {}.\newline

  \begin{1=}
    \metaelement{report-group-description-1}
  \end{1=}\ldots
\end{syntax}

where \metaelement{report-group-description} is

\begin{syntax}
  \pending{
    \metaelement{level-number} \metaelement{entry-name}
    \begin{0-1}
      \metaelement{blank-clause} \\
      \metaelement{column-clause} \\
      \key{GROUP} INDICATE \\
      \metaelement{justified-clause} \\
      \metaelement{line-clause} \\
      \metaelement{next-group-clause} \\
      \metaelement{picture-clause} \\
      \metaelement{present-when-clause} \\
      \metaelement{occurs-clause} \\
      \metaelement{sign-clause} \\
      \metaelement{source-clause} \\
      \metaelement{sum-clause} \\
      \metaelement{type-clause} \\
      \key{USAGE} IS \key{DISPLAY} \\
      \metaelement{value-clause} \\
      \metaelement{varying-clause}
    \end{0-1}\ldots\ {}.
  }
\end{syntax}

\subsubsection{Syntax rules}

\subsubsection{General rules}

\section{Screen section}
\begin{syntax}
  \key{SCREEN} \key{SECTION}.\newline
  \begin{0-1}
    \metaelement{constant-definition} \\
    \metaelement{screen-description}
  \end{0-1}\ldots
\end{syntax}

\subsubsection{Syntax rules}

\subsubsection{General rules}

\subsection{Screen description}
\begin{syntax}
  \metaelement{level-number} \metaelement{entry-name}
  \begin{0-1}
    \metaelement{appearance-attribute-clauses} \\

    \begin{1=}
      \key{AUTO} \\
      \miscext{\key{AUTO-SKIP}} \\
      \miscext{\key{AUTOTERMINATE}} \\
    \end{1=} \\

    \metaelement{column-clause} \\

    \key{ERASE}
    \begin{1=}
      \key{EOL} \\
      \key{EOS} \\

      \begin{0-1}
        \key{END} OF
      \end{0-1}
      \begin{1=}
        \key{LINE} \\
        \key{SCREEN}
      \end{1=}
    \end{1=} \\

    \begin{1=}
      \key{FULL} \\
      \miscext{\key{LENGTH-CHECK}} \\
    \end{1=} \\

    IS \key{GLOBAL} \\
    \miscext{\pending{\key{GRID}}} \\

    \miscext{\key{INITIAL}} \\
    \miscext{\pending{\key{LEFTLINE}}} \\
    \metaelement{justified-clause} \\
    \metaelement{line-clause} \\

    \miscext{
      \begin{1=}
        \key{NO-ECHO} \\
        \key{NO} \key{ECHO} \\
        \key{OFF}
      \end{1=}
    } \\

    \metaelement{occurs-clause} \\
    \metaelement{picture-clause} \\

    \begin{1=}
      \key{REQUIRED} \\
      \miscext{\key{EMPTY-CHECK}}
    \end{1=} \\

    \metaelement{source-destination-clauses} \\

    \key{SECURE} \\
    \metaelement{sign-clause} \\
    \metaelement{usage-clause} \\
    \metaelement{value-clause}
  \end{0-1}\ldots\ {}.
\end{syntax}

where \metaelement{appearance-attribute-clauses} is

\begin{syntax}
  \begin{0-1}
    \begin{1=}
      \key{BACKGROUND-COLOR} \\
      \miscext{\key{BACKGROUND-COLOUR}}
    \end{1=}
    IS
    \begin{1=}
      \identifier \\
      \integer
    \end{1=}
  \end{0-1}

  \begin{0-1}
    \key{BELL} \\
    \key{BEEP}
  \end{0-1}

  \key{BLANK}
  \begin{1=}
    \key{LINE} \\
    \key{SCREEN}
  \end{1=}

  \begin{1=}
    \key{BLINK} \\
    \miscext{\key{BLINKING}}
  \end{1=}

  \begin{0-1}
    \begin{1=}
      \key{FOREGROUND-COLOR} \\
      \miscext{\key{FOREGROUND-COLOUR}}
    \end{1=}
    IS
    \begin{1=}
      \identifier \\
      \integer
    \end{1=}
  \end{0-1}

  \begin{0-1}
    \key{HIGHLIGHT} \\
    \miscext{\key{HIGH}} \\
    \miscext{\key{BOLD}} \\
    \key{LOWLIGHT} \\
    \miscext{\key{LOW}}
  \end{0-1}

  \begin{0-1}
    \miscext{\pending{\key{OVERLINE}}}
  \end{0-1}

  \miscext{
    \begin{0-1}
      \key{PROMPT}
      \begin{0-1}
        \key{CHARACTER} IS
        \begin{1=}
          \identifier \\
          \literal
        \end{1=}
      \end{0-1}
    \end{0-1}
  }

  \begin{0-1}
    \key{REVERSE-VIDEO} \\
    \miscext{\key{REVERSED}} \\
    \miscext{\key{REVERSE}}
  \end{0-1}

  \begin{0-1}
    \key{UNDERLINE} \\
    \miscext{\key{UNDERLINED}}
  \end{0-1}
\end{syntax}

where \metaelement{source-destination-clauses} is

\begin{syntax}
  \begin{0-1}
    \key{FROM}
    \begin{1=}
      \identifier \\
      \literal
    \end{1=} \\
  \end{0-1}

  \begin{0-1}
    \key{TO} \identifier
  \end{0-1}

  \begin{0-1}
    \key{USING} \identifier
  \end{0-1}
\end{syntax}

\subsubsection{Syntax rules}

\subsubsection{General rules}

\section{Record description}
\format{data-description}
\begin{syntax}
  \metaelement{level-number} \metaelement{entry-name}
  \begin{0-1}
    \key{ANY}
    \begin{1=}
      \key{LENGTH} \\
      \gnucobol{\key{NUMERIC}}
    \end{1=} \\

    \metaelement{blank-clause} \\

    IS \key{EXTERNAL}
    \begin{0-1}
      \key{AS} \literal
    \end{0-1} \\

    IS \key{GLOBAL} \\

    \metaelement{justified-clause} \\

    \metaelement{occurs-clause} \\

    \key{PICTURE} \metaelement{picture-string-1} \\

    \key{REDEFINES} \identifier \\

    \metaelement{sign-clause} \\

    \begin{1=}
      \key{SYNCHRONIZED} \\
      \miscext{\key{SYNCHRONISED}} \\
      \key{SYNC}
    \end{1=}
    \begin{0-1}
      \key{LEFT} \\
      \key{RIGHT}
    \end{0-1} \\

    \metaelement{usage-clause} \\

    \metaelement{value-clause}
  \end{0-1}\ldots\ {}.
\end{syntax}

\format{renames}
\begin{syntax}
  66 \identifier \key{RENAMES} \identifier
  \begin{0-1}
    \begin{1=}
      \key{THROUGH} \\
      \key{THRU}
    \end{1=}
    \identifier
  \end{0-1}.
\end{syntax}

\format{condition-name}
\begin{syntax}
  88 \identifier
  \begin{1=}
    \key{VALUE} \\
    \key{VALUES}
  \end{1=}
  \begin{0-1}
    IS \\
    ARE
  \end{0-1}
  \begin{1=}
    \literal
    \begin{0-1}
      \begin{1=}
        \key{THROUGH} \\
        \key{THRU}
      \end{1=}
      \literal
    \end{0-1}
  \end{1=} \ldots

  \begin{0-1}
    WHEN SET TO \key{FALSE} IS \literal
  \end{0-1}.
\end{syntax}

\subsubsection{Syntax rules}

\subsubsection{General rules}

\section{Constant definition}

\format{standard}
\begin{syntax}
  \begin{1=}
    1 \\
    01
  \end{1=}
  \identifier \key{CONSTANT}
  \begin{0-1}
    IS \key{GLOBAL}
  \end{0-1}
  \begin{1=}
    AS
    \begin{1=}
      \literal \\
      \begin{1=}
        \key{BYTE-LENGTH} \\
        \key{LENGTH}
      \end{1=}
      OF \identifier
    \end{1=} \\
    \pending{\key{FROM} \identifier}
  \end{1=}.
\end{syntax}

\format{micro-focus}
\begin{syntax}
  \miscext{
    78 \identifier
    \gnucobol{
      \begin{0-1}
        IS \key{GLOBAL}
      \end{0-1}
    }
    \key{VALUE}
    \begin{0-1}
      IS \\
      ARE
    \end{0-1}
    \literal .
  }
\end{syntax}

\subsubsection{Syntax rules}

\subsubsection{General rules}

\section{Data division clauses}

\subsection{BLANK WHEN ZERO clause}
\begin{syntax}
  \key{BLANK} WHEN \key{ZERO}
\end{syntax}

\subsubsection{Syntax rules}

\subsubsection{General rules}

\subsection{BLOCK clause}
\begin{syntax}
  \key{BLOCK} CONTAINS \integer
  \begin{0-1}
    \key{TO} \integer
  \end{0-1}
  \begin{0-1}
    \key{CHARACTERS} \\
    \key{RECORDS}
  \end{0-1}\\
\end{syntax}

\subsubsection{Syntax rules}

\subsubsection{General rules}

\subsection{COLUMN clause}

\format{report-section}
\begin{syntax}
  \begin{1=}
    \begin{1=}
      \key{COLUMN} \\
      \key{COL}
    \end{1=}
    NUMBERS
    \begin{0-1}
      \key{IS} \\
      \key{ARE}
    \end{0-1} \\

    \begin{1=}
      \key{COLUMNS} \\
      \key{COLS}
    \end{1=}
    ARE
  \end{1=}
  \begin{1=}
    \begin{0-1}
      \key{PLUS}
    \end{0-1}
    \integer
  \end{1=}
  \ldots
\end{syntax}

\format{screen-section}
\begin{syntax}
  \begin{1=}
    \key{COLUMN} \\
    \key{COL}
  \end{1=}
  NUMBER IS
  \begin{0-1}
    + \\
    - \\
    \key{PLUS} \\
    \key{MINUS}
  \end{0-1}
  \begin{1=}
    \identifier \\
    \integer
  \end{1=}
\end{syntax}

\subsubsection{Syntax rules}

\subsubsection{General rules}

\subsection{Entry name}
\begin{syntax}
  \begin{0-1}
    \key{FILLER} \\
    \identifier
  \end{0-1}
\end{syntax}

\subsubsection{Syntax rules}

\subsubsection{General rules}

\subsection{JUSTIFIED clause}
\begin{syntax}
  \begin{1=}
    \key{JUSTIFIED} \\
    \key{JUST}
  \end{1=}
  RIGHT
\end{syntax}

\subsubsection{Syntax rules}

\subsubsection{General rules}

\subsection{LINAGE clause}
\begin{syntax}
  \key{LINAGE} IS
  \begin{1=}
    \identifier \\
    \literal
  \end{1=}
  LINES
  \begin{0-1}
    \begin{1=}
      \key{BOTTOM} \\
      \key{TOP} \\
      WITH \key{FOOTING} AT
    \end{1=}
    \begin{1=}
      \identifier \\
      \literal
    \end{1=} \\
  \end{0-1}
  \ldots \\
\end{syntax}

\subsubsection{Syntax rules}

\subsubsection{General rules}

\subsection{LINE clause}

\format{report section}
\begin{syntax}
  \begin{1=}
    \key{LINE} NUMBERS
    \begin{0-1}
      \key{IS} \\
      \key{ARE}
    \end{0-1} \\

    \key{LINES} ARE
  \end{1=}
  \begin{1=}
    \begin{0-1}
      \key{PLUS}
    \end{0-1}
    \integer \\
    \key{NEXT} \key{PAGE}
  \end{1=}\ldots
\end{syntax}

\format{screen section}
\begin{syntax}
  \key{LINE} NUMBER IS
  \begin{0-1}
    + \\
    - \\
    \key{MINUS} \\
    \key{PLUS}
  \end{0-1}
  \begin{1=}
    \identifier \\
    \integer
  \end{1=}
\end{syntax}

\subsubsection{Syntax rules}

\subsubsection{General rules}

\subsection{Level-number}

A 1- or 2-digit integer having a value that is either between 1 and 49 or is 66, 77, \miscext{78} or 88.

\subsubsection{Syntax rules}

\subsubsection{General rules}

\subsection{NEXT GROUP clause}
\begin{syntax}
  \key{NEXT} \key{GROUP} IS
  \begin{1=}
    \begin{0-1}
      \key{PLUS}
    \end{0-1}
    \integer \\
    \key{NEXT} \key{PAGE}
  \end{1=}
\end{syntax}

\subsubsection{Syntax rules}

\subsubsection{General rules}

\subsection{OCCURS clause}

\format{usual}
\begin{syntax}
  \key{OCCURS}

  \begin{1=}
    \integer
    \begin{0-1}
      \key{TO} \integer
    \end{0-1}
    TIMES
    \begin{0-1}
      \key{DEPENDING} ON \identifier
    \end{0-1} \\

    \pending{
      \key{DYNAMIC}
      \begin{0-1}
        \key{CAPACITY} IN \identifier
      \end{0-1}
      \begin{0-1}
        \key{FROM} \integer
      \end{0-1}
      \begin{0-1}
        \key{TO} \integer
      \end{0-1}
      \begin{0-1}
        \key{INITIALIZED}
      \end{0-1}
    }
  \end{1=}

  \begin{0-1}
    \begin{1=}
      \key{ASCENDING} \\
      \key{DESCENDING}
    \end{1=}
    KEY IS
    \begin{1=}
      \identifier
    \end{1=}\ldots
  \end{0-1}\ldots

  \begin{0-1}
    \key{INDEXED} BY
    \begin{1=}
      \cobolindexname
    \end{1=}\ldots
  \end{0-1}
\end{syntax}

\format{report section}
\begin{syntax}
  \key{OCCURS} \integer
  \begin{0-1}
    \key{TO} \integer
  \end{0-1}
  TIMES
  \begin{0-1}
    \key{DEPENDING} ON \identifier
  \end{0-1}
  \begin{0-1}
    \key{STEP} \integer
  \end{0-1}
\end{syntax}

\format{screen section}
\begin{syntax}
  \key{OCCURS} \integer TIMES
\end{syntax}

\subsubsection{Syntax rules}

\subsubsection{General rules}

\subsection{PRESENT WHEN clause}
\begin{syntax}
  \key{PRESENT} \key{WHEN} \condition
\end{syntax}

\subsubsection{Syntax rules}

\subsubsection{General rules}

\subsection{RECORD clause}
\begin{syntax}
  \key{RECORD}
  \begin{1=}
    CONTAINS \integer
    \begin{0-1}
      \key{TO} \integer
    \end{0-1}
    CHARACTERS \\
    IS \key{VARYING} in size
    \begin{0-1}
      FROM \integer
    \end{0-1}
    \begin{0-1}
      \key{TO} \integer
    \end{0-1}
    CHARACTERS \\\qquad
    \key{DEPENDING} ON \identifier
  \end{1=}
\end{syntax}

\subsubsection{Syntax rules}

\subsubsection{General rules}

\subsection{SIGN clause}
\begin{syntax}
  SIGN IS
  \begin{1=}
    \key{LEADING} \\
    \key{TRAILING}
  \end{1=}
  \begin{0-1}
    \key{SEPARATE} CHARACTER
  \end{0-1}
\end{syntax}

\subsubsection{Syntax rules}

\subsubsection{General rules}

\subsection{SOURCE clause}
\begin{syntax}
  \key{SOURCE} IS \metaelement{number-1}
  \begin{0-1}
    \metaelement{rounded-phrase}
  \end{0-1}
\end{syntax}

\subsubsection{Syntax rules}

\subsubsection{General rules}

\subsection{SUM clause}
\begin{syntax}
  \key{SUM} OF
  \begin{1=}
    \metaelement{number-1}
  \end{1=}\ldots
  \begin{0-1}
    \key{RESET} ON
    \begin{1=}
      \identifier \\
      \key{FINAL}
    \end{1=}
  \end{0-1}
\end{syntax}

\subsubsection{Syntax rules}

\subsubsection{General rules}

\subsection{TYPE clause}
\begin{syntax}
  \key{TYPE} IS
  \begin{1=}
    \begin{1=}
      \begin{1=}
        \key{CONTROL} \key{HEADING} \\
        \key{CH}
      \end{1=} \\
      \begin{1=}
        \key{CONTROL} \key{FOOTING} \\
        \key{CF}
      \end{1=}
    \end{1=}
    \begin{1=}
      \identifier \\
      \key{FINAL}
    \end{1=}
    \begin{0-1}
      \key{OR} \key{PAGE}
    \end{0-1} \\

    \begin{1=}
      \key{DETAIL} \\
      \key{DE}
    \end{1=} \\

    \begin{1=}
      \key{PAGE} \key{FOOTING} \\
      \key{PF}
    \end{1=} \\

    \begin{1=}
      \key{PAGE} \key{HEADING} \\
      \key{PH}
    \end{1=} \\

    \begin{1=}
      \key{REPORT} \key{FOOTING} \\
      \key{RF}
    \end{1=} \\

    \begin{1=}
      \key{REPORT} \key{HEADING} \\
      \key{RH}
    \end{1=}
  \end{1=}
\end{syntax}

\subsubsection{Syntax rules}

\subsubsection{General rules}

\subsection{USAGE clause}

\begin{syntax}
  \begin{0-1}
    \key{USAGE} IS
  \end{0-1}
  \begin{1=}
    \key{BINARY} \\

    \metaelement{fixed-length-integers} \\
    \metaelement{computational-usages} \\

    \key{DISPLAY} \\

    \pending{\key{FLOAT-BINARY-32}} \\
    \pending{\key{FLOAT-BINARY-64}} \\
    \pending{\key{FLOAT-BINARY-128}} \\
    \key{FLOAT-DECIMAL-16} \\
    \key{FLOAT-DECIMAL-34} \\
    \key{FLOAT-LONG} \\
    \key{FLOAT-SHORT} \\

    \key{INDEX} \\

    \pending{\key{NATIONAL}} \\

    \key{PACKED} \key{DECIMAL} \\
    \miscext{\key{POINTER}} \\
    \key{PROGRAM-POINTER} \\
  \end{1=}
\end{syntax}

where \metaelement{fixed-length-integers} is

\begin{syntax}
  \begin{1=}
    \begin{1=}
      \key{BINARY-CHAR} \\

      \begin{1=}
        \key{BINARY-LONG} \\
        \gnucobol{\key{BINARY-INT}}
      \end{1=} \\

      \gnucobol{\key{BINARY-C-LONG}} \\

      \begin{1=}
        \key{BINARY-DOUBLE} \\
        \gnucobol{\key{BINARY-LONG-LONG}} \\
      \end{1=} \\
    \end{1=}
    \begin{0-1}
      \key{SIGNED} \\
      \key{UNSIGNED}
    \end{0-1} \\

    \miscext{\key{SIGNED-SHORT}} \\
    \miscext{\key{SIGNED-INT}} \\
    \miscext{\key{SIGNED-LONG}} \\

    \miscext{\key{UNSIGNED-SHORT}} \\
    \miscext{\key{UNSIGNED-INT}} \\
    \miscext{\key{UNSIGNED-LONG}}
  \end{1=}
\end{syntax}

where \metaelement{computation-usages} is

\begin{syntax}
  \begin{1=}
    \begin{1=}
      \key{COMP} \\
      \key{COMPUTATIONAL}
    \end{1=} \\

    \miscext{
      \begin{1=}
        \key{COMP-1} \\
        \key{COMPUTATIONAL-1}
      \end{1=}
    } \\

    \miscext{
      \begin{1=}
        \key{COMP-2} \\
        \key{COMPUTATIONAL-2}
      \end{1=}
    } \\

    \xopen{
      \begin{1=}
        \key{COMP-3} \\
        \key{COMPUTATIONAL-3}
      \end{1=}
    } \\

    \miscext{
      \begin{1=}
        \key{COMP-4} \\
        \key{COMPUTATIONAL-4}
      \end{1=}
    } \\

    \xopen{
      \begin{1=}
        \key{COMP-5} \\
        \key{COMPUTATIONAL-5}
      \end{1=}
    } \\

    \miscext{
      \begin{1=}
        \key{COMP-6} \\
        \key{COMPUTATIONAL-6}
      \end{1=}
    } \\

    \miscext{
      \begin{1=}
        \key{COMP-X} \\
        \key{COMPUTATIONAL-X}
      \end{1=}
    } \\
  \end{1=}
\end{syntax}

\subsubsection{Syntax rules}

\subsubsection{General rules}

\subsection{VALUE clause}
\format{initialization}
\begin{syntax}
  \begin{1=}
    \key{VALUE} \\
    \gnucobol{\key{VALUES}}
  \end{1=}
  \begin{0-1}
    IS \\
    \gnucobol{ARE}
  \end{0-1}
  \literal
\end{syntax}

\format{condition}
\begin{syntax}
  \begin{1=}
    \key{VALUE} \\
    \key{VALUES}
  \end{1=}
  \begin{0-1}
    IS \\
    ARE
  \end{0-1}
  \begin{1=}
    \literal
    \begin{0-1}
      \begin{1=}
        \key{THROUGH} \\
        \key{THRU}
      \end{1=}
      \literal
    \end{0-1}
  \end{1=} \ldots

  \begin{0-1}
    WHEN SET TO \key{FALSE} IS \literal
  \end{0-1}
\end{syntax}

\subsubsection{Syntax rules}

\subsubsection{General rules}

\subsection{VARYING clause}
\begin{syntax}
  \key{VARYING} \identifier \key{FROM} \metaelement{number-1} \key{BY} \metaelement{number-2}
\end{syntax}

\subsubsection{Syntax rules}

\subsubsection{General rules}

%%% Local Variables:
%%% mode: latex
%%% TeX-master: "grammar.tex"
%%% End:

\chapter{Procedure division}

\begin{syntax}
  \key{PROCEDURE} \key{DIVISION}
  \begin{0-1}
    \metaelement{using-chaining-clause}
  \end{0-1}
  \begin{0-1}
    \key{RETURNING}
    \begin{1=}
      \identifier \\
      \key{OMITTED}
    \end{1=}
  \end{0-1}.\newline
  \begin{0-1}
    \metaelement{declaratives}
  \end{0-1}\newline
  \begin{0-1}
    \metaelement{section-name-2} \key{SECTION}. \\
    \metaelement{paragraph-name-2}. \\
    \imperativestatement .
  \end{0-1} \ldots
\end{syntax}

where \metaelement{using-chaining-clause} is

\begin{syntax}
  \begin{1=}
    \key{USING} \\
    \miscext{\key{CHAINING}}
  \end{1=}

  \begin{1=}
    BY
    \begin{1=}
      \key{REFERENCE} \\
      \pending{\key{VALUE}}
    \end{1=}
    \miscext{
      \begin{0-1}
        \begin{0-1}
          \key{UNSIGNED}
        \end{0-1}
        \key{SIZE} IS
        \begin{1=}
          \key{AUTO} \\
          \integer
        \end{1=} \\

        \key{SIZE} IS \key{DEFAULT}
      \end{0-1}
    }

    \begin{0-1}
      \key{OPTIONAL}
    \end{0-1}
    \identifier
  \end{1=}\ldots
\end{syntax}

where \metaelement{declaratives} is

\begin{syntax}
  \key{DECLARATIVES}.\newline
  \begin{0-1}
    \metaelement{section-name-1} \key{SECTION}.
    \metaelement{use-statement}
    \begin{0-1}
      \metaelement{paragraph-name-2}. \\
      \imperativestatement .
    \end{0-1} \ldots
  \end{0-1}\ldots\newline
  \key{END} \key{DECLARATIVES}.
\end{syntax}

\subsubsection{Syntax rules}

\subsubsection{General rules}

\section{Common phrases}

\subsection{RETRY phrase}

\begin{syntax}
  \pending{
    \key{RETRY}
    \begin{1=}
      \gnucobol{FOR} \arithmeticexpression \key{TIMES} \\
      FOR \arithmeticexpression \key{SECONDS} \\
      \key{FOREVER}
    \end{1=}
  }
\end{syntax}

\subsubsection{Syntax rules}

\subsubsection{General rules}


\subsection{ROUNDED phrase}

\begin{syntax}
  \key{ROUNDED}
  \begin{0-1}
    \key{MODE} IS
    \begin{1=}
      \key{AWAY-FROM-ZERO} \\
      \key{NEAREST-AWAY-FROM-ZERO} \\
      \key{NEAREST-EVEN} \\
      \key{NEAREST-TOWARD-ZERO} \\
      \key{PROHIBITED} \\
      \key{TOWARD-GREATER} \\
      \key{TOWARD-LESSER} \\
      \key{TRUNCATION}
    \end{1=}
  \end{0-1}
\end{syntax}

\subsubsection{Syntax rules}

\subsubsection{General rules}


\subsection{SIZE phrase}

\begin{syntax}
  \gnucobol{
    \begin{1=}
      \key{SIZE} IS \key{AUTO} \\
      \key{SIZE} IS \key{DEFAULT} \\
      \key{SIZE} IS \integer \\
      \key{UNSIGNED} \key{SIZE} IS \key{AUTO} \\
      \key{UNSIGNED} \key{SIZE} IS \integer
    \end{1=}
  }
\end{syntax}

\subsubsection{Syntax rules}

\subsubsection{General rules}

\section{ACCEPT statement}

\format{device}
\begin{syntax}
  \key{ACCEPT}
  \begin{1=}
    \identifier \\
    \miscext{\key{OMITTED}}
  \end{1=}
  \begin{0-1}
    \key{FROM} \mnemonicname
  \end{0-1}
  \begin{0-1}
    \key{END-ACCEPT}
  \end{0-1}
\end{syntax}

\format{screen}
\begin{syntax}
  \key{ACCEPT}
  \begin{1=}
    \identifier \\
    \miscext{\key{OMITTED}}
  \end{1=}

  \begin{0+}
    \begin{1=}
      \begin{1+}
        AT \key{LINE} NUMBER
        \begin{1=}
          \identifier \\
          \integer
        \end{1=} \\

        AT
        \begin{1=}
          \key{COLUMN} \\
          \key{COL} \\
          \miscext{\key{POSITION}}
        \end{1=}
        NUMBER
        \begin{1=}
          \identifier \\
          \integer
        \end{1=}
      \end{1+} \\

      \miscext{
        \key{AT}
        \begin{1=}
          \identifier \\
          \integer
        \end{1=}
      }
    \end{1=} \\

    \miscext{\key{FROM} \key{CRT}} \\
    \miscext{\key{MODE} IS \key{BLOCK}} \\
    \miscext{\metaelement{appearance-attribute-clauses}} \\
    \miscext{\metaelement{accept-attribute-clauses}}
  \end{0+}
\end{syntax}

where \metaelement{appearance-attribute-clauses} is

\begin{syntax}

  \begin{0-1}
    WITH
    \begin{1=}
      \key{BELL} \\
      \key{BEEP}
    \end{1=}
  \end{0-1}

  \begin{0-1}
    WITH
    \begin{1=}
      \key{BLINK} \\
      \key{BLINKING}
    \end{1=}
  \end{0-1}

  \begin{0-1}
    WITH
    \begin{1=}
      \key{HIGHLIGHT} \\
      \key{HIGH} \\
      \key{BOLD} \\
      \key{LOWLIGHT} \\
      \key{LOW}
    \end{1=}
  \end{0-1}

  \begin{0-1}
    WITH \key{LEFTLINE}
  \end{0-1}

  \begin{0-1}
    WITH \key{OVERLINE}
  \end{0-1}

  \begin{0-1}
    WITH \key{PROMPT}
    \begin{0-1}
      \key{CHARACTER} IS
      \begin{1=}
        \identifier \\
        \literal
      \end{1=}
    \end{0-1}
  \end{0-1}

  \begin{0-1}
    WITH
    \begin{1=}
      \key{REVERSE-VIDEO} \\
      \key{REVERSED} \\
      \key{REVERSE}
    \end{1=}
  \end{0-1}


  \begin{0-1}
    WITH PROTECTED \key{SIZE} IS
    \begin{1=}
      \identifier \\
      \integer
    \end{1=}
  \end{0-1}
  
  \begin{0-1}
    WITH
    \begin{1=}
      \key{UNDERLINE} \\
      \key{UNDERLINED}
    \end{1=}
  \end{0-1}

  \begin{0-1}
    WITH
    \begin{1=}
      \key{FOREGROUND-COLOR} \\
      \key{FOREGROUND-COLOUR}
    \end{1=}
    IS
    \begin{1=}
      \identifier \\
      \integer
    \end{1=}
  \end{0-1}

  \begin{0-1}
    WITH
    \begin{1=}
      \key{BACKGROUND-COLOR} \\
      \key{BACKGROUND-COLOUR}
    \end{1=}
    IS
    \begin{1=}
      \identifier \\
      \integer
    \end{1=}
  \end{0-1}

  \begin{0-1}
    WITH \key{SCROLL}
    \begin{1=}
      \key{UP} \\
      \key{DOWN}
    \end{1=}
    \begin{0-1}
      \begin{1=}
        \identifier \\
        \integer
      \end{1=}
      \begin{1=}
        \key{LINE} \\
        \key{LINES}
      \end{1=}
    \end{0-1}
  \end{0-1}
\end{syntax}

where \metaelement{accept-attribute-clauses} is

\begin{syntax}
  \begin{0-1}
    WITH
    \begin{1=}
      \key{AUTO} \\
      \key{TAB}
    \end{1=}
  \end{0-1}

  \begin{0-1}
    WITH
    \begin{1=}
      \key{CONVERSION} \\
      \key{CONVERT}
    \end{1=}
  \end{0-1}

  \begin{0-1}
    WITH
    \begin{1=}
      \key{FULL} \\
      \key{LENGTH-CHECK} \\
    \end{1=}
  \end{0-1}

  \begin{0-1}
    WITH
    \begin{1=}
      \key{LOWER} \\
      \key{UPPER}
    \end{1=}
  \end{0-1}

  \begin{0-1}
    WITH
    \begin{1=}
      \key{NO-ECHO} \\
      \key{NO} \key{ECHO} \\
      \key{OFF}
    \end{1=}
  \end{0-1}
  
  \begin{0-1}
    WITH
    \begin{1=}
      \key{REQUIRED} \\
      \key{EMPTY-CHECK}
    \end{1=}
  \end{0-1}

  \begin{0-1}
    WITH \key{SECURE}
  \end{0-1}

  \begin{0-1}
    WITH
    \begin{0-1}
      \key{NO}
    \end{0-1}
    \begin{1=}
      \key{DEFAULT} \\
      \key{UPDATE}
    \end{1=}
  \end{0-1}
  
  \begin{0-1}
    \begin{1=}
      \key{WITH}
      % TIME-OUT is context-sensitive, so WITH is required.
      \begin{1=}
        \key{TIMEOUT} \\
        \key{TIME-OUT} \\
      \end{1=}
      AFTER \\

      BEFORE \key{TIME}
    \end{1=}

    \begin{1=}
      \identifier \\
      \integer
    \end{1=}
  \end{0-1}
\end{syntax}

% Shouldn't all the clauses except AT LINE/COLUMN be miscext?

\format{temporal}
\begin{syntax}
  \key{ACCEPT} \identifier \key{FROM}
  \begin{1=}
    \key{DATE}
    \begin{0-1}
      \key{YYYYMMDD}
    \end{0-1} \\

    \key{DAY}
    \begin{0-1}
      \key{YYYYDDD}
    \end{0-1} \\

    \key{DAY-OF-WEEK} \\
    \key{TIME} \\
  \end{1=}
\end{syntax}

\format{environment}
\begin{syntax}
  \miscext{
    \key{ACCEPT} \identifier \key{FROM}
    \begin{1=}
      \key{ARGUMENT-NUMBER} \\

      \begin{1=}
        \key{COLUMNS} \\
        \key{COLS}
      \end{1=} \\

      \key{COMMAND-LINE} \\
      \key{ESCAPE} \key{KEY} \\
      \key{EXCEPTION} \key{STATUS} \\

      \begin{1=}
        \key{LINES} \\
        \key{LINE} \key{NUMBER}
      \end{1=} \\

      \key{USER} \key{NAME} \\
      \key{WORD}
    \end{1=}
  }
\end{syntax}

\format{environment-exception}
% TO-DO: Fix!
\begin{syntax}
  \miscext{
    \begin{minipage}[!h]{1.0\linewidth}
      \key{ACCEPT} \identifier \key{FROM}
      \begin{1=}
        \key{ARGUMENT-VALUE} \\
        \key{ENVIRONMENT}
        \begin{1=}
          \identifier \\
          \literal
        \end{1=} \\
        \key{ENVIRONMENT-VALUE} \\
      \end{1=}

      \begin{0+}
        ON
        \begin{1=}
          \key{EXCEPTION} \\
          \key{ESCAPE}
        \end{1=}
        \imperativestatement \\

        \key{NOT} ON
        \begin{1=}
          \key{EXCEPTION} \\
          \key{ESCAPE}
        \end{1=}
        \imperativestatement \\
      \end{0+}
    \end{minipage}
  }
\end{syntax}

\subsubsection{Syntax rules}

\subsubsection{General rules}

\section{ADD statement}

\format{simple}
\begin{syntax}
  \key{ADD}
  \begin{1=}
    \identifier \\
    \literal \\
  \end{1=} \ldots
  \key{TO}
  \begin{1=}
    \identifier
  \end{1=} \ldots

  \begin{0+}
    ON \key{SIZE} \key{ERROR} \imperativestatement \\
    \key{NOT} ON \key{SIZE} \key{ERROR} \imperativestatement
  \end{0+}

  \begin{0-1}
    \key{END-ADD}
  \end{0-1}
\end{syntax}

\format{giving}
\begin{syntax}
  \key{ADD}
  \begin{1=}
    \identifier \\
    \literal \\
  \end{1=} \ldots
  \begin{0-1}
    \key{TO}
    \begin{0-1}
      \identifier
    \end{0-1} \ldots
  \end{0-1}

  \key{GIVING}
  \begin{1=}
    \identifier
    \begin{0-1}
      \metaelement{rounded-phrase}
    \end{0-1}
  \end{1=} \ldots

  \begin{0+}
    ON \key{SIZE} \key{ERROR} \imperativestatement \\
    \key{NOT} ON \key{SIZE} \key{ERROR} \imperativestatement
  \end{0+}

  \begin{0-1}
    \key{END-ADD}
  \end{0-1}
\end{syntax}

\format{corresponding}
\begin{syntax}
  \key{ADD}
  \begin{1=}
    \key{CORRESPONDING} \\
    \key{CORR}
  \end{1=}
  \identifier \key{TO} \identifier
  \begin{0-1}
    \metaelement{rounded-phrase}
  \end{0-1}

  \begin{0+}
    ON \key{SIZE} \key{ERROR} \imperativestatement \\
    \key{NOT} ON \key{SIZE} \key{ERROR} \imperativestatement
  \end{0+}

  \begin{0-1}
    \key{END-ADD}
  \end{0-1}
\end{syntax}

\subsubsection{Syntax rules}

\subsubsection{General rules}

\section{ALLOCATE statement}

\begin{syntax}
  \key{ALLOCATE}
  \begin{1=}
    \identifier
    \begin{0-1}
      \key{INITIALIZED}
    \end{0-1} \\
    \arithmeticexpression
    \begin{0-1}
      \key{INITIALIZED}
      \gnucobol{
        \begin{0-1}
          \key{TO}
          \begin{1=}
            \identifier \\
            \literal
          \end{1=}
        \end{0-1}
      }
    \end{0-1}
  \end{1=}

  \begin{0-1}
    \key{RETURNING} \identifier
  \end{0-1}
\end{syntax}

\subsubsection{Syntax rules}

\subsubsection{General rules}

\section{ALTER statement}

\begin{syntax}
  \deleted{
    \key{ALTER}
    \begin{1=}
      \procedurename TO PROCEED \key{TO} \procedurename
    \end{1=} \ldots
  }
\end{syntax}

\subsubsection{Syntax rules}

\subsubsection{General rules}

\section{CALL statement}

\begin{syntax}
  \key{CALL}
  \miscext{
    \begin{0-1}
      \mnemonicname \\
      \key{STATIC} \\
      \key{STDCALL}
    \end{0-1}
  }
  \begin{1=}
    \identifier \\
    \literal \\
    \functionname
  \end{1=}

  \begin{0-1}
    \key{USING}
    \begin{1=}
      \begin{0-1}
        BY
        \begin{1=}
          \key{REFERENCE} \\
          \key{CONTENT} \\
          \key{VALUE}
        \end{1=}
      \end{0-1}
      \begin{1=}
        \key{OMITTED} \\

        \gnucobol{
          \begin{0-1}
            size-phrase
          \end{0-1}
        }
        \begin{1=}
          \identifier \\
          \literal
        \end{1=}
      \end{1=}
    \end{1=}\ldots
  \end{0-1}

  \begin{0-1}
    \begin{1=}
      \key{RETURNING} \\
      \miscext{\key{GIVING}}
    \end{1=}
    \begin{1=}
      INTO \identifier \\
      \key{ADDRESS} OF \identifier \\
      \gnucobol{\key{NOTHING}} \\
      \key{NULL} \\
      \key{OMITTED} \\
    \end{1=}
  \end{0-1}

  \begin{0+}
    ON
    \begin{1=}
      \key{EXCEPTION} \\
      \archaic{\key{OVERFLOW}}
    \end{1=}
    \imperativestatement \\
    \key{NOT} ON \key{EXCEPTION} \imperativestatement
  \end{0+}

  \begin{0-1}
    \key{END-CALL}
  \end{0-1}
\end{syntax}

\subsubsection{Syntax rules}

\subsubsection{General rules}

\section{CANCEL statement}

\begin{syntax}
  \key{CANCEL}
  \begin{1=}
    \identifier \\
    \literal
  \end{1=} \ldots
\end{syntax}

\subsubsection{Syntax rules}

\subsubsection{General rules}

\section{CLOSE statement}

\begin{syntax}
  \key{CLOSE}
  \begin{1=}
    \filename
    \begin{0-1}
      \begin{1=}
        \key{REEL} \\
        \key{UNIT}
      \end{1=}
      \begin{0-1}
        FOR \key{REMOVAL}
      \end{0-1} \\

      WITH \key{NO} \key{REWIND} \\
      WITH \key{LOCK}
    \end{0-1}
  \end{1=} \ldots
\end{syntax}

\subsubsection{Syntax rules}

\subsubsection{General rules}

\section{COMMIT statement}

\begin{syntax}
  \miscext{\key{COMMIT}}
\end{syntax}

\subsubsection{Syntax rules}

\subsubsection{General rules}

\section{COMPUTE statement}

\begin{syntax}
  \key{COMPUTE}
  \begin{1=}
    \identifier
    \begin{0-1}
      \metaelement{rounded-phrase}
    \end{0-1}
  \end{1=} \ldots
  \begin{1=}
    = \\
    \miscext{\key{EQUAL}} \\
    \miscext{\key{EQUALS}}
  \end{1=}
  \arithmeticexpression

  \begin{0+}
    ON \key{SIZE} \key{ERROR} \imperativestatement \\
    \key{NOT} ON \key{SIZE} \key{ERROR} \imperativestatement
  \end{0+}

  \begin{0-1}
    \key{END-COMPUTE}
  \end{0-1}
\end{syntax}

\subsubsection{Syntax rules}

\subsubsection{General rules}

\section{CONTINUE statement}

\begin{syntax}
  \key{CONTINUE}
\end{syntax}

\subsubsection{Syntax rules}

\subsubsection{General rules}

\section{DELETE statement}

\format{record}
\begin{syntax}
  \key{DELETE} \filename RECORD

  \begin{0-1}
    \metaelement{retry-phrase}
  \end{0-1}
  
  \begin{0+}
    \key{INVALID} KEY \imperativestatement \\
    \key{NOT} \key{INVALID} KEY \imperativestatement
  \end{0+}

  \begin{0-1}
    \key{END-DELETE}
  \end{0-1}
\end{syntax}

\format{file}
\begin{syntax}
  \miscext{
    \key{DELETE} \key{FILE}
    \begin{1=}
      \filename
    \end{1=} \ldots
  }
  \gnucobol{
    \begin{0-1}
      \key{END-DELETE}
    \end{0-1}
  }
\end{syntax}

\subsubsection{Syntax rules}

\subsubsection{General rules}

\section{DISPLAY statement}

\format{device}
\begin{syntax}
  \key{DISPLAY}
  \begin{1=}
    \identifier \\
    \literal
  \end{1=} \ldots
  \begin{0+}
    \key{UPON} \mnemonicname \\
    WITH \key{NO} \key{ADVANCING}
  \end{0+}

  \begin{0+}
    ON \key{EXCEPTION} \imperativestatement \\
    \key{NOT} ON \key{EXCEPTION} \imperativestatement \\
  \end{0+}

  \begin{0-1}
    \key{END-DISPLAY}
  \end{0-1}
\end{syntax}

\format{environment}
\begin{syntax}
  \miscext{
    \begin{minipage}[!h]{1.0\linewidth}
      \key{DISPLAY}
      \begin{1=}
        \identifier \\
        \literal
      \end{1=}
      \key{UPON}
      \begin{1=}
        \key{ARGUMENT-NUMBER} \\
        \key{COMMAND-LINE} \\
        \key{ENVIRONMENT-NAME} \\
        \key{ENVIRONMENT-VALUE} \\
      \end{1=}

      \begin{0+}
        ON \key{EXCEPTION} \imperativestatement \\
        \key{NOT} ON \key{EXCEPTION} \imperativestatement \\
      \end{0+}

      \begin{0-1}
        \key{END-DISPLAY}
      \end{0-1}
    \end{minipage}
  }
\end{syntax}

\format{screen}
\begin{syntax}
  \key{DISPLAY}
  \begin{1=}
    \begin{1=}
      \identifier \\
      \miscext{\literal} \\
      \miscext{\key{OMITTED}}
    \end{1=}

    \begin{0-1}
      \metaelement{position-clauses} \\

      \miscext{
        \key{UPON}
        \begin{1=}
          \key{CRT} \\
          \key{CRT-UNDER}
        \end{1=}
      } \\

      \miscext{\key{MODE} IS \key{BLOCK}} \\ 
      \miscext{\metaelement{appearance-attribute-clauses}} 
    \end{0-1}
  \end{1=} \miscext{\ldots}

  \begin{0+}
    ON \key{EXCEPTION} \imperativestatement \\
    \key{NOT} ON \key{EXCEPTION} \imperativestatement \\
  \end{0+}

  \begin{0-1}
    \key{END-DISPLAY}
  \end{0-1}
\end{syntax}

where \metaelement{position-clauses} is

\begin{syntax}
  \begin{1=}
    \begin{1+}
      \key{LINE} NUMBER
      \begin{1=}
        \identifier \\
        \literal
      \end{1=} \\

      \key{AT}
      \begin{1=}
        \key{COLUMN} \\
        \key{COL} \\
        \miscext{\key{POSITION}}
      \end{1=}

      \begin{1=}
        \identifier \\
        \literal
      \end{1=}
    \end{1+} \\


    \miscext{
      \key{AT}
      \begin{1=}
        \identifier \\
        \literal
      \end{1=}
    }
  \end{1=}
\end{syntax}

where \metaelement{appearance-attribute-clauses} is

\begin{syntax}
  WITH
  \begin{1=}
    \key{BELL} \\
    \key{BEEP} \\
  \end{1=} \\

  WITH \key{BLANK}
  \begin{1=}
    \key{LINE} \\
    \key{SCREEN}
  \end{1=} \\

  WITH
  \begin{1=}
    \key{BLINK} \\
    \key{BLINKING}
  \end{1=} \\

  WITH
  \begin{1=}
    \key{CONVERSION} \\
    \key{CONVERT}
  \end{1=} \\

  WITH \key{ERASE}
  \begin{1=}
    \key{EOL} \\
    \key{EOS} \\

    \begin{0-1}
      \key{END} OF
    \end{0-1}
    \begin{1=}
      \key{LINE} \\
      \key{SCREEN}
    \end{1=}
  \end{1=} \\

  WITH
  \begin{1=}
    \key{HIGHLIGHT} \\
    \key{HIGH} \\
    \key{BOLD} \\
    \key{LOWLIGHT} \\
    \key{LOW}
  \end{1=} \\

  WITH \key{OVERLINE} \\

  WITH
  \begin{1=}
    \key{REVERSE-VIDEO} \\
    \key{REVERSED} \\
    \key{REVERSE}
  \end{1=} \\

  WITH \key{SIZE} IS
  \begin{1=}
    \identifier \\
    \literal
  \end{1=} \\

  WITH
  \begin{1=}
    \key{UNDERLINE} \\
    \key{UNDERLINED}
  \end{1=} \\

  WITH
  \begin{1=}
    \key{FOREGROUND-COLOR} \\
    \key{FOREGROUND-COLOUR}
  \end{1=}
  IS
  \begin{1=}
    \identifier \\
    \integer
  \end{1=} \\

  WITH
  \begin{1=}
    \key{BACKGROUND-COLOR} \\
    \key{BACKGROUND-COLOUR}
  \end{1=}
  IS
  \begin{1=}
    \identifier \\
    \integer
  \end{1=} \\

  WITH \key{SCROLL}
  \begin{1=}
    \key{UP} \\
    \key{DOWN}
  \end{1=} 
  \begin{0-1}
    \begin{1=}
      \identifier \\
      \integer
    \end{1=}
    \begin{1=}
      \key{LINE} \\
      \key{LINES}
    \end{1=}
  \end{0-1}
\end{syntax}

\subsubsection{Syntax rules}

\subsubsection{General rules}

\section{DIVIDE statement}

\format{into}
\begin{syntax}
  \key{DIVIDE}
  \begin{1=}
    \identifier \\
    \literal
  \end{1=}
  \key{INTO}
  \begin{1=}
    \begin{1=}
      \identifier \\
      \literal
    \end{1=}
    \begin{0-1}
      \metaelement{rounded-phrase}
    \end{0-1}
  \end{1=} \ldots

  \begin{0+}
    ON \key{SIZE} \key{ERROR} \imperativestatement \\
    \key{NOT} ON \key{SIZE} \key{ERROR} \imperativestatement
  \end{0+}

  \begin{0-1}
    \key{END-DIVIDE}
  \end{0-1}
\end{syntax}

\format{giving}
\begin{syntax}
  \key{DIVIDE}
  \begin{1=}
    \identifier \\
    \literal
  \end{1=}
  \begin{1=}
    \key{BY} \\
    \key{INTO}
  \end{1=}
  \begin{1=}
    \identifier \\
    \literal
  \end{1=}

  \key{GIVING}
  \begin{1=}
    \begin{1=}
      \identifier \\
      \literal
    \end{1=}
    \begin{0-1}
      \metaelement{rounded-phrase}
    \end{0-1}
  \end{1=}
  \ldots

  \begin{0-1}
    \key{REMAINDER}
    \begin{1=}
      \identifier \\
      \literal
    \end{1=}
  \end{0-1}

  \begin{0+}
    ON \key{SIZE} \key{ERROR} \imperativestatement \\
    \key{NOT} ON \key{SIZE} \key{ERROR} \imperativestatement
  \end{0+}

  \begin{0-1}
    \key{END-DIVIDE}
  \end{0-1}
\end{syntax}

\subsubsection{Syntax rules}

\subsubsection{General rules}

\section{ENTRY statement}

\begin{syntax}
  \miscext{
    \begin{minipage}[!h]{1.0\linewidth}
      \key{ENTRY} \literal

      \begin{0-1}
        \key{USING}

        \begin{1=}
          \begin{0-1}
            BY
            \begin{1=}
              \key{REFERENCE} \\
              \gnucobol{\key{CONTENT}} \\
              \key{VALUE}
            \end{1=}
          \end{0-1}

          \begin{1=}
            \gnucobol{\key{OMITTED}} \\

            \begin{1=}
              \gnucobol{
                \begin{0-1}
                  size-phrase
                \end{0-1}
              }
              \begin{1=}
                \identifier \\
                \literal
              \end{1=}
            \end{1=}
          \end{1=}\ldots
        \end{1=}
      \end{0-1}
    \end{minipage}
  }
\end{syntax}

\subsubsection{Syntax rules}

\subsubsection{General rules}

\section{EVALUATE statement}

\begin{syntax}
  \key{EVALUATE}
  \begin{1=}
    \expression \\
    \key{TRUE} \\
    \key{FALSE}
  \end{1=}
  \begin{0-1}
    \key{ALSO}
    \begin{1=}
      \expression \\
      \key{TRUE} \\
      \key{FALSE}
    \end{1=}
  \end{0-1} \ldots

  \begin{1=}
    \key{WHEN}
    \metaelement{selection-object}
    \begin{0-1}
      \key{ALSO} \metaelement{selection-object}
    \end{0-1}\ldots\ {}
    \imperativestatement
  \end{1=} \ldots

  \begin{0-1}
    \key{WHEN} \key{OTHER} \imperativestatement
  \end{0-1}

  \begin{0-1}
    \key{END-EVALUATE}
  \end{0-1}
\end{syntax}

where \metaelement{selection-object} is

\begin{syntax}
  \begin{1=}
    \metaelement{partial-expression-1}
    \begin{0-1}
      \begin{1=}
        \key{THROUGH} \\
        \key{THRU}
      \end{1=}
      \expression
    \end{0-1} \\

    \key{ANY} \\
    \key{TRUE} \\
    \key{FALSE}
  \end{1=}
\end{syntax}

\subsubsection{Syntax rules}

\subsubsection{General rules}

\section{EXIT statement}

\begin{syntax}
  \key{EXIT}
  \begin{0-1}
    \key{FUNCTION} \\
    \key{PARAGRAPH} \\

    \key{PERFORM}
    \begin{0-1}
      \key{CYCLE}
    \end{0-1} \\

    \key{PROGRAM}
    \miscext{
      \begin{0-1}
        \begin{1=}
          \key{RETURNING} \\
          \key{GIVING}
        \end{1=}
      \end{0-1}
      \begin{1=}
        \identifier \\
        \literal
      \end{1=}
    } \\

    \key{SECTION} \\
  \end{0-1}
\end{syntax}

\subsubsection{Syntax rules}

\subsubsection{General rules}

\section{FREE statement}

\begin{syntax}
  \key{FREE}
  \begin{1=}
    \identifier
  \end{1=} \ldots
\end{syntax}

\subsubsection{Syntax rules}

\subsubsection{General rules}

\section{GENERATE statement}

\begin{syntax}
  \pending{\key{GENERATE} \reportname}
\end{syntax}

\subsubsection{Syntax rules}

\subsubsection{General rules}

\section{GO TO statement}

\begin{syntax}
  \key{GO} TO
  \begin{1=}
    \procedurename
  \end{1=} \ldots
  \begin{0-1}
    \key{DEPENDING} ON \identifier
  \end{0-1}
\end{syntax}

\subsubsection{Syntax rules}

\subsubsection{General rules}

\section{GOBACK statement}

\begin{syntax}
  \key{GOBACK}
  \begin{0-1}
    \begin{1=}
      \key{RETURNING} \\
      \miscext{\key{GIVING}}
    \end{1=}
    \begin{1=}
      \identifier \\
      \literal
    \end{1=}
  \end{0-1}
\end{syntax}

\subsubsection{Syntax rules}

\subsubsection{General rules}

\section{IF statement}

\begin{syntax}
  \key{IF} \metaelement{condition} THEN
  \begin{1=}
    \imperativestatement \\
    \key{ELSE} \imperativestatement
  \end{1=} \ldots

  \begin{0-1}
    \key{END-IF}
  \end{0-1}
\end{syntax}

\subsubsection{Syntax rules}

\subsubsection{General rules}

\section{INITIALIZE statement}

\begin{syntax}
  \begin{1=}
    \key{INITIALIZE} \\
    \miscext{\key{INITIALISE}}
  \end{1=}
  \begin{1=}
    \identifier \\
    \metaelement{basic-literal-1}
  \end{1=} \ldots
  \begin{0-1}
    WITH \key{FILLER}
  \end{0-1}

  \begin{0-1}
    \begin{1=}
      \key{ALL} \\
      \key{ALPHABETIC} \\
      \key{ALPHANUMERIC} \\
      \key{ALPHANUMERIC-EDITED} \\
      \key{NATIONAL} \\
      \key{NATIONAL-EDITED} \\
      \key{NUMERIC} \\
      \key{NUMERIC-EDITED}
    \end{1=}
    TO \key{VALUE}
  \end{0-1}

  \begin{0-1}
    \key{REPLACING}
    \begin{1=}
      \begin{1=}
        \key{ALPHABETIC} \\
        \key{ALPHANUMERIC} \\
        \key{ALPHANUMERIC-EDITED} \\
        \key{NATIONAL} \\
        \key{NATIONAL-EDITED} \\
        \key{NUMERIC} \\
        \key{NUMERIC-EDITED}
      \end{1=}
      DATA \key{BY}
      \begin{1=}
        \identifier \\
        \literal
      \end{1=}
    \end{1=} \ldots
  \end{0-1}

  \begin{0-1}
    THEN TO \key{DEFAULT}
  \end{0-1}
\end{syntax}

\subsubsection{Syntax rules}

\subsubsection{General rules}

\section{INITIATE statement}
\begin{syntax}
  \pending{
    \key{INITIATE}
    \begin{1=}
      \reportname
    \end{1=} \ldots
  }
\end{syntax}

\subsubsection{Syntax rules}

\subsubsection{General rules}

\section{INSPECT statement}

\begin{syntax}
  \key{INSPECT}
  \begin{1=}
    \identifier \\
    \literal \\
    \functionname
  \end{1=}
  \begin{1=}
    \metaelement{tallying-phrase}
    \begin{0-1}
      \metaelement{replacing-phrase}
    \end{0-1} \\

    \metaelement{replacing-phrase} \\
    \metaelement{converting-phrase}
  \end{1=}
\end{syntax}

where \metaelement{tallying-phrase} is

\begin{syntax}
  \key{TALLYING}
  \begin{1=}
    \begin{1=}
      \begin{1=}
        \identifier \\
        \literal
      \end{1=}
      \key{FOR}
      \begin{1=}
        \key{CHARACTERS} \\

        \begin{1=}
          \key{ALL} \\
          \key{LEADING} \\
          \key{TRAILING}
        \end{1=}
        \begin{1=}
          \identifier \\
          \literal
        \end{1=}
      \end{1=}
    \end{1=}\ldots
    \begin{0-1}
      before-after-phrase
    \end{0-1}
  \end{1=} \ldots
\end{syntax}

where \metaelement{replacing-phrase} is

\begin{syntax}
  \key{REPLACING}
  \begin{1=}
    \begin{1=}
      \key{CHARACTERS} \\

      \begin{0-1}
        \key{ALL} \\
        \key{LEADING} \\
        \key{FIRST} \\
        \key{TRAILING}
      \end{0-1}
      \begin{1=}
        \identifier \\
        \literal
      \end{1=}
    \end{1=}
    \key{BY}
    \begin{1=}
      \identifier \\
      \literal \\
    \end{1=}
    \begin{0-1}
      \metaelement{before-after-phrase}
    \end{0-1} \\
  \end{1=} \ldots
\end{syntax}

where \metaelement{converting-phrase} is

\begin{syntax}
  \key{CONVERTING}
  \begin{1=}
    \identifier \\
    \literal
  \end{1=}
  \key{TO}
  \begin{1=}
    \identifier \\
    \literal
  \end{1=}
  \begin{0-1}
    \metaelement{before-after-phrase}
  \end{0-1}
\end{syntax}

where \metaelement{before-after-phrase} is

\begin{syntax}
  \begin{0+}
    \key{BEFORE} INITIAL
    \begin{1=}
      \identifier \\
      \literal
    \end{1=} \\

    \key{AFTER} INITIAL
    \begin{1=}
      \identifier \\
      \literal
    \end{1=}
  \end{0+}
\end{syntax}

\subsubsection{Syntax rules}

\subsubsection{General rules}

\section{MERGE statement}

\begin{syntax}
  \key{MERGE} \identifier
  \begin{0-1}
    ON
    \begin{1=}
      \key{ASCENDING} \\
      \key{DESCENDING}
    \end{1=}
    KEY
    \begin{0-1}
      \identifier
    \end{0-1}\ldots
  \end{0-1} \ldots

  \begin{0-1}
    WITH \key{DUPLICATES}
    \begin{0-1}
      IN \key{ORDER}
    \end{0-1}
  \end{0-1}

  \begin{0-1}
    COLLATING \key{SEQUENCE} IS \identifier
  \end{0-1}

  \begin{0-1}
    \key{USING}
    \begin{1=}
      \filename
    \end{1=}\ldots
  \end{0-1}

  \begin{0-1}
    \key{GIVING}
    \begin{1=}
      \filename
    \end{1=}\ldots \\

    \key{OUTPUT} \key{PROCEDURE} IS
    \procedurename
    \begin{0-1}
      \begin{1=}
        \key{THROUGH} \\
        \key{THRU}
      \end{1=}
      \procedurename
    \end{0-1}
  \end{0-1}
\end{syntax}

\subsubsection{Syntax rules}

\subsubsection{General rules}

\section{MOVE statement}

\begin{syntax}
  \key{MOVE}
  \begin{0-1}
    \key{CORRESPONDING} \\
    \key{CORR}
  \end{0-1}
  \begin{1=}
    \identifier \\
    \literal
  \end{1=}
  \key{TO}
  \begin{1=}
    \identifier
  \end{1=} \ldots
\end{syntax}

\subsubsection{Syntax rules}

\subsubsection{General rules}

\section{MULTIPLY statement}

\format{simple}
\begin{syntax}
  \key{MULTIPLY}
  \begin{1=}
    \identifier \\
    \literal
  \end{1=}
  \key{BY}
  \begin{1=}
    \begin{1=}
      \identifier \\
      \literal
    \end{1=}
    \begin{0-1}
      \metaelement{rounded-phrase}
    \end{0-1}
  \end{1=} \ldots

  \begin{0+}
    ON \key{SIZE} \key{ERROR} \imperativestatement \\
    \key{NOT} ON \key{SIZE} \key{ERROR} \imperativestatement
  \end{0+}

  \begin{0-1}
    \key{END-MULTIPLY}
  \end{0-1}
\end{syntax}

\format{giving}
\begin{syntax}
  \key{MULTIPLY}
  \begin{1=}
    \identifier \\
    \literal
  \end{1=}
  \key{BY}
  \begin{1=}
    \identifier \\
    \literal
  \end{1=}

  \key{GIVING}
  \begin{1=}
    \begin{1=}
      \identifier \\
      \literal
    \end{1=}
    \begin{0-1}
      \metaelement{rounded-phrase}
    \end{0-1}
  \end{1=} \ldots

  \begin{0+}
    ON \key{SIZE} \key{ERROR} \imperativestatement \\
    \key{NOT} ON \key{SIZE} \key{ERROR} \imperativestatement
  \end{0+}

  \begin{0-1}
    \key{END-MULTIPLY}
  \end{0-1}
\end{syntax}

\subsubsection{Syntax rules}

\subsubsection{General rules}

\section{NEXT SENTENCE statement}

\begin{syntax}
  \archaic{
    \key{NEXT} \key{SENTENCE}
  }
\end{syntax}

\subsubsection{Syntax rules}

\subsubsection{General rules}

\section{OPEN statement}

\begin{syntax}
  \key{OPEN}
  \begin{1=}
    \begin{1=}
      \key{INPUT} \\
      \key{OUTPUT} \\
      \key{I-O} \\
      \key{EXTEND}
    \end{1=}
    \begin{0-1}
      \metaelement{sharing-mode}
    \end{0-1}
    \begin{0-1}
      \metaelement{retry-phrase}
    \end{0-1}
    \begin{1=}
      \filename
    \end{1=} \ldots
    \begin{0-1}
      WITH \key{NO} \key{REWIND} \\
      WITH \key{LOCK} \\
      \deleted{\key{REVERSED}}
    \end{0-1}
  \end{1=}\ldots
\end{syntax}

where \metaelement{sharing-mode} is

\begin{syntax}
  \key{SHARING} WITH
  \begin{1=}
    \key{ALL} OTHER \\
    \key{NO} OTHER \\
    \key{READ} \key{ONLY}
  \end{1=}
\end{syntax}

\subsubsection{Syntax rules}

\subsubsection{General rules}

\section{PERFORM statement}

\format{procedure}
\begin{syntax}
  \key{PERFORM} \procedurename
  \begin{0-1}
    \begin{1=}
      \key{THROUGH} \\
      \key{THRU}
    \end{1=}
    \procedurename
  \end{0-1}
  \begin{0-1}
    \gnucobol{\key{FOREVER}} \\
    \metaelement{times-phrase} \\
    \metaelement{until-phrase} \\
    \metaelement{varying-phrase}
  \end{0-1}
\end{syntax}

\format{inline}
\begin{syntax}
  \key{PERFORM}
  \begin{0-1}
    \gnucobol{\key{FOREVER}} \\
    \metaelement{times-phrase} \\
    \metaelement{until-phrase} \\
    \metaelement{varying-phrase}
  \end{0-1}
  \imperativestatement
  \begin{0-1}
    \key{END-PERFORM}
  \end{0-1}
\end{syntax}

where \metaelement{times-phrase} is

\begin{syntax}
  \begin{1=}
    \identifier \\
    \literal \\
    \functionname
  \end{1=}
  \key{TIMES} \\
\end{syntax}

where \metaelement{until-phrase} is

\begin{syntax}
  \begin{0-1}
    WITH \key{TEST}
    \begin{1=}
      \key{BEFORE} \\
      \key{AFTER} \\
    \end{1=}
  \end{0-1}
  \key{UNTIL}
  \begin{1=}
    \condition \\
    \gnucobol{\key{EXIT}}
  \end{1=} \\
\end{syntax}

and where \metaelement{varying-phrase} is

\begin{syntax}
  \begin{0-1}
    WITH \key{TEST}
    \begin{1=}
      \key{BEFORE} \\
      \key{AFTER} \\
    \end{1=}
  \end{0-1}

  \key{VARYING} \identifier \key{FROM}
  \begin{1=}
    \identifier \\
    \literal
  \end{1=}
  \key{BY}
  \begin{1=}
    \identifier \\
    \literal
  \end{1=}
  \key{UNTIL}
  \condition

  \begin{0-1}
    \key{AFTER} \identifier \key{FROM}
    \begin{1=}
      \identifier \\
      \literal
    \end{1=}
    \key{BY}
    \begin{1=}
      \identifier \\
      \literal
    \end{1=}

    \key{UNTIL}
    \condition
  \end{0-1} \ldots
\end{syntax}

% TO-DO: Improve

\subsubsection{Syntax rules}

\subsubsection{General rules}

\section{READ statement}

\begin{syntax}
  \key{READ} \filename
  \begin{0-1}
    \key{NEXT} \\
    \key{PREVIOUS}
  \end{0-1}
  RECORD
  \begin{0-1}
    \key{INTO} \identifier
  \end{0-1}

  \begin{0-1}
    \begin{1=}
      \key{IGNORING} \key{LOCK} \\
      \miscext{WITH \key{IGNORE} \key{LOCK}}
    \end{1=} \\

    \begin{0-1}
      \pending{\key{ADVANCING} ON \key{LOCK}} \\
      \pending{\metaelement{retry-phrase}}
    \end{0-1}
    \begin{0-1}
      WITH
      \begin{1=}
        \begin{0-1}
          \key{NO} \\
          \miscext{\key{KEPT}}
        \end{0-1}
        \key{LOCK} \\

        \miscext{\key{WAIT}}
      \end{1=}
    \end{0-1}
  \end{0-1}

  \begin{0-1}
    \key{KEY} IS \identifier
  \end{0-1}

  \begin{0-1}
    \begin{1+}
      \key{INVALID} \key{KEY} \imperativestatement \\
      \key{NOT} \key{INVALID} \key{KEY} \imperativestatement
    \end{1+} \\

    \begin{1+}
      AT \key{END} \imperativestatement \\
      \key{NOT} AT \key{END} \imperativestatement
    \end{1+}
  \end{0-1}

  \begin{0-1}
    \key{END-READ}
  \end{0-1}
\end{syntax}

\subsubsection{Syntax rules}

\subsubsection{General rules}

\section{READY statement}

\begin{syntax}
  \miscext{\key{READY} \key{TRACE}}
\end{syntax}

\subsubsection{Syntax rules}

\subsubsection{General rules}

\section{RELEASE statement}

\begin{syntax}
  \key{RELEASE} \identifier
  \begin{0-1}
    \key{FROM}
    \begin{1=}
      \identifier \\
      \literal \\
      \metaelement{function-call-1}
    \end{1=}
  \end{0-1}
\end{syntax}

\subsubsection{Syntax rules}

\subsubsection{General rules}

\section{RESET statement}

\begin{syntax}
  \miscext{\key{RESET} \key{TRACE}}
\end{syntax}

\subsubsection{Syntax rules}

\subsubsection{General rules}

\section{RETURN statement}

\begin{syntax}
  \key{RETURN} \filename RECORD
  \begin{0-1}
    \key{INTO} \identifier
  \end{0-1}

  AT \key{END} \imperativestatement

  \begin{0-1}
    \key{NOT} AT \key{END} \imperativestatement
  \end{0-1}

  \begin{0-1}
    \key{END-RETURN}
  \end{0-1}
\end{syntax}

\subsubsection{Syntax rules}

\subsubsection{General rules}

\section{REWRITE statement}

\begin{syntax}
  \key{REWRITE}
  \recordname
  \begin{0-1}
    \key{FROM}
    \begin{1=}
      \identifier \\
      \literal \\
      \functionname
    \end{1=}
  \end{0-1} 
  \begin{0-1}
    \pending{\metaelement{retry-phrase}}
  \end{0-1}
  \begin{0-1}
    WITH
    \begin{0-1}
      \key{NO}
    \end{0-1}
    \key{LOCK}
  \end{0-1}

  \begin{0+}
    \key{INVALID} \key{KEY} \imperativestatement \\
    \key{NOT} \key{INVALID} \key{KEY} \imperativestatement
  \end{0+} \\

  \begin{0-1}
    \key{END-REWRITE}
  \end{0-1}
\end{syntax}

\subsubsection{Syntax rules}

\subsubsection{General rules}

\section{ROLLBACK statement}

\begin{syntax}
  \miscext{\key{ROLLBACK}}
\end{syntax}

\subsubsection{Syntax rules}

\subsubsection{General rules}

\section{SEARCH statement}

\format{simple}
\begin{syntax}
  \key{SEARCH} \identifier
  \begin{0-1}
    \key{VARYING} \identifier
  \end{0-1}

  \begin{0-1}
    AT \key{END} \imperativestatement
  \end{0-1}

  \begin{1=}
    \key{WHEN} \condition \imperativestatement
  \end{1=} \ldots

  \begin{0-1}
    \key{END-SEARCH}
  \end{0-1}
\end{syntax}

\format{all}
\begin{syntax}
  \key{SEARCH} \key{ALL} \identifier

  \begin{0-1}
    AT \key{END} \imperativestatement
  \end{0-1}

  \key{WHEN} \expression \imperativestatement

  \begin{0-1}
    \key{END-SEARCH}
  \end{0-1}
\end{syntax}

\subsubsection{Syntax rules}

\subsubsection{General rules}

\section{SET statement}

\format{simple}
\begin{syntax}
  \key{SET} \identifier \key{TO}
  \begin{1=}
    \identifier \\
    \literal \\
    \arithmeticexpression
  \end{1=}
\end{syntax}

\format{entry}
\begin{syntax}
  \gnucobol{
    \key{SET} \identifier \key{TO} \key{ENTRY}
    \begin{1=}
      \identifier \\
      \literal
    \end{1=}
  }
\end{syntax}

\format{environment}
\begin{syntax}
  \miscext{
    \key{SET} \key{ENVIRONMENT}
    \begin{1=}
      \identifier \\
      \literal
    \end{1=}
    \key{TO}
    \begin{1=}
      \identifier \\
      \literal
    \end{1=}
  }
\end{syntax}

\format{attribute}
\begin{syntax}
  \key{SET} \identifier \key{ATTRIBUTE}
  \begin{1=}
    \begin{1=}
      \begin{1=}
        \key{BELL} \\
        \key{BEEP}
      \end{1=} \\

      \key{BLINK} \\
      \key{HIGHLIGHT} \\
      \key{LOWLIGHT} \\
      \key{REVERSE-VIDEO} \\
      \key{UNDERLINE} \\
      \key{LEFTLINE} \\
      \key{OVERLINE}
    \end{1=}
    \begin{1=}
      \key{ON} \\
      \key{OFF}
    \end{1=}
  \end{1=}\ldots
\end{syntax}

\format{arithmetic}
\begin{syntax}
  \key{SET}
  \begin{1=}
    \cobolindexname
  \end{1=}\ldots
  \begin{1=}
    \key{UP} \\
    \key{DOWN}
  \end{1=}
  \key{BY}
  \arithmeticexpression
\end{syntax}

\format{on\slash{}off}
\begin{syntax}
  \key{SET}
  \begin{1=}
    \begin{1=}
      \mnemonicname
    \end{1=}\ldots
    \key{TO}
    \begin{1=}
      \key{ON} \\
      \key{OFF}
    \end{1=}
  \end{1=} \ldots
\end{syntax}

\format{true\slash{}false}
\begin{syntax}
  \key{SET}
  \begin{1=}
    \begin{1=}
      \conditionname
    \end{1=}\ldots
    \key{TO}
    \begin{1=}
      \key{TRUE} \\
      \key{FALSE}
    \end{1=}
  \end{1=} \ldots
\end{syntax}

\format{exception}
\begin{syntax}
  \key{SET} \key{LAST} \key{EXCEPTION} \key{TO} \key{OFF}
\end{syntax}

\subsubsection{Syntax rules}

\subsubsection{General rules}

\section{SORT statement}

\begin{syntax}
  \key{SORT} \identifier
  \begin{0-1}
    ON
    \begin{1=}
      \key{ASCENDING} \\
      \key{DESCENDING}
    \end{1=}
    KEY
    \begin{0-1}
      \identifier
    \end{0-1}\ldots
  \end{0-1} \ldots

  \begin{0-1}
    WITH \key{DUPLICATES}
    \begin{0-1}
      IN \key{ORDER}
    \end{0-1}
  \end{0-1}

  \begin{0-1}
    COLLATING \key{SEQUENCE} IS \identifier
  \end{0-1}

  \begin{0-1}
    \key{USING}
    \begin{1=}
      \filename
    \end{1=}\ldots \\

    \key{INPUT} \key{PROCEDURE} IS
    \procedurename
    \begin{0-1}
      \begin{1=}
        \key{THROUGH} \\
        \key{THRU}
      \end{1=}
      \procedurename
    \end{0-1}
  \end{0-1}

  \begin{0-1}
    \key{GIVING}
    \begin{1=}
      \filename
    \end{1=}\ldots \\

    \key{OUTPUT} \key{PROCEDURE} IS
    \procedurename
    \begin{0-1}
      \begin{1=}
        \key{THROUGH} \\
        \key{THRU}
      \end{1=}
      \procedurename
    \end{0-1}
  \end{0-1}
\end{syntax}

\subsubsection{Syntax rules}

\subsubsection{General rules}

\section{START statement}

\begin{syntax}
  \key{START} \filename
  \begin{0-1}
    \key{FIRST} \\

    \key{KEY} IS
    relational-operator
    \identifier\\

    \key{LAST}
  \end{0-1}

  \begin{0-1}
    WITH
    \begin{1=}
      \key{SIZE} \\
      \gnucobol{\key{LENGTH}}
    \end{1=}
    \arithmeticexpression
  \end{0-1}

  \begin{0+}
    \key{INVALID} KEY \imperativestatement \\
    \key{NOT} \key{INVALID} KEY \imperativestatement
  \end{0+}

  \begin{0-1}
    \key{END-START}
  \end{0-1}
\end{syntax}

\subsubsection{Syntax rules}

\subsubsection{General rules}

\section{STOP statement}

\format{standard}

\begin{syntax}
  \key{STOP} \key{RUN}
  \begin{0-1}
    \begin{1=}
      \key{RETURNING} \\
      \miscext{\key{GIVING}}
    \end{1=}
    \begin{1=}
      \identifier \\
      \literal
    \end{1=} \\

    WITH
    \begin{1=}
      \key{ERROR} \\
      \key{NORMAL}
    \end{1=}
    STATUS
    \begin{0-1}
      \identifier \\
      \literal
    \end{0-1}
  \end{0-1}
\end{syntax}

\format{literal}

\begin{syntax}
  \deleted{\key{STOP} \literal}
\end{syntax}

\format{ACUCOBOL}

\begin{syntax}
  \miscext{
    \key{STOP} \key{RUN}
    \begin{1=}
      \identifier \\
      \literal
    \end{1=}
  }
\end{syntax}

\subsubsection{Syntax rules}

\subsubsection{General rules}

\section{STRING statement}

\begin{syntax}
  \key{STRING}
  \begin{1=}
    \begin{1=}
      \identifier \\
      \literal
    \end{1=}

    \begin{0-1}
      \key{DELIMITED} BY
      \begin{1=}
        \key{SIZE} \\
        \identifier \\
        \literal
      \end{1=}
    \end{0-1}
  \end{1=} \ldots\ {}
  \key{INTO} \identifier

  \begin{0-1}
    WITH \key{POINTER} IS \identifier
  \end{0-1}

  \begin{0+}
    ON \key{OVERFLOW} \imperativestatement \\
    \key{NOT} ON \key{OVERFLOW} \imperativestatement
  \end{0+}
\end{syntax}

\subsubsection{Syntax rules}

\subsubsection{General rules}

\section{SUBTRACT statement}

\format{simple}
\begin{syntax}
  \key{SUBTRACT}
  \begin{1=}
    \identifier \\
    \literal
  \end{1=} \ldots
  \key{FROM}
  \begin{1=}
    \begin{1=}
      \identifier \\
      \literal
    \end{1=}
    \begin{0-1}
      \metaelement{rounded-phrase}
    \end{0-1}
  \end{1=} \ldots

  \begin{0+}
    ON \key{SIZE} \key{ERROR} \imperativestatement \\
    \key{NOT} ON \key{SIZE} \key{ERROR} \imperativestatement
  \end{0+}

  \begin{0-1}
    \key{END-SUBTRACT}
  \end{0-1}
\end{syntax}

\format{giving}
\begin{syntax}
  \key{SUBTRACT}
  \begin{1=}
    \identifier \\
    \literal
  \end{1=} \ldots
  \key{FROM}
  \begin{1=}
    \identifier \\
    \literal
  \end{1=}

  \key{GIVING}
  \begin{1=}
    \begin{1=}
      \identifier \\
      \literal
    \end{1=}
    \begin{0-1}
      \metaelement{rounded-phrase}
    \end{0-1}
  \end{1=} \ldots

  \begin{0+}
    ON \key{SIZE} \key{ERROR} \imperativestatement \\
    \key{NOT} ON \key{SIZE} \key{ERROR} \imperativestatement
  \end{0+}

  \begin{0-1}
    \key{END-SUBTRACT}
  \end{0-1}
\end{syntax}

\format{corresponding}
\begin{syntax}
  \key{SUBTRACT}
  \begin{1=}
    \key{CORR} \\
    \key{CORRESPONDING}
  \end{1=}
  \identifier{} \key{FROM} \identifier
  \begin{0-1}
    \metaelement{rounded-phrase}
  \end{0-1}

  \begin{0+}
    ON \key{SIZE} \key{ERROR} \imperativestatement \\
    \key{NOT} ON \key{SIZE} \key{ERROR} \imperativestatement
  \end{0+}

  \begin{0-1}
    \key{END-SUBTRACT}
  \end{0-1}
\end{syntax}

\subsubsection{Syntax rules}

\subsubsection{General rules}

\section{SUPPRESS statement}

\begin{syntax}
  \pending{
    \key{SUPPRESS} PRINTING
  }
\end{syntax}

\subsubsection{Syntax rules}

\subsubsection{General rules}

\section{TERMINATE statement}

\begin{syntax}
  \pending{
    \key{TERMINATE}
    \begin{1=}
      \reportname
    \end{1=} \ldots
  }
\end{syntax}

\subsubsection{Syntax rules}

\subsubsection{General rules}

\section{TRANSFORM statement}

\begin{syntax}
  \deleted{
    \key{TRANSFORM} \identifier \key{FROM}
    \begin{1=}
      \identifier \\
      \literal
    \end{1=}
    \key{TO}
    \begin{1=}
      \identifier \\
      \literal
    \end{1=}
  }
\end{syntax}

\subsubsection{Syntax rules}

\subsubsection{General rules}

\section{UNLOCK statement}

\begin{syntax}
  \key{UNLOCK} \filename
  \begin{0-1}
    \key{RECORD} \\
    \key{RECORDS}
  \end{0-1}
\end{syntax}

\subsubsection{Syntax rules}

\subsubsection{General rules}

\section{UNSTRING statement}

\begin{syntax}
  \key{UNSTRING} \identifier

  \begin{0-1}
    \key{DELIMITED} BY
    \begin{0-1}
      \key{ALL}
    \end{0-1}
    \begin{1=}
      \identifier \\
      \literal
    \end{1=}
    \begin{1=}
      \key{OR}
      \begin{0-1}
        \key{ALL}
      \end{0-1}
      \begin{1=}
        \identifier \\
        \literal
      \end{1=}
    \end{1=} \ldots
  \end{0-1}

  \key{INTO}
  \begin{1=}
    \identifier
    \begin{0-1}
      \key{DELIMITER} IN \identifier
    \end{0-1}
    \begin{0-1}
      \key{COUNT} IN \identifier
    \end{0-1}
  \end{1=} \ldots

  \begin{0-1}
    WITH \key{POINTER} IS \identifier
  \end{0-1}

  \begin{0-1}
    \key{TALLYING} IN \identifier
  \end{0-1}

  \begin{0+}
    ON \key{OVERFLOW} \imperativestatement \\
    \key{NOT} ON \key{OVERFLOW} \imperativestatement
  \end{0+}

  \begin{0-1}
    \key{END-OVERFLOW}
  \end{0-1}
\end{syntax}

\subsubsection{Syntax rules}

\subsubsection{General rules}

\section{USE statement}

\format{file exception}
\begin{syntax}
  \key{USE}
  \begin{0-1}
    \key{GLOBAL}
  \end{0-1}
  AFTER STANDARD
  \begin{1=}
    \key{EXCEPTION} \\
    \key{ERROR}
  \end{1=}
  PROCEDURE ON

  \begin{1=}
    \begin{1=}
      \filename
    \end{1=} \ldots
    \begin{0+}
      \key{INPUT} \\
      \key{OUTPUT} \\
      \key{I-O} \\
      \key{EXTEND}
    \end{0+} \ldots
  \end{1=}
\end{syntax}

\format{debugging}
\begin{syntax}
  \deleted{
    \key{USE} FOR \key{DEBUGGING} ON
    \begin{1=}
      \procedurename \\
      \key{ALL} \key{PROCEDURES} \\
      \key{ALL} REFERENCES OF \identifier
    \end{1=} \ldots
  }
\end{syntax}

\format{start\slash{}end}
\begin{syntax}
  \miscext{\pending{
      \key{USE} AT \key{PROGRAM}
      \begin{1=}
        \key{START} \\
        \key{END}
      \end{1=}
    }}
\end{syntax}

\format{reporting}
\begin{syntax}
  \key{USE}
  \begin{0-1}
    \key{GLOBAL}
  \end{0-1}
  \key{BEFORE} \key{REPORTING} \identifier
\end{syntax}

\format{exception}
\begin{syntax}
  \pending{
    \key{USE}
    \begin{1=}
      \key{EXCEPTION-CONDITION} \\
      \key{EC}
    \end{1=}
  }
\end{syntax}

\subsubsection{Syntax rules}

\subsubsection{General rules}

\section{WRITE statement}

\format{sequential}
\begin{syntax}
  \key{WRITE} \recordname
  \begin{0-1}
    \key{FROM}
    \begin{1=}
      \identifier \\
      \literal \\
      \functionname
    \end{1=}
  \end{0-1}

  \begin{0-1}
    \begin{1=}
      \key{BEFORE} \\
      \key{AFTER}
    \end{1=}
    ADVANCING
    \begin{1=}
      \begin{1=}
        \identifier \\
        \literal
      \end{1=}
      \begin{0-1}
        \key{LINE} \\
        \key{LINES}
      \end{0-1} \\

      \mnemonicname \\

      \key{PAGE}
    \end{1=}
  \end{0-1}

  \begin{0-1}
    \pending{\metaelement{retry-phrase}}
  \end{0-1}
  \begin{0-1}
    WITH
    \begin{0-1}
      \key{NO}
    \end{0-1}
    \key{LOCK}
  \end{0-1}

  \begin{0+}
    AT
    \begin{1=}
      \key{END-OF-PAGE} \\
      \key{EOP}
    \end{1=}
    \imperativestatement \\

    \key{NOT} AT
    \begin{1=}
      \key{END-OF-PAGE} \\
      \key{EOP}
    \end{1=}
    \imperativestatement
  \end{0+}

  \begin{0-1}
    \key{END-WRITE}
  \end{0-1}
\end{syntax}

\format{random}
\begin{syntax}
  \key{WRITE} \recordname
  \begin{0-1}
    \key{FROM}
    \begin{1=}
      \identifier \\
      \literal \\
      \functionname
    \end{1=}
  \end{0-1}

  \begin{0-1}
    \begin{1=}
      \key{BEFORE} \\
      \key{AFTER}
    \end{1=}
    ADVANCING
    \begin{1=}
      \begin{1=}
        \identifier \\
        \literal
      \end{1=}
      \begin{0-1}
        \key{LINE} \\
        \key{LINES}
      \end{0-1} \\

      \mnemonicname \\

      \key{PAGE}
    \end{1=}
  \end{0-1}

  \begin{0-1}
    \pending{\metaelement{retry-phrase}}
  \end{0-1}
  \begin{0-1}
    WITH
    \begin{0-1}
      \key{NO}
    \end{0-1}
    \key{LOCK}
  \end{0-1}

  \begin{0+}
    \key{INVALID} \key{KEY} \imperativestatement \\
    \key{NOT} \key{INVALID} \key{KEY} \imperativestatement
  \end{0+} \\

  \begin{0-1}
    \key{END-WRITE}
  \end{0-1}
\end{syntax}

\subsubsection{Syntax rules}

\subsubsection{General rules}


%%% Local Variables:
%%% mode: latex
%%% TeX-master: "grammar.tex"
%%% End:

\chapter{Intrinsic functions}

\section{ABS function}

\begin{syntax}
  \key{FUNCTION} \key{ABS} ( \argument )
\end{syntax}

\subsubsection{Syntax rules}

\subsubsection{General rules}

\section{ACOS function}

\begin{syntax}
  \key{FUNCTION} \key{ACOS} ( \argument)
\end{syntax}

\subsubsection{Syntax rules}

\subsubsection{General rules}

\section{ANNUITY function}

\begin{syntax}
  \key{FUNCTION} \key{ANNUITY} ( \argument \argument )
\end{syntax}

\subsubsection{Syntax rules}

\subsubsection{General rules}

\section{ASIN function}

\begin{syntax}
  \key{FUNCTION} \key{ASIN} ( \argument )
\end{syntax}

\subsubsection{Syntax rules}

\subsubsection{General rules}

\section{ATAN function}

\begin{syntax}
  \key{FUNCTION} \key{ATAN} ( \argument )
\end{syntax}

\subsubsection{Syntax rules}

\subsubsection{General rules}

\section{BOOLEAN-OF-INTEGER function}

\begin{syntax}
  \pending{
    \key{FUNCTION} \key{BOOLEAN-OF-INTEGER} ( \argument \argument )
  }
\end{syntax}

\subsubsection{Syntax rules}

\subsubsection{General rules}

\section{BYTE-LENGTH function}

\begin{syntax}
  \key{FUNCTION} \key{BYTE-LENGTH} ( \argument )
\end{syntax}

\subsubsection{Syntax rules}

\subsubsection{General rules}

\section{CHAR function}

\begin{syntax}
  \key{FUNCTION} \key{CHAR} ( \argument )
\end{syntax}

\subsubsection{Syntax rules}

\subsubsection{General rules}

\section{CHAR-NATIONAL function}

\begin{syntax}
  \pending{
    \key{FUNCTION} \key{CHAR-NATIONAL} ( \argument )
  }
\end{syntax}

\subsubsection{Syntax rules}

\subsubsection{General rules}

\section{COMBINED-DATETIME function}

\begin{syntax}
  \key{FUNCTION} \key{COMBINED-DATETIME} ( \argument \argument )
\end{syntax}

\subsubsection{Syntax rules}

\subsubsection{General rules}

\section{CONCATENATE function}

\begin{syntax}
  \key{FUNCTION} \key{CONCATENATE} (
  \begin{1=}
    \argument
  \end{1=}
  \ldots
  \ {})
\end{syntax}

\subsubsection{Syntax rules}

\subsubsection{General rules}

\section{COS function}

\begin{syntax}
  \key{FUNCTION} \key{COS} ( \argument )
\end{syntax}

\subsubsection{Syntax rules}

\subsubsection{General rules}

\section{CURRENCY-SYMBOL function}

\begin{syntax}
  \gnucobol{
    \key{FUNCTION} \key{CURRENCY-SYMBOL}
  }
\end{syntax}

\subsubsection{Syntax rules}

\subsubsection{General rules}

\section{CURRENT-DATE function}

\begin{syntax}
  \key{FUNCTION} \key{CURRENT-DATE}
\end{syntax}

\subsubsection{Syntax rules}

\subsubsection{General rules}

\section{DATE-OF-INTEGER function}

\begin{syntax}
  \key{FUNCTION} \key{DATE-OF-INTEGER} ( \argument )
\end{syntax}

\subsubsection{Syntax rules}

\subsubsection{General rules}

\section{DATE-TO-YYYYMMDD function}

\begin{syntax}
  \key{FUNCTION} \key{DATE-TO-YYYYMMDD} (
  \argument
  \begin{0-1}
    \argument
    \begin{0-1}
      \argument
    \end{0-1}
  \end{0-1}
  \ {})
\end{syntax}

\subsubsection{Syntax rules}

\subsubsection{General rules}

\section{DAY-OF-INTEGER function}

\begin{syntax}
  \key{FUNCTION} \key{DAY-OF-INTEGER} ( \argument )
\end{syntax}

\subsubsection{Syntax rules}

\subsubsection{General rules}

\section{DAY-TO-YYYYDDD function}

\begin{syntax}
  \key{FUNCTION} \key{DAY-TO-YYYYDDD} (
  \argument
  \begin{0-1}
    \argument
    \begin{0-1}
      \argument
    \end{0-1}
  \end{0-1}
  \ {})
\end{syntax}

\subsubsection{Syntax rules}

\subsubsection{General rules}

\section{DISPLAY-OF function}

\begin{syntax}
  \pending{
    \key{FUNCTION} \key{DISPLAY-OF} ( \argument )
  }
\end{syntax}

\subsubsection{Syntax rules}

\subsubsection{General rules}

\section{E function}

\begin{syntax}
  \key{FUNCTION} \key{E}
\end{syntax}

\subsubsection{Syntax rules}

\subsubsection{General rules}

\section{EXCEPTION-FILE function}

\begin{syntax}
  \key{FUNCTION} \key{EXCEPTION-FILE}
\end{syntax}

\subsubsection{Syntax rules}

\subsubsection{General rules}

\section{EXCEPTION-FILE-N function}

\begin{syntax}
  \pending{
    \key{FUNCTION} \key{EXCEPTION-FILE-N}
  }
\end{syntax}

\subsubsection{Syntax rules}

\subsubsection{General rules}

\section{EXCEPTION-LOCATION function}

\begin{syntax}
  \key{FUNCTION} \key{EXCEPTION-LOCATION}
\end{syntax}

\subsubsection{Syntax rules}

\subsubsection{General rules}

\section{EXCEPTION-LOCATION-N function}

\begin{syntax}
  \pending{
    \key{FUNCTION} \key{EXCEPTION-LOCATION-N}
  }
\end{syntax}

\subsubsection{Syntax rules}

\subsubsection{General rules}

\section{EXCEPTION-STATEMENT function}

\begin{syntax}
  \key{FUNCTION} \key{EXCEPTION-STATEMENT}
\end{syntax}

\subsubsection{Syntax rules}

\subsubsection{General rules}

\section{EXCEPTION-STATUS function}

\begin{syntax}
  \key{FUNCTION} \key{EXCEPTION-STATUS}
\end{syntax}

\subsubsection{Syntax rules}

\subsubsection{General rules}

\section{EXP function}

\begin{syntax}
  \key{FUNCTION} \key{EXP} ( \argument )
\end{syntax}

\subsubsection{Syntax rules}

\subsubsection{General rules}

\section{EXP10 function}

\begin{syntax}
  \key{FUNCTION} \key{EXP10} ( \argument )
\end{syntax}

\subsubsection{Syntax rules}

\subsubsection{General rules}

\section{FACTORIAL function}

\begin{syntax}
  \key{FUNCTION} \key{FACTORIAL} ( \argument )
\end{syntax}

\subsubsection{Syntax rules}

\subsubsection{General rules}

\section{FORMATTED-CURRENT-DATE function}

\begin{syntax}
  \key{FUNCTION} \key{FORMATTED-CURRENT-DATE} ( \argument )
\end{syntax}

\subsubsection{Syntax rules}

\subsubsection{General rules}

\section{FORMATTED-DATE function}

\begin{syntax}
  \key{FUNCTION} \key{FORMATTED-DATE} ( \argument \argument)
\end{syntax}

\subsubsection{Syntax rules}

\subsubsection{General rules}

\section{FORMATTED-DATETIME function}

\begin{syntax}
  \key{FUNCTION} \key{FORMATTED-DATETIME}

  ( \argument \argument \argument
  \begin{0-1}
    \argument \\
    \gnucobol{\key{SYSTEM-OFFSET}}
  \end{0-1}
  )
\end{syntax}

\subsubsection{Syntax rules}

\subsubsection{General rules}

\section{FORMATTED-TIME function}

\begin{syntax}
  \key{FUNCTION} \key{FORMATTED-TIME} ( \argument \argument
  \begin{0-1}
    \argument \\
    \gnucobol{\key{SYSTEM-OFFSET}}
  \end{0-1}
  )
\end{syntax}

\subsubsection{Syntax rules}

\subsubsection{General rules}

\section{FRACTION-PART function}

\begin{syntax}
  \key{FUNCTION} \key{FRACTION-PART} ( \argument )
\end{syntax}

\subsubsection{Syntax rules}

\subsubsection{General rules}

\section{HIGHEST-ALGEBRAIC function}

\begin{syntax}
  \key{FUNCTION} \key{HIGHEST-ALGEBRAIC} ( \argument )
\end{syntax}

\subsubsection{Syntax rules}

\subsubsection{General rules}

\section{INTEGER function}

\begin{syntax}
  \key{FUNCTION} \key{INTEGER} ( \argument )
\end{syntax}

\subsubsection{Syntax rules}

\subsubsection{General rules}

\section{INTEGER-OF-BOOLEAN function}

\begin{syntax}
  \pending{
    \key{FUNCTION} \key{INTEGER-OF-BOOLEAN} ( \argument )
  }
\end{syntax}

\subsubsection{Syntax rules}

\subsubsection{General rules}

\section{INTEGER-OF-DATE function}

\begin{syntax}
  \key{FUNCTION} \key{INTEGER-OF-DATE} ( \argument )
\end{syntax}

\subsubsection{Syntax rules}

\subsubsection{General rules}

\section{INTEGER-OF-DAY function}

\begin{syntax}
  \key{FUNCTION} \key{INTEGER-OF-DAY} ( \argument )
\end{syntax}

\subsubsection{Syntax rules}

\subsubsection{General rules}

\section{INTEGER-OF-FORMATTED-DATE function}

\begin{syntax}
  \key{FUNCTION} \key{INTEGER-OF-FORMATTED-DATE} ( \argument \argument )
\end{syntax}

\subsubsection{Syntax rules}

\subsubsection{General rules}

\section{INTEGER-PART function}

\begin{syntax}
  \key{FUNCTION} \key{INTEGER-PART} ( \argument )
\end{syntax}

\subsubsection{Syntax rules}

\subsubsection{General rules}

\section{LENGTH function}

\begin{syntax}
  \key{FUNCTION} \key{LENGTH} ( \argument )
\end{syntax}

\subsubsection{Syntax rules}

\subsubsection{General rules}

\section{LENGTH-AN function}

\begin{syntax}
  \miscext{
    \key{FUNCTION} \key{LENGTH-AN} ( \argument )
  }
\end{syntax}

\subsubsection{Syntax rules}

\subsubsection{General rules}

\section{LOCALE-COMPARE function}

\begin{syntax}
  \key{FUNCTION} \key{LOCALE-COMPARE} ( \argument \argument
  \begin{0-1}
    \argument
  \end{0-1}
  )
\end{syntax}

\subsubsection{Syntax rules}

\subsubsection{General rules}

\section{LOCALE-DATE function}

\begin{syntax}
  \key{FUNCTION} \key{LOCALE-DATE} ( \argument
  \begin{0-1}
    \argument
  \end{0-1}
  )
\end{syntax}

\subsubsection{Syntax rules}

\subsubsection{General rules}

\section{LOCALE-TIME function}

\begin{syntax}
  \key{FUNCTION} \key{LOCALE-TIME} ( \argument
  \begin{0-1}
    \argument
  \end{0-1}
  )
\end{syntax}

\subsubsection{Syntax rules}

\subsubsection{General rules}

\section{LOCALE-TIME-FROM-SECONDS function}

\begin{syntax}
  \key{FUNCTION} \key{LOCALE-TIME-FROM-SECONDS} ( \argument
  \begin{0-1}
    \argument
  \end{0-1}
  )
\end{syntax}

\subsubsection{Syntax rules}

\subsubsection{General rules}

\section{LOG function}

\begin{syntax}
  \key{FUNCTION} \key{LOG} ( \argument )
\end{syntax}

\subsubsection{Syntax rules}

\subsubsection{General rules}

\section{LOG10 function}

\begin{syntax}
  \key{FUNCTION} \key{LOG10} ( \argument )
\end{syntax}

\subsubsection{Syntax rules}

\subsubsection{General rules}

\section{LOWER-CASE function}

\begin{syntax}
  \key{FUNCTION} \key{LOWER-CASE} ( \argument )
\end{syntax}

\subsubsection{Syntax rules}

\subsubsection{General rules}

\section{LOWEST-ALGEBRAIC function}

\begin{syntax}
  \key{FUNCTION} \key{LOWEST-ALGEBRAIC} ( \argument )
\end{syntax}

\subsubsection{Syntax rules}

\subsubsection{General rules}

\section{MAX function}

\begin{syntax}
  \key{FUNCTION} \key{MAX} (
  \begin{1=}
    \argument
  \end{1=}\ldots
  \ {})
\end{syntax}

\subsubsection{Syntax rules}

\subsubsection{General rules}

\section{MEAN function}

\begin{syntax}
  \key{FUNCTION} \key{MEAN} (
  \begin{1=}
    \argument
  \end{1=}\ldots
  \ {})
\end{syntax}

\subsubsection{Syntax rules}

\subsubsection{General rules}

\section{MEDIAN function}

\begin{syntax}
  \key{FUNCTION} \key{MEDIAN} (
  \begin{1=}
    \argument
  \end{1=}\ldots
  \ {})
\end{syntax}

\subsubsection{Syntax rules}

\subsubsection{General rules}

\section{MIDRANGE function}

\begin{syntax}
  \key{FUNCTION} \key{MIDRANGE} (
  \begin{1=}
    \argument
  \end{1=}\ldots
  \ {})
\end{syntax}

\subsubsection{Syntax rules}

\subsubsection{General rules}

\section{MIN function}

\begin{syntax}
  \key{FUNCTION} \key{MIN} (
  \begin{1=}
    \argument
  \end{1=}\ldots
  \ {})
\end{syntax}

\subsubsection{Syntax rules}

\subsubsection{General rules}

\section{MOD function}

\begin{syntax}
  \key{FUNCTION} \key{MOD} ( \argument \argument )
\end{syntax}

\subsubsection{Syntax rules}

\subsubsection{General rules}

\section{MODULE-CALLER-ID function}

\begin{syntax}
  \gnucobol{
    \key{FUNCTION} \key{MODULE-CALLER-ID}
  }
\end{syntax}

\subsubsection{Syntax rules}

\subsubsection{General rules}

\section{MODULE-DATE function}

\begin{syntax}
  \gnucobol{
    \key{FUNCTION} \key{MODULE-DATE}
  }
\end{syntax}

\subsubsection{Syntax rules}

\subsubsection{General rules}

\section{MODULE-FORMATTED-DATE function}

\begin{syntax}
  \gnucobol{
    \key{FUNCTION} \key{MODULE-FORMATTED-DATE}
  }
\end{syntax}

\subsubsection{Syntax rules}

\subsubsection{General rules}

\section{MODULE-ID function}

\begin{syntax}
  \gnucobol{
    \key{FUNCTION} \key{MODULE-ID}
  }
\end{syntax}

\subsubsection{Syntax rules}

\subsubsection{General rules}

\section{MODULE-PATH function}

\begin{syntax}
  \gnucobol{
    \key{FUNCTION} \key{MODULE-PATH}
  }
\end{syntax}

\subsubsection{Syntax rules}

\subsubsection{General rules}

\section{MODULE-SOURCE function}

\begin{syntax}
  \gnucobol{
    \key{FUNCTION} \key{MODULE-SOURCE}
  }
\end{syntax}

\subsubsection{Syntax rules}

\subsubsection{General rules}

\section{MODULE-TIME function}

\begin{syntax}
  \gnucobol{
    \key{FUNCTION} \key{MODULE-TIME}
  }
\end{syntax}

\subsubsection{Syntax rules}

\subsubsection{General rules}

\section{MONETARY-DECIMAL-POINT function}

\begin{syntax}
  \gnucobol{
    \key{FUNCTION} \key{MONETARY-DECIMAL-POINT}
  }
\end{syntax}

\subsubsection{Syntax rules}

\subsubsection{General rules}

\section{MONETARY-THOUSANDS-SEPARATOR function}

\begin{syntax}
  \gnucobol{
    \key{FUNCTION} \key{MONETARY-THOUSANDS-SEPARATOR}
  }
\end{syntax}

\subsubsection{Syntax rules}

\subsubsection{General rules}

\section{NATIONAL-OF function}

\begin{syntax}
  \pending{
    \key{FUNCTION} \key{NATIONAL-OF} ( \argument
    \begin{0-1}
      \argument
    \end{0-1}
    )
  }
\end{syntax}

\subsubsection{Syntax rules}

\subsubsection{General rules}

\section{NUMERIC-DECIMAL-POINT function}

\begin{syntax}
  \gnucobol{
    \key{FUNCTION} \key{NUMERIC-DECIMAL-POINT}
  }
\end{syntax}

\subsubsection{Syntax rules}

\subsubsection{General rules}

\section{NUMERIC-THOUSANDS-SEPARATOR function}

\begin{syntax}
  \gnucobol{
    \key{FUNCTION} \key{NUMERIC-THOUSANDS-SEPARATOR}
  }
\end{syntax}

\subsubsection{Syntax rules}

\subsubsection{General rules}

\section{NUMVAL function}

\begin{syntax}
  \key{FUNCTION} \key{NUMVAL} ( \argument )
\end{syntax}

\subsubsection{Syntax rules}

\subsubsection{General rules}

\section{NUMVAL-C function}

\begin{syntax}
  \key{FUNCTION} \key{NUMVAL-C} ( \argument
  \begin{0-1}
    \argument
  \end{0-1}
  )
\end{syntax}

\subsubsection{Syntax rules}

\subsubsection{General rules}

\section{NUMVAL-F function}

\begin{syntax}
  \key{FUNCTION} \key{NUMVAL-F} ( \argument )
\end{syntax}

\subsubsection{Syntax rules}

\subsubsection{General rules}

\section{ORD function}

\begin{syntax}
  \key{FUNCTION} \key{ORD} ( \argument )
\end{syntax}

\subsubsection{Syntax rules}

\subsubsection{General rules}

\section{ORD-MAX function}

\begin{syntax}
  \key{FUNCTION} \key{ORD-MAX} (
  \begin{1=}
    \argument
  \end{1=} \ldots
  \ {})
\end{syntax}

\subsubsection{Syntax rules}

\subsubsection{General rules}

\section{ORD-MIN function}

\begin{syntax}
  \key{FUNCTION} \key{ORD-MIN} (
  \begin{1=}
    \argument
  \end{1=} \ldots
  \ {})
\end{syntax}

\subsubsection{Syntax rules}

\subsubsection{General rules}

\section{PI function}

\begin{syntax}
  \key{FUNCTION} \key{PI}
\end{syntax}

\subsubsection{Syntax rules}

\subsubsection{General rules}

\section{PRESENT-VALUE function}

\begin{syntax}
  \key{FUNCTION} \key{PRESENT-VALUE} (
  \begin{1=}
    \argument
  \end{1=} \ldots
  \ {})
\end{syntax}

\subsubsection{Syntax rules}

\subsubsection{General rules}

\section{RANDOM function}

\begin{syntax}
  \key{FUNCTION} \key{RANDOM}
  \begin{0-1}
    (
    \begin{0-1}
      \argument
    \end{0-1} \gnucobol{\ldots}\ {}
    )
  \end{0-1}
\end{syntax}

\subsubsection{Syntax rules}

\subsubsection{General rules}

\section{RANGE function}

\begin{syntax}
  \key{FUNCTION} \key{RANGE} (
  \begin{1=}
    \argument
  \end{1=}\ldots
  \ {})
\end{syntax}

\subsubsection{Syntax rules}

\subsubsection{General rules}

\section{REM function}

\begin{syntax}
  \key{FUNCTION} \key{REM} ( \argument \argument )
\end{syntax}

\subsubsection{Syntax rules}

\subsubsection{General rules}

\section{REVERSE function}

\begin{syntax}
  \key{FUNCTION} \key{REVERSE} ( \argument )
\end{syntax}

\subsubsection{Syntax rules}

\subsubsection{General rules}

\section{SECONDS-FROM-FORMATTED-TIME function}

\begin{syntax}
  \key{FUNCTION} \key{SECONDS-FROM-FORMATTED-TIME} ( \argument \argument )
\end{syntax}

\subsubsection{Syntax rules}

\subsubsection{General rules}

\section{SECONDS-PAST-MIDNIGHT function}

\begin{syntax}
  \key{FUNCTION} \key{SECONDS-PAST-MIDNIGHT} ( \argument )
\end{syntax}

\subsubsection{Syntax rules}

\subsubsection{General rules}

\section{SIGN function}

\begin{syntax}
  \key{FUNCTION} \key{SIGN} ( \argument )
\end{syntax}

\subsubsection{Syntax rules}

\subsubsection{General rules}

\section{SIN function}

\begin{syntax}
  \key{FUNCTION} \key{SIN} ( \argument )
\end{syntax}

\subsubsection{Syntax rules}

\subsubsection{General rules}

\section{SQRT function}

\begin{syntax}
  \key{FUNCTION} \key{SQRT} ( \argument )
\end{syntax}

\subsubsection{Syntax rules}

\subsubsection{General rules}

\section{STANDARD-COMPARE function}

\begin{syntax}
  \pending{
    \key{FUNCTION} \key{STANDARD-COMPARE}
  }

  \pending{
    ( \argument \argument
    \begin{0-1}
      \argument
    \end{0-1}
    \begin{0-1}
      \argument
    \end{0-1}
    )
  }
\end{syntax}

\subsubsection{Syntax rules}

\subsubsection{General rules}

\section{STANDARD-DEVIATION function}

\begin{syntax}
  \key{FUNCTION} \key{STANDARD-DEVIATION} (
  \begin{1=}
    \argument
  \end{1=}\ldots
  \ {})
\end{syntax}

\subsubsection{Syntax rules}

\subsubsection{General rules}

\section{STORED-CHAR-LENGTH function}

\begin{syntax}
  \gnucobol{
    \key{FUNCTION} \key{STORED-CHAR-LENGTH} ( \argument )
  }
\end{syntax}

\subsubsection{Syntax rules}

\subsubsection{General rules}

\section{SUBSTITUTE function}

\begin{syntax}
  \gnucobol{
    \key{FUNCTION} \key{SUBSTITUTE} ( \argument
    \begin{1=}
      \argument \argument
    \end{1=}\ldots\ {}
    )
  }
\end{syntax}

\subsubsection{Syntax rules}

\subsubsection{General rules}

\section{SUBSTITUTE-CASE function}

\begin{syntax}
  \gnucobol{
    \key{FUNCTION} \key{SUBSTITUTE-CASE} ( \argument
    \begin{1=}
      \argument \argument
    \end{1=}\ldots\ {}
    )
  }
\end{syntax}

\subsubsection{Syntax rules}

\subsubsection{General rules}

\section{SUM function}

\begin{syntax}
  \key{FUNCTION} \key{SUM} (
  \begin{1=}
    \argument
  \end{1=}\ldots
  \ {})
\end{syntax}

\subsubsection{Syntax rules}

\subsubsection{General rules}

\section{TAN function}

\begin{syntax}
  \key{FUNCTION} \key{TAN} ( \argument )
\end{syntax}

\subsubsection{Syntax rules}

\subsubsection{General rules}

\section{TEST-DATE-YYYYMMDD function}

\begin{syntax}
  \key{FUNCTION} \key{TEST-DATE-YYYYMMDD} ( \argument )
\end{syntax}

\subsubsection{Syntax rules}

\subsubsection{General rules}

\section{TEST-DAY-YYYYDDD function}

\begin{syntax}
  \key{FUNCTION} \key{TEST-DAY-YYYYDDD} ( \argument )
\end{syntax}

\subsubsection{Syntax rules}

\subsubsection{General rules}

\section{TEST-FORMATTED-DATETIME function}

\begin{syntax}
  \key{FUNCTION} \key{TEST-FORMATTED-DATETIME} ( \argument \argument )
\end{syntax}

\subsubsection{Syntax rules}

\subsubsection{General rules}

\section{TEST-NUMVAL function}

\begin{syntax}
  \key{FUNCTION} \key{TEST-NUMVAL} ( \argument )
\end{syntax}

\subsubsection{Syntax rules}

\subsubsection{General rules}

\section{TEST-NUMVAL-C function}

\begin{syntax}
  \key{FUNCTION} \key{TEST-NUMVAL-C} ( \argument \argument )
\end{syntax}

\subsubsection{Syntax rules}

\subsubsection{General rules}

\section{TEST-NUMVAL-F function}

\begin{syntax}
  \key{FUNCTION} \key{TEST-NUMVAL-F} ( \argument )
\end{syntax}

\subsubsection{Syntax rules}

\subsubsection{General rules}

\section{TRIM function}

\begin{syntax}
  \key{FUNCTION} \key{TRIM} ( \argument
  \begin{0-1}
    \key{LEADING} \\
    \key{TRAILING}
  \end{0-1}
  )
\end{syntax}

\subsubsection{Syntax rules}

\subsubsection{General rules}

\section{UPPER-CASE function}

\begin{syntax}
  \key{FUNCTION} \key{UPPER-CASE} ( \argument )
\end{syntax}

\subsubsection{Syntax rules}

\subsubsection{General rules}

\section{VARIANCE function}

\begin{syntax}
  \key{FUNCTION} \key{VARIANCE} (
  \begin{1=}
    \argument
  \end{1=}\ldots
  \ {})
\end{syntax}

\subsubsection{Syntax rules}

\subsubsection{General rules}

\section{WHEN-COMPILED function}

\begin{syntax}
  \key{FUNCTION} \key{WHEN-COMPILED}
\end{syntax}

\subsubsection{Syntax rules}

\subsubsection{General rules}

\section{YEAR-TO-YYYY function}

\begin{syntax}
  \key{FUNCTION} \key{YEAR-TO-YYYY} ( \argument
  \begin{0-1}
    \argument
    \begin{0-1}
      \argument
    \end{0-1}
  \end{0-1}
  )
\end{syntax}

\subsubsection{Syntax rules}

\subsubsection{General rules}

%%% Local Variables:
%%% mode: latex
%%% TeX-master: "grammar.tex"
%%% End:


\begin{appendices}
  \input{gfdl}
\end{appendices}

\end{document}
