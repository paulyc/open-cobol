\chapter{Intrinsic functions}

\section{ABS function}

\begin{syntax}
  \key{FUNCTION}
  \begin{1=}
    \key{ABS} \\
    \miscext{\key{ABSOLUTE-VALUE}}
  \end{1=}
  ( \argument )
\end{syntax}

\subsubsection{Syntax rules}

\subsubsection{General rules}

\section{ACOS function}

\begin{syntax}
  \key{FUNCTION} \key{ACOS} ( \argument)
\end{syntax}

\subsubsection{Syntax rules}

\subsubsection{General rules}

\section{ANNUITY function}

\begin{syntax}
  \key{FUNCTION} \key{ANNUITY} ( \argument \argument )
\end{syntax}

\subsubsection{Syntax rules}

\subsubsection{General rules}

\section{ASIN function}

\begin{syntax}
  \key{FUNCTION} \key{ASIN} ( \argument )
\end{syntax}

\subsubsection{Syntax rules}

\subsubsection{General rules}

\section{ATAN function}

\begin{syntax}
  \key{FUNCTION} \key{ATAN} ( \argument )
\end{syntax}

\subsubsection{Syntax rules}

\subsubsection{General rules}

\section{BOOLEAN-OF-INTEGER function}

\begin{syntax}
  \pending{
    \key{FUNCTION} \key{BOOLEAN-OF-INTEGER} ( \argument \argument )
  }
\end{syntax}

\subsubsection{Syntax rules}

\subsubsection{General rules}

\section{BYTE-LENGTH function}

\begin{syntax}
  \key{FUNCTION} \key{BYTE-LENGTH} ( \argument )
\end{syntax}

\subsubsection{Syntax rules}

\subsubsection{General rules}

\section{CHAR function}

\begin{syntax}
  \key{FUNCTION} \key{CHAR} ( \argument )
\end{syntax}

\subsubsection{Syntax rules}

\subsubsection{General rules}

\section{CHAR-NATIONAL function}

\begin{syntax}
  \pending{
    \key{FUNCTION} \key{CHAR-NATIONAL} ( \argument )
  }
\end{syntax}

\subsubsection{Syntax rules}

\subsubsection{General rules}

\section{COMBINED-DATETIME function}

\begin{syntax}
  \key{FUNCTION} \key{COMBINED-DATETIME} ( \argument \argument )
\end{syntax}

\subsubsection{Syntax rules}

\subsubsection{General rules}

\section{CONCATENATE function}

\begin{syntax}
  \key{FUNCTION} \key{CONCATENATE} (
  \begin{1=}
    \argument
  \end{1=}
  \ldots
  \ {})
\end{syntax}

\subsubsection{Syntax rules}

\subsubsection{General rules}

\section{CONTENT-LENGTH function}

\begin{syntax}[\gnucobolcolour]
  \key{FUNCTION} \key{CONTENT-LENGTH} ( \argument )
\end{syntax}

\subsubsection{Syntax rules}

\subsubsection{General rules}

\section{CONTENT-OF function}

\begin{syntax}[\gnucobolcolour]
  \key{FUNCTION} \key{CONTENT-OF} ( \argument
  \begin{0-1}
    \argument
  \end{0-1}
  \ {})
\end{syntax}

\subsubsection{Syntax rules}

\subsubsection{General rules}

\section{COS function}

\begin{syntax}
  \key{FUNCTION} \key{COS} ( \argument )
\end{syntax}

\subsubsection{Syntax rules}

\subsubsection{General rules}

\section{CURRENCY-SYMBOL function}

\begin{syntax}[\gnucobolcolour]
  \key{FUNCTION} \key{CURRENCY-SYMBOL}
\end{syntax}

\subsubsection{Syntax rules}

\subsubsection{General rules}

\section{CURRENT-DATE function}

\begin{syntax}
  \key{FUNCTION} \key{CURRENT-DATE}
\end{syntax}

\subsubsection{Syntax rules}

\subsubsection{General rules}

\section{DATE-OF-INTEGER function}

\begin{syntax}
  \key{FUNCTION} \key{DATE-OF-INTEGER} ( \argument )
\end{syntax}

\subsubsection{Syntax rules}

\subsubsection{General rules}

\section{DATE-TO-YYYYMMDD function}

\begin{syntax}
  \key{FUNCTION} \key{DATE-TO-YYYYMMDD} (
  \argument
  \begin{0-1}
    \argument
    \begin{0-1}
      \argument
    \end{0-1}
  \end{0-1}
  \ {})
\end{syntax}

\subsubsection{Syntax rules}

\subsubsection{General rules}

\section{DAY-OF-INTEGER function}

\begin{syntax}
  \key{FUNCTION} \key{DAY-OF-INTEGER} ( \argument )
\end{syntax}

\subsubsection{Syntax rules}

\subsubsection{General rules}

\section{DAY-TO-YYYYDDD function}

\begin{syntax}
  \key{FUNCTION} \key{DAY-TO-YYYYDDD} (
  \argument
  \begin{0-1}
    \argument
    \begin{0-1}
      \argument
    \end{0-1}
  \end{0-1}
  \ {})
\end{syntax}

\subsubsection{Syntax rules}

\subsubsection{General rules}

\section{DISPLAY-OF function}

\begin{syntax}
  \pending{
    \key{FUNCTION} \key{DISPLAY-OF} ( \argument )
  }
\end{syntax}

\subsubsection{Syntax rules}

\subsubsection{General rules}

\section{E function}

\begin{syntax}
  \key{FUNCTION} \key{E}
\end{syntax}

\subsubsection{Syntax rules}

\subsubsection{General rules}

\section{EXCEPTION-FILE function}

\begin{syntax}
  \key{FUNCTION} \key{EXCEPTION-FILE}
\end{syntax}

\subsubsection{Syntax rules}

\subsubsection{General rules}

\section{EXCEPTION-FILE-N function}

\begin{syntax}
  \pending{
    \key{FUNCTION} \key{EXCEPTION-FILE-N}
  }
\end{syntax}

\subsubsection{Syntax rules}

\subsubsection{General rules}

\section{EXCEPTION-LOCATION function}

\begin{syntax}
  \key{FUNCTION} \key{EXCEPTION-LOCATION}
\end{syntax}

\subsubsection{Syntax rules}

\subsubsection{General rules}

\section{EXCEPTION-LOCATION-N function}

\begin{syntax}
  \pending{
    \key{FUNCTION} \key{EXCEPTION-LOCATION-N}
  }
\end{syntax}

\subsubsection{Syntax rules}

\subsubsection{General rules}

\section{EXCEPTION-STATEMENT function}

\begin{syntax}
  \key{FUNCTION} \key{EXCEPTION-STATEMENT}
\end{syntax}

\subsubsection{Syntax rules}

\subsubsection{General rules}

\section{EXCEPTION-STATUS function}

\begin{syntax}
  \key{FUNCTION} \key{EXCEPTION-STATUS}
\end{syntax}

\subsubsection{Syntax rules}

\subsubsection{General rules}

\section{EXP function}

\begin{syntax}
  \key{FUNCTION} \key{EXP} ( \argument )
\end{syntax}

\subsubsection{Syntax rules}

\subsubsection{General rules}

\section{EXP10 function}

\begin{syntax}
  \key{FUNCTION} \key{EXP10} ( \argument )
\end{syntax}

\subsubsection{Syntax rules}

\subsubsection{General rules}

\section{FACTORIAL function}

\begin{syntax}
  \key{FUNCTION} \key{FACTORIAL} ( \argument )
\end{syntax}

\subsubsection{Syntax rules}

\subsubsection{General rules}

\section{FORMATTED-CURRENT-DATE function}

\begin{syntax}
  \key{FUNCTION} \key{FORMATTED-CURRENT-DATE} ( \argument )
\end{syntax}

\subsubsection{Syntax rules}

\subsubsection{General rules}

\section{FORMATTED-DATE function}

\begin{syntax}
  \key{FUNCTION} \key{FORMATTED-DATE} ( \argument \argument)
\end{syntax}

\subsubsection{Syntax rules}

\subsubsection{General rules}

\section{FORMATTED-DATETIME function}

\begin{syntax}
  \key{FUNCTION} \key{FORMATTED-DATETIME}

  ( \argument \argument \argument
  \begin{0-1}
    \argument \\
    \gnucobol{\key{SYSTEM-OFFSET}}
  \end{0-1}
  )
\end{syntax}

\subsubsection{Syntax rules}

\subsubsection{General rules}

\section{FORMATTED-TIME function}

\begin{syntax}
  \key{FUNCTION} \key{FORMATTED-TIME} ( \argument \argument
  \begin{0-1}
    \argument \\
    \gnucobol{\key{SYSTEM-OFFSET}}
  \end{0-1}
  )
\end{syntax}

\subsubsection{Syntax rules}

\subsubsection{General rules}

\section{FRACTION-PART function}

\begin{syntax}
  \key{FUNCTION} \key{FRACTION-PART} ( \argument )
\end{syntax}

\subsubsection{Syntax rules}

\subsubsection{General rules}

\section{HIGHEST-ALGEBRAIC function}

\begin{syntax}
  \key{FUNCTION} \key{HIGHEST-ALGEBRAIC} ( \argument )
\end{syntax}

\subsubsection{Syntax rules}

\subsubsection{General rules}

\section{INTEGER function}

\begin{syntax}
  \key{FUNCTION} \key{INTEGER} ( \argument )
\end{syntax}

\subsubsection{Syntax rules}

\subsubsection{General rules}

\section{INTEGER-OF-BOOLEAN function}

\begin{syntax}
  \pending{
    \key{FUNCTION} \key{INTEGER-OF-BOOLEAN} ( \argument )
  }
\end{syntax}

\subsubsection{Syntax rules}

\subsubsection{General rules}

\section{INTEGER-OF-DATE function}

\begin{syntax}
  \key{FUNCTION} \key{INTEGER-OF-DATE} ( \argument )
\end{syntax}

\subsubsection{Syntax rules}

\subsubsection{General rules}

\section{INTEGER-OF-DAY function}

\begin{syntax}
  \key{FUNCTION} \key{INTEGER-OF-DAY} ( \argument )
\end{syntax}

\subsubsection{Syntax rules}

\subsubsection{General rules}

\section{INTEGER-OF-FORMATTED-DATE function}

\begin{syntax}
  \key{FUNCTION} \key{INTEGER-OF-FORMATTED-DATE} ( \argument \argument )
\end{syntax}

\subsubsection{Syntax rules}

\subsubsection{General rules}

\section{INTEGER-PART function}

\begin{syntax}
  \key{FUNCTION} \key{INTEGER-PART} ( \argument )
\end{syntax}

\subsubsection{Syntax rules}

\subsubsection{General rules}

\section{LENGTH function}

\begin{syntax}
  \key{FUNCTION} \key{LENGTH} ( \argument
  \pending{
    \begin{0-1}
      \key{PHYSICAL}
    \end{0-1}
  }
  )
\end{syntax}

\subsubsection{Syntax rules}

\subsubsection{General rules}

\section{LENGTH-AN function}

\begin{syntax}[\miscextcolour]
  \key{FUNCTION} \key{LENGTH-AN} ( \argument )
\end{syntax}

\subsubsection{Syntax rules}

\subsubsection{General rules}

\section{LOCALE-COMPARE function}

\begin{syntax}
  \key{FUNCTION} \key{LOCALE-COMPARE} ( \argument \argument
  \begin{0-1}
    \argument
  \end{0-1}
  )
\end{syntax}

\subsubsection{Syntax rules}

\subsubsection{General rules}

\section{LOCALE-DATE function}

\begin{syntax}
  \key{FUNCTION} \key{LOCALE-DATE} ( \argument
  \begin{0-1}
    \argument
  \end{0-1}
  )
\end{syntax}

\subsubsection{Syntax rules}

\subsubsection{General rules}

\section{LOCALE-TIME function}

\begin{syntax}
  \key{FUNCTION} \key{LOCALE-TIME} ( \argument
  \begin{0-1}
    \argument
  \end{0-1}
  )
\end{syntax}

\subsubsection{Syntax rules}

\subsubsection{General rules}

\section{LOCALE-TIME-FROM-SECONDS function}

\begin{syntax}
  \key{FUNCTION} \key{LOCALE-TIME-FROM-SECONDS} ( \argument
  \begin{0-1}
    \argument
  \end{0-1}
  )
\end{syntax}

\subsubsection{Syntax rules}

\subsubsection{General rules}

\section{LOG function}

\begin{syntax}
  \key{FUNCTION} \key{LOG} ( \argument )
\end{syntax}

\subsubsection{Syntax rules}

\subsubsection{General rules}

\section{LOG10 function}

\begin{syntax}
  \key{FUNCTION} \key{LOG10} ( \argument )
\end{syntax}

\subsubsection{Syntax rules}

\subsubsection{General rules}

\section{LOWER-CASE function}

\begin{syntax}
  \key{FUNCTION} \key{LOWER-CASE} ( \argument )
\end{syntax}

\subsubsection{Syntax rules}

\subsubsection{General rules}

\section{LOWEST-ALGEBRAIC function}

\begin{syntax}
  \key{FUNCTION} \key{LOWEST-ALGEBRAIC} ( \argument )
\end{syntax}

\subsubsection{Syntax rules}

\subsubsection{General rules}

\section{MAX function}

\begin{syntax}
  \key{FUNCTION} \key{MAX} (
  \begin{1=}
    \argument
  \end{1=}\ldots
  \ {})
\end{syntax}

\subsubsection{Syntax rules}

\subsubsection{General rules}

\section{MEAN function}

\begin{syntax}
  \key{FUNCTION} \key{MEAN} (
  \begin{1=}
    \argument
  \end{1=}\ldots
  \ {})
\end{syntax}

\subsubsection{Syntax rules}

\subsubsection{General rules}

\section{MEDIAN function}

\begin{syntax}
  \key{FUNCTION} \key{MEDIAN} (
  \begin{1=}
    \argument
  \end{1=}\ldots
  \ {})
\end{syntax}

\subsubsection{Syntax rules}

\subsubsection{General rules}

\section{MIDRANGE function}

\begin{syntax}
  \key{FUNCTION} \key{MIDRANGE} (
  \begin{1=}
    \argument
  \end{1=}\ldots
  \ {})
\end{syntax}

\subsubsection{Syntax rules}

\subsubsection{General rules}

\section{MIN function}

\begin{syntax}
  \key{FUNCTION} \key{MIN} (
  \begin{1=}
    \argument
  \end{1=}\ldots
  \ {})
\end{syntax}

\subsubsection{Syntax rules}

\subsubsection{General rules}

\section{MOD function}

\begin{syntax}
  \key{FUNCTION} \key{MOD} ( \argument \argument )
\end{syntax}

\subsubsection{Syntax rules}

\subsubsection{General rules}

\section{MODULE-CALLER-ID function}

\begin{syntax}[\gnucobolcolour]
  \key{FUNCTION} \key{MODULE-CALLER-ID}
\end{syntax}

\subsubsection{Syntax rules}

\subsubsection{General rules}

\section{MODULE-DATE function}

\begin{syntax}[\gnucobolcolour]
  \key{FUNCTION} \key{MODULE-DATE}
\end{syntax}

\subsubsection{Syntax rules}

\subsubsection{General rules}

\section{MODULE-FORMATTED-DATE function}

\begin{syntax}[\gnucobolcolour]
  \key{FUNCTION} \key{MODULE-FORMATTED-DATE}
\end{syntax}

\subsubsection{Syntax rules}

\subsubsection{General rules}

\section{MODULE-ID function}

\begin{syntax}[\gnucobolcolour]
  \key{FUNCTION} \key{MODULE-ID}
\end{syntax}

\subsubsection{Syntax rules}

\subsubsection{General rules}

\section{MODULE-PATH function}

\begin{syntax}[\gnucobolcolour]
  \key{FUNCTION} \key{MODULE-PATH}
\end{syntax}

\subsubsection{Syntax rules}

\subsubsection{General rules}

\section{MODULE-SOURCE function}

\begin{syntax}[\gnucobolcolour]
  \key{FUNCTION} \key{MODULE-SOURCE}
\end{syntax}

\subsubsection{Syntax rules}

\subsubsection{General rules}

\section{MODULE-TIME function}

\begin{syntax}[\gnucobolcolour]
  \key{FUNCTION} \key{MODULE-TIME}
\end{syntax}

\subsubsection{Syntax rules}

\subsubsection{General rules}

\section{MONETARY-DECIMAL-POINT function}

\begin{syntax}[\gnucobolcolour]
  \key{FUNCTION} \key{MONETARY-DECIMAL-POINT}
\end{syntax}

\subsubsection{Syntax rules}

\subsubsection{General rules}

\section{MONETARY-THOUSANDS-SEPARATOR function}

\begin{syntax}[\gnucobolcolour]
  \key{FUNCTION} \key{MONETARY-THOUSANDS-SEPARATOR}
\end{syntax}

\subsubsection{Syntax rules}

\subsubsection{General rules}

\section{NATIONAL-OF function}

\begin{syntax}
  \pending{
    \key{FUNCTION} \key{NATIONAL-OF} ( \argument
    \begin{0-1}
      \argument
    \end{0-1}
    )
  }
\end{syntax}

\subsubsection{Syntax rules}

\subsubsection{General rules}

\section{NUMERIC-DECIMAL-POINT function}

\begin{syntax}[\gnucobolcolour]
  \key{FUNCTION} \key{NUMERIC-DECIMAL-POINT}
\end{syntax}

\subsubsection{Syntax rules}

\subsubsection{General rules}

\section{NUMERIC-THOUSANDS-SEPARATOR function}

\begin{syntax}[\gnucobolcolour]
  \key{FUNCTION} \key{NUMERIC-THOUSANDS-SEPARATOR}
\end{syntax}

\subsubsection{Syntax rules}

\subsubsection{General rules}

\section{NUMVAL function}

\begin{syntax}
  \key{FUNCTION} \key{NUMVAL} ( \argument )
\end{syntax}

\subsubsection{Syntax rules}

\subsubsection{General rules}

\section{NUMVAL-C function}

\begin{syntax}
  \key{FUNCTION} \key{NUMVAL-C} ( \argument
  \begin{0-1}
    \argument
  \end{0-1}
  )
\end{syntax}

\subsubsection{Syntax rules}

\subsubsection{General rules}

\section{NUMVAL-F function}

\begin{syntax}
  \key{FUNCTION} \key{NUMVAL-F} ( \argument )
\end{syntax}

\subsubsection{Syntax rules}

\subsubsection{General rules}

\section{ORD function}

\begin{syntax}
  \key{FUNCTION} \key{ORD} ( \argument )
\end{syntax}

\subsubsection{Syntax rules}

\subsubsection{General rules}

\section{ORD-MAX function}

\begin{syntax}
  \key{FUNCTION} \key{ORD-MAX} (
  \begin{1=}
    \argument
  \end{1=} \ldots
  \ {})
\end{syntax}

\subsubsection{Syntax rules}

\subsubsection{General rules}

\section{ORD-MIN function}

\begin{syntax}
  \key{FUNCTION} \key{ORD-MIN} (
  \begin{1=}
    \argument
  \end{1=} \ldots
  \ {})
\end{syntax}

\subsubsection{Syntax rules}

\subsubsection{General rules}

\section{PI function}

\begin{syntax}
  \key{FUNCTION} \key{PI}
\end{syntax}

\subsubsection{Syntax rules}

\subsubsection{General rules}

\section{PRESENT-VALUE function}

\begin{syntax}
  \key{FUNCTION} \key{PRESENT-VALUE} (
  \argument \argument
  \begin{0-1}
    \argument
  \end{0-1} \ldots
  \ {})
\end{syntax}

\subsubsection{Syntax rules}

\subsubsection{General rules}

\section{RANDOM function}

\begin{syntax}
  \key{FUNCTION} \key{RANDOM} (
  \begin{0-1}
    \argument
  \end{0-1}
  )
\end{syntax}

\subsubsection{Syntax rules}

\subsubsection{General rules}

\section{RANGE function}

\begin{syntax}
  \key{FUNCTION} \key{RANGE} (
  \begin{1=}
    \argument
  \end{1=}\ldots
  \ {})
\end{syntax}

\subsubsection{Syntax rules}

\subsubsection{General rules}

\section{REM function}

\begin{syntax}
  \key{FUNCTION} \key{REM} ( \argument \argument )
\end{syntax}

\subsubsection{Syntax rules}

\subsubsection{General rules}

\section{REVERSE function}

\begin{syntax}
  \key{FUNCTION} \key{REVERSE} ( \argument )
\end{syntax}

\subsubsection{Syntax rules}

\subsubsection{General rules}

\section{SECONDS-FROM-FORMATTED-TIME function}

\begin{syntax}
  \key{FUNCTION} \key{SECONDS-FROM-FORMATTED-TIME} ( \argument \argument )
\end{syntax}

\subsubsection{Syntax rules}

\subsubsection{General rules}

\section{SECONDS-PAST-MIDNIGHT function}

\begin{syntax}
  \key{FUNCTION} \key{SECONDS-PAST-MIDNIGHT} ( \argument )
\end{syntax}

\subsubsection{Syntax rules}

\subsubsection{General rules}

\section{SIGN function}

\begin{syntax}
  \key{FUNCTION} \key{SIGN} ( \argument )
\end{syntax}

\subsubsection{Syntax rules}

\subsubsection{General rules}

\section{SIN function}

\begin{syntax}
  \key{FUNCTION} \key{SIN} ( \argument )
\end{syntax}

\subsubsection{Syntax rules}

\subsubsection{General rules}

\section{SQRT function}

\begin{syntax}
  \key{FUNCTION} \key{SQRT} ( \argument )
\end{syntax}

\subsubsection{Syntax rules}

\subsubsection{General rules}

\section{STANDARD-COMPARE function}

\begin{syntax}
  \pending{
    \key{FUNCTION} \key{STANDARD-COMPARE}
  }

  \pending{
    ( \argument \argument
    \begin{0-1}
      \argument
    \end{0-1}
    \begin{0-1}
      \argument
    \end{0-1}
    )
  }
\end{syntax}

\subsubsection{Syntax rules}

\subsubsection{General rules}

\section{STANDARD-DEVIATION function}

\begin{syntax}
  \key{FUNCTION} \key{STANDARD-DEVIATION} (
  \begin{1=}
    \argument
  \end{1=}\ldots
  \ {})
\end{syntax}

\subsubsection{Syntax rules}

\subsubsection{General rules}

\section{STORED-CHAR-LENGTH function}

\begin{syntax}[\gnucobolcolour]
  \key{FUNCTION} \key{STORED-CHAR-LENGTH} ( \argument )
\end{syntax}

\subsubsection{Syntax rules}

\subsubsection{General rules}

\section{SUBSTITUTE function}

\begin{syntax}[\gnucobolcolour]
  \key{FUNCTION} \key{SUBSTITUTE} ( \argument
  \begin{1=}
    \argument \argument
  \end{1=}\ldots\ {}
  )
\end{syntax}

\subsubsection{Syntax rules}

\subsubsection{General rules}

\section{SUBSTITUTE-CASE function}

\begin{syntax}[\gnucobolcolour]
  \key{FUNCTION} \key{SUBSTITUTE-CASE} ( \argument
  \begin{1=}
    \argument \argument
  \end{1=}\ldots\ {}
  )
\end{syntax}

\subsubsection{Syntax rules}

\subsubsection{General rules}

\section{SUM function}

\begin{syntax}
  \key{FUNCTION} \key{SUM} (
  \begin{1=}
    \argument
  \end{1=}\ldots
  \ {})
\end{syntax}

\subsubsection{Syntax rules}

\subsubsection{General rules}

\section{TAN function}

\begin{syntax}
  \key{FUNCTION} \key{TAN} ( \argument )
\end{syntax}

\subsubsection{Syntax rules}

\subsubsection{General rules}

\section{TEST-DATE-YYYYMMDD function}

\begin{syntax}
  \key{FUNCTION} \key{TEST-DATE-YYYYMMDD} ( \argument )
\end{syntax}

\subsubsection{Syntax rules}

\subsubsection{General rules}

\section{TEST-DAY-YYYYDDD function}

\begin{syntax}
  \key{FUNCTION} \key{TEST-DAY-YYYYDDD} ( \argument )
\end{syntax}

\subsubsection{Syntax rules}

\subsubsection{General rules}

\section{TEST-FORMATTED-DATETIME function}

\begin{syntax}
  \key{FUNCTION} \key{TEST-FORMATTED-DATETIME} ( \argument \argument )
\end{syntax}

\subsubsection{Syntax rules}

\subsubsection{General rules}

\section{TEST-NUMVAL function}

\begin{syntax}
  \key{FUNCTION} \key{TEST-NUMVAL} ( \argument )
\end{syntax}

\subsubsection{Syntax rules}

\subsubsection{General rules}

\section{TEST-NUMVAL-C function}

\begin{syntax}
  \key{FUNCTION} \key{TEST-NUMVAL-C} ( \argument \argument )
\end{syntax}

\subsubsection{Syntax rules}

\subsubsection{General rules}

\section{TEST-NUMVAL-F function}

\begin{syntax}
  \key{FUNCTION} \key{TEST-NUMVAL-F} ( \argument )
\end{syntax}

\subsubsection{Syntax rules}

\subsubsection{General rules}

\section{TRIM function}

\begin{syntax}
  \key{FUNCTION} \key{TRIM} ( \argument
  \begin{0-1}
    \key{LEADING} \\
    \key{TRAILING}
  \end{0-1}
  )
\end{syntax}

\subsubsection{Syntax rules}

\subsubsection{General rules}

\section{UPPER-CASE function}

\begin{syntax}
  \key{FUNCTION} \key{UPPER-CASE} ( \argument )
\end{syntax}

\subsubsection{Syntax rules}

\subsubsection{General rules}

\section{VARIANCE function}

\begin{syntax}
  \key{FUNCTION} \key{VARIANCE} (
  \begin{1=}
    \argument
  \end{1=}\ldots
  \ {})
\end{syntax}

\subsubsection{Syntax rules}

\subsubsection{General rules}

\section{WHEN-COMPILED function}

\begin{syntax}
  \key{FUNCTION} \key{WHEN-COMPILED}
\end{syntax}

\subsubsection{Syntax rules}

\subsubsection{General rules}

\section{YEAR-TO-YYYY function}

\begin{syntax}
  \key{FUNCTION} \key{YEAR-TO-YYYY} ( \argument
  \begin{0-1}
    \argument
    \begin{0-1}
      \argument
    \end{0-1}
  \end{0-1}
  )
\end{syntax}

\subsubsection{Syntax rules}

\subsubsection{General rules}

%%% Local Variables:
%%% mode: latex
%%% TeX-master: "grammar.tex"
%%% End:
